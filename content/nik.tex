% !TEX root = ../main.tex
% !TEX program = XeLaTeX
% !TEX encoding = utf8
% !TEX spellcheck = el_EL-ModernGreek

\section{Σημειώσεις Ανδρόνικου}

 Ο 
Βιτρούβιος είναι γνωστός για το έργο του "Περί αρχιτεκτονικής" (De 
architectura), ένα σύνολο συγγραμμάτων, βιβλίων τα οποία περιγράφουν τη δομή 
της αρχιτεκτονικής.**

Τα επαγγέλματα των μηχανικών κανονικά μεταφέρονταν από γονιό σε παιδί.  *** 
Ο Αύγουστος χρησιμοποίησε σαν ευκαιρία την παρουσία των νέων ατόμων, ειδικών 
για να αυξήσει την ήδη υψηλή πολιτική και στρατιωτική του δύναμη. Ο Βιτρούβιος 
ήταν ένα από αυτά τα άτομα. (Status, Pay, Pleasure p.1) ***

και στη περίπτωση του Βιτρούβιου οι γονείς φρόντισαν για την εκπαίδευση του σαν 
αρχιτέκτονας. Ο Βιτρούβιος μάλιστα κάνει αναφορά στο σύγγραμμά του στους 
δασκάλους του που τον εκπαίδευσαν. (Λέφας σελ 13).

Στη περίοδο που έζησε, της ύστερης Ρωμαϊκής δημοκρατίας, οι αρχιτέκτονες 
όφειλαν να υποστηρίζουν το λαϊκό κόμμα του Καίσαρα και του κλώδιου (Λέφας σελ 
13). Συμπεραίνεται πως ο Βιτρούβιος άσκησε το επάγγελμα του αρχιτέκτονα στο 
στρατό του αυτοκράτορα Καίσαρα με ιδιότητα ως τεχνικός στρατιωτικών μηχανών 
όπου συμπεραίνουμε ότι ο ίδιος είχε ακολουθήσει τον στρατό του Καίσαρα σε 
διάφορες μετακινήσεις από τμήματα της Γαλλίας μέχρι την Αφρική. Οι αναμνήσεις 
και η εμπειρίας από τα ταξίδια του μεταφέρονται και στο έργο του. φαίνεται πως 
ο ίδος ήαν ένας "aparitor", δηλαδή ένας δημόσιος υπάλληλος του οποίου ο μισθός 
προερχόταν από το δημόσιο ταμέιο.

Η εποχή που έζησε ο Βιτρούβιος είχε να κάνει με την αλλαγή. Μία εποχή όπου οι 
τέχνες και τα επαγγέλματα είχαν πια αναβαθμιστεί, κερδίζοντας νέα σημασία σε 
σχέση με την παράδοση. Η κλασική ανατροφή, όπου είχε θρέψει την κλασική παλιά 
ρωμαϊκή κουλτούρα δεν ήταν πια σχετική. Η επαγγελματική εκπαίδευση είχε πια 
ξεπεράσει την κλασική. Βρήκαν πάτημα στην ελληνιστική παράδωση. έτσι και η 
αρχιτεκτονική. (Vitruvius and the liberal arts of architecture)

Με σκοπό να βρεθεί μία λύση στην αναγκαία αυτή αλλαγή από την κλασική ρωμαϊκή 
κουλτούρα, σε μία νέα, σκεπτικιστές υποστήριξαν πως μία σύνθεση μεταξύ 
ελληνιστικού και ρωμαϊκού ήταν αυτό που χρειάζονταν. Προέβλεψαν πως μέσω μιας 
σωστής εκπαίδευσης θα επιτυγχανόταν αυτό. Και βρήκαν αυτό που έψαχναν στη 
εγκύκλιος παιδεία, τις ελευθέριες τέχνες της ελληνιστικής περιόδου. Άτομα τα 
οποία το είπαν αυτή ήταν ο Κικέρον, ο Βάρρος και ο Βιτρούβιος. Αυτό που είχαν 
στο νου τους ήταν μία γενική απόκτηση γνώσης για έναν επιστήμονα δεν είχε να 
κάνει (απαραίτητα) με την επαγγελματική του κατάρτιση πάνω στη λογική, φυσική 
και ηθική, δηλαδή να γίνει φιλόσοφος. Αντί αυτού στο νου τους είχαν μια γενική 
απόκτηση γνώσεων και πειθαρχίας. (Vitruvius and the liberal arts of 
arcitecture)

\subsection{Το έργο του} 

Ο Βιτρούβιος είναι γνωστός για την συγγραφή του "Περί αρχιτεκτονικής" (De 
architectura), ένα σύνολο από δέκα συγγράμματα όπου αναλύουν και περιγράφουν 
τις δομές της αρχιτεκτονικής και την αξία του αρχιτέκτονα.
Το σύγγραμμα ολοκληρώθηκε και δημοσιεύτηκε κατά την τελευταία περίοδο ζωής του 
αρχιτέκτονα και πιθανότατα κατά τη περίοδο όπου αυτοκράτορας ήταν πια ο 
Αύγουστος, υιοθετημένος ιός του Καίσαρα. ***
Τα βιβλία αποτελούνται από τεκμηριωμένες και ολοκληρωμένες σκέψεις και 
παρατηρήσεις του συγγραφέα τα οποία είναι αφιερωμένα στον ίδιο τον Αυτοκράτορα. 
Οι λόγοι που τον οδήγησαν να συγγράψει το περί αρχιτεκτονικής είναι μάλλον προς 
παροχή υποστήριξης στο οικοδομικό πρόγραμμα του Αύγουστου και στην ενημέρωση 
πάνω στην αρχιτεκτονική. (Λέφας σελ 14-15)***

Το περιεχόμενο αφορά τους τομείς του αρχιτέκτονα (1.2-3),την εκπαίδευση και τη 
συμπεριφορά του (1.6), και τα ορθά χαρακτηριστικά των δημοσίων και ιδιωτικών 
κτισμάτων (βοοκ 5-6), Υλικά (Book 2), ορθές τοποθεσίες κτηρίων (1.4-7). 
(Status, Pay, Pleasure p.6)

Στον πρόλογο του πρώτου βιβλίου αποκαλύπτει το κύριο στόχο του, είδη 
συλλογισμών πάνω στην αρχιτεκτονική. Στην αρχή του πρώτου βιβλίου συζητάει την 
γνώση του αρχιτέκτονα. έπειτα στο δεύτερο κεφάλαιο του πρώτου βιβλίου αναλύει 
την θεωρητική δομή της αρχιτεκτονικής. Από το τρίτο κεφάλαιο και μετά αφιερώνει 
το υπόλοιπο έργο του στο πρακτικό κομμάτι της αρχιτεκτονικής. (vitruvious arts 
of architectue p.2)

Το πιο σημαντικό ίσως τμήμα του συγγράμματος είναι η απαιτούμενη γνώση που 
οφείλει να έχει ένας αρχιτέκτονας.

\subsection{Η εκπαίδευση του αρχιτέκτονα}

\subsection{πρακτική και θεωρητηκή. Fabrica \& Ratiocinatio}

Στο Βιβλίο 1. 1 παράγραφος 2, ο Βιτρούβιος αναφέρει πως άτομα τα οποία 
βασίστηκαν αποκλειστικά πάνω στη πρακτική εξάσκηση ή  στα γράμματα και τη 
θεωρία δεν κατάφεραν να επιτύχουν και να προσφέρουν ένα σωστό κτίσμα. 
Αντιθέτως, άτομα τα οποία ήταν οχυρωμένα και με πρακτική και θεωρητική θεωρία 
κατάφεραν να επιτύχουν. *** (Λέγφας σελ37 1.1 παρ 2)

Αυτό που ξεχωρίζει τον αρχιτέκτονα από ένα τυπικό τεχνίτη είναι η υπεροχή στα 
γράμματα. Φυσικά η ιδιότητα του τεχνίτη είναι εξίσου σημαντική. ένας 
αρχιτέκτονας πρέπει να κατέχει και τις δύο κατηγορίες. Συνθετική υπεροχή 
(Status, Pay, Pleasure p. 9)

Ο Βιτρούβιος απαιτεί από τον αρχιτέκτονα να κατέχει μία πληθώρα γνώσεων, από 
γεωμετρία και φιλοσοφία μέχρι και ιατρική. Τονίζει βέβαια ότι δεν απαιτείται η 
υπεροχή στους τομείς αυτούς αλλά μία τουλάχιστον ενασχόληση και εξοικείωση. 
Υποστηρίζει πως όλοι οι κλάδοι της γνώσης συνδέονται και έχουν κάτι κοινό 
μεταξύ τους. (Λέφας σελ45 11-12)
Φυσικά μία τέτοια απαίτηση σε γνώσεις απαιτεί παράλληλα μελέτη εκτενούς 
διάρκειας. (Status, Pay, Pleasure p.7)

Όλες οι τέχνες και επιστήμες έχουν μία κοινή σχέση μεταξύ τους, δηλαδή 
συνδέονται. Ο Βιτρούβιος χρησιμοποιεί αυτή την πρόταση σαν επιχείρημα για το 
λόγω που ο αρχιτέκτονας πρέπει να κατέχει μία πληθώρα γνώσεων. Γνώσεις που 
στις μέρες μας θεωρούνται μη απαραίτητες ή σχεδόν μη χρήσιμες για έναν 
αρχιτέκτονα. Ένα χαρακτηριστικό παράδειγμα είναι η αστρονομία. (sollertiae p.5)

Ο Βιτρούβιος τοποθετεί την αρχιτεκτονική στην εγκύκλιο παιδεία. (Status, Pay, 
Pleasure) (Λέφας σελ45 11-12). Η αρχιτεκτονική καταλαμβάνει την υψηλότερη θέση 
στον τομέα της κατασκευής, πράγμα που επιτυγχάνεται από τη νεαρή ενασχόληση και 
μελέτη πάνω στις τέχνες και τα γράμματα.

\subsection{Πρακτική και γνωστική}

\subsection{Από Πλάτωνα σε Βιτρούβιο}

Ο Πλάτωνας και ο Βιτρούβιος έχουν κοινές απόψεις πάνω στο πρακτικό και το
θεωρητικό.

Κατά τη εποχή του Πλάτωνα δεν υπήρχε διαφορά μεταξύ του αρχιτέκτονα και του 
τεχνίτη. Ο αρχιτέκτονας είναι ο πρώτος τεχνίτης αλλά δεν ήταν εξακριβωμένο τι 
το ξεχώριζε. Εξάλλου δεν θεωρούταν από τις καλές τέχνες. (παρ.3-4)

Ο Πλάτωνας υποστηρίζει πως η "πρακτική" (praktike) και η "γνωστική" (gnostike) 
αποτελούν συστατικά της ενότητας της επιστήμης στο σύνολό της. (παρ. 1)

Παραδειγματίζει ένα διακριτικό είδος πρακτικής γνώσης το οποίο είναι επιτακτική
ή εκτελεστική παρά καθαρά καίρια (φιλοσοφική, μαθηματική κλπ.). Έχει να κάνει με
την εντολή παρά με επιστημονικά γεγονότα ή υπολογισμούς. (παρ. 2)

Τόσο και ο Πλάτωνας όσο και ο Βιτρούβιος χαρακτηρίζουν τον αρχιτέκτονα όχι ως 
έναν απλό τεχνίτη αλλά ως ένα ορθά μελετημένο ο οποίος κατέχει το πρακτικό 
κομμάτι του χτισίματος και ταυτόχρονα την γνώση πάνω σε επιστήμες. (παρ. 8)

Για τον Πλάτωνα η πρακτική και η γνωστική είναι πνευματικές γνώσεις που δεν 
απαιτούν χειρονακτική ενασχόληση. Ακριβώς και ο Βιτρούβιος περιγράφει 
σαφέστατα πως η λέξη fabrica είναι επίσης συνδεδεμένη με τη πνευματική 
διαδικασία. Ονομάζει δε τη συγκεκριμένη διαδικασία meditatio. (παρ. 9)

Η τέχνη της αρχιτεκτονικής απαιτεί γνώσεις που αφορούν το "ξέρω πως" και το 
"ξέρω αυτό". (παρ 12)

Κι οι δύο συγκρίνουν την αρχιτεκτονική με τη μουσική. Από τη μία το τελείως 
πρακτικό κομμάτι της μουσικής, δηλαδή η ενασχόληση με τα όργανα και η παραγωγή 
της μουσικής μέσω αυτών και από την άλλη το τελείως θεωρητικό κομμάτι, που 
αφορά τα μαθηματικά και τη μεταφυσική. (παρ. 12)

\subsection{Βιτρούβιος}

Στην αρχή του έργου του, περιγράφει τη διαφορά μεταξύ της πρακτικής μεριάς της
αρχιτεκτονικής και την θεωρητική (fabtica και ratiocinatio αντίστοιχα).

Η γνώση του αρχιτέκτονα πηγάζει τόσο από την μελέτη και θεωρία (ratiocinatio)
αλλά και από την προσωπική ενασχόληση με τα πράγματα δηλαδή την πράξη, τέχνη
(fabrica). Δηλαδή ο αρχιτέκτονας δεν μελετάει απλά αλλά συγχρόνος συνθέτει και
κατασκευάζει. (sollertiae p.4)

Δυστοιχώς η γλώσσα του Βιτρούβιου είναι αρκετά περίπλοκη. Μελετητές και
μεταφραστές αδυνατούν να καταλήξουν σε ένα κοινό συμπέρασμα. *** (παρ. 6)

Η γλώσσα του Βιτρούβιου αφείνει ορισμένα κενά δυσκολεύοντας έτσι την ερμεινία
των προτάσεων του. Διάφοροι μεταφραστές έχουν αποδώσει ορισμένα τμήματα του
έργου με διαφορετικό τρόπο. Έχουν δώσει διαφορετικές ερμεινίες. (sollertioa p.1)

\subsection{Fabrica}

Από τους περισώτερους μεταφραστές υποστηρίζεται πως η λέξη fabrica μεταφράζεται
ως πρακτική, πράκτις. Με την ένοια δηλαδή της επαναλαμβανόμενης άσκησης του
χεριού. Η λέξη έτσι όπως χρησημοποιείται από τον Βιτρούβιο δεν σημαίνει πρακτικό
κτίσμα ή την τέχνη της κατασκευής. (παρ.7)

Κατά τον μεταφραστή Joseph Gwilt η λέξη fabrica σημαίνει συχνή και συνεχή
περισυλλογή (meditatio) των τεχνών σχετικά με τη διαδικασία του χτισήματος. Η
διαδικασία αυτή είανι πνευματική και λετσι η fabrica μέσω της διαδικασίας
meditatio αποδίδει την απιτούμενη επαγγελματική γνώση και εμπειρία. Στην ουσία η
fabrica δεν αφορά αποκλειστικά χειρονακτική γνώση, αν και επιτρέπεται, και
αποκτάται με την άμεση ενασχόληση στο κτίσιμο και ανάλογες τέχνες. (παρ 10)

Η τέχνη κατά τον Βιτρούβιο είναι η εξάσκηση της αρχιτεκτονικής ενατένισης. Το
ρπατκό κομμάτο του "ξέρω πως". ¨εχει να κάνει με την ανάλυση του χώρου,
χωροταξία κλπ. Περιλαμβάνει την όλη επαγκελματική γνώση του αρχιτέκτονα. (παρ
11)

\subsection{ratiocinatio}

Όπως προαναφέρθηκε η γλώσσα του Βιτρούβιου είναι αρκετά περίπλοκή και αποτελεί
κομμάτι σύγχησης μεταξύ μεταφραστών. Η πρόταση:  Ratiocinatio autem est quae res
fabricates sollertiae ac
rationis pro portione demonstrare atque explicare potest, δυσκολεύευι ακόμα τους
μελετητές. (sollertioa p.4)

Πρόθεση του Βιτρούβιου είναι να σχηματίσει μία έννοια, θεωρεία που αοφρά την
αναλογία. Η λε΄ξη κατα αυτή δεν παρατειρέιται στο κείμενο του. Η λέξη
ratiocinatio έχει ως βάση τη λέξη ratio δηλαδή λόγος. Ο λόγος στα ελληνικά έχει
μια πληθόρα εννοιών. Η μαθηματική του έννοια είναι η σχέση μεταξύ δυο αριθμών. Η
λέξ λόγος είχε να κάνει με την τη σχέση μεταξύ δύο αριθμών (couple of two
numbers) πριν πάρει την σημερινή έννοια με βάση τον Ευκλείδη, δηλαδή το πιλίκο
δύο αριθμών. (sollertiae p.5)

Η λέξη ratio πηγάζει από την ελληνική λέξη λόγος. Με την ένοια λόγος μπορούμε
να αναφερθούμε στη θεωρεία, επεξήγηση, περιγραφή, λόγος με τη μαθηματική
έννοια.. κλπ. Ο Αριστοτέλης αναφερόταν στη λέξη λόγος ως τον λογικό ορισμό
εννοιών, ακριβώς όπως και οι διάδοχοι χρησιμοποιούν τον όρο. Ομολιως και οι
λέξεις ratio, rationale και ratiocinatio. (παρ. 14)

Η θεωρεία δηλαδή έχει να κάνει με τη θεωρεία της αναλογίας (ratio). 

Ο βιτρούβιος χρησημοποιεί σαν παράδειγμα τη σχέση ενός μουσικού μέ έναν ιατρό.
Σημειώνει ως κοινό στοιχείο μεταξύ των δύο αυτών επαγγελμα΄τον τον ρυμό. Από τη
μία πλευρά την κίνηση του ποδιού ενός μουσικού ωστε να μετράει σωστά τις νότες
και από την άλλη την παλμό του ασθενή που μετράει ένας ιατρός. Ο Βιτρούβιος
κάνει αναφορά στον Αριστόξενος ο Ταραντίνος Ο οποίος έδωσε ιδηέτερη σημασία στη
θεωρία της μουσικής αναλογίας, κάτι το ποίο δεν ήταν εγκεκριμένο από την
αρχαιότητα ** Σχέση μεταξύ της ακουστικής αρμονίας και της χωρικής αναλογίας. 

Η διαφορά του ύψους των ήχων βρσίκεται στην αναλογία των νούμεεων και τον
ταχητήτων. Κατά τον Αριστόξενο η πρώταση αυτή είναι λάθος. Υποστηρίζει πως οι
μουσικοί ήχοι έχουν να κάνου αποκλειστικά με την αίσθηση της ακοής και όχι με
την αναλογία. Ο ίδιος δεν ήταν μαθηματικός, έδινε σημασία όμως στη αίθσηση. 

Όπως φαίνεται για τη λέξη ratiocinatio υπάρχει ακόμα σύγχηση. Το τμήμα του
κειμένου του Βιτρούβιου που περιγράφει τον ορισμό δεν είναι εύκολο να
μεταφρστεί. Όλοι οι μεταφραστές παρουσιάζουν διαφορετικές εκδοχές. Σκατά

Ratiocinatio autem est, quae res fabricatas sollertiae ac rationis proportione
demonstrare atque explicare potest

Φαέινεται να υπάρχει σύγχηση με την λέξη proportione ή pro portione. Ο λέφας
τη λέει μία λέξη. 
Σύμφωνα με την ικανότητα και τη συλλογιστική του αρχιτέκτονα.


  
  Ο {\color{red}\textbf{βιτρούβιος}} στο \emph{κείμενό} του \textit{αναφέρει} τις 6 αρχές της αρχιτεκτονικής. Τάξη, Διάθεση, ευρυθμία, συμετρία, κοσμιότητα και οικονομια. \cite{scranton_vitruvius_1974, vitruvius-lefas}
  
\begin{enumerate}[noitemsep] %αλλίως itemize
  \item ένα
  \item Δύο
  
  \begin{itemize}
    \item ένα.1
    \item ένα.2
  \end{itemize}
  
  \item Τρία
  
    \begin{description}
      \item[όρος 1] μπλαμπλαμπλά ηςθεφη.
      \item[όρος 2] ηςθεφη.
    \end{description}
  
  \item Τέσσερα
\end{enumerate}

Διαβάζοντας το εκίεμνο του παρατηστουμε οτι αυτές οι αρχές τις αρχιτεκτονικής 
αναφέρονται και αποτελούν το αισθητικό κομάτι ενός κτηρίου, τις ποιότητες δηάδή 
ενός κτηρίου. Το προιόν του αρχιτέκτονα. Ο Βιτρούβιο όμως έχει κάτι άλλο στο 
νου του. ναι μεν λέει πως οι αρχές αφορούν το αισθητικό κομμάτι, δηλαδή το έργο 
τέχνης καθέ αυτού, αλλά αναφέρει πως αφορούν και τηη ίδια την τέχνη του 
αρχιτέκτονα, δηλα΄δή τι κάνιε ο ίδιος. Η πρακτική του ενέργεια. Κατά μία έννοια 
τις χωρίζει σε τάξη, διάθεση, και οικονομία και η δεύτερη κατηγορία είναι η 
ευριθμία, η συμετρία και η κοσμιότητα. Η πρώτη έχει να κάνει με την 
διαδικάσίατου αρχιτέκτονα. Η δεύτερη έχει να κάνει με τις ποιότητες. 
(Vitruvious arts of architecture p.3 και 5)
  
  Όπως έχει αναφερθεί η γλώσσα του Βιτρούβιου είναι κάπως περίπλοκή. ο ίδιος 
  αναφέρει στο τέτλος του πρώτου κεφαλάιου του πρώτου βιβλίου, οτι δεν είναι 
  φιλόσοφος ούτε είναι ειδικός στη γραμματιιή αλλά ε΄νας αρχιτέκτονας, ειδικός 
  στον τομέα του. Πάνω κάτω υπονοεί πως τα δικά του λε΄γην έχουν μία δοση 
  προτοτυπίας. Όπως είπαμε η γλώσσα που χρησιμοποιεί δεν επιτρέπει την σίγουρη 
  και ορθή μετάφραση. και οι διαφορετικές μεταφράσεις δεν βοηθάνε επίσης. 
  Αφείνεται έτσι χώρος για περετέρο ερμεινεία των λε΄γην του. (Vitruvious arts 
  of architecture p.3)
  
  Για παράδειγμα η λέξη διάθεσης (arragement), μία απότ ις αρχές της αρχιτεκτονικής που αναφέρει ο ΒΙτρούβιος, έχει διαφορετικές σημασίες στα αγγλικά. Η πράξη του να τοποθετούμε πράγματα στο χώρο ή η τοποθέτηση των πραγμάτων μεταξύ τουυς σε ένα σύνολο. (Vitruvious arts of architecture p.3)  
  
  \subsection{Τάξης(Ordinatio)}
  
Μία ελαφριά σύγχυση μεταξύ των μεταφραστών. Δεν φαίνεται να έχουν διαφορετικές απόψεις, αλλά δεν μπορούν να καταλήξουν στο συμπέρασμα εάν η ταξης και η συμετρία συνδέονται ή εάν υπάρχει λόγος όπου θεματολογία του Βιτρούβιου είναι έτσι (πρώτα η τάξης και μετά η γεωμετρία. Ενώ η γεωμετρία είνια στην ουσιία το θεμέλιο της σκέψης του).*** \cite[σ.~187]{vitruvius-lefas}

Στην περιγραφή της τάξης, ο Βιτρούβιος κάνει χρήση δυο λέξεων. Commoditas και modica. Και οι δύο λέξειείναι παράγωγες από την ελλνική μέτρο

Ορισμός από το βιβλίο. 

Comparatio. Η σημασία της λέξης αυτής είνια διπλή στη περιγραφή της τάξης. Αρχικά σημαίναι σύνθεση, κατασκευή, τοποθέτηση πραγμάτων μαζί. Παράλληλα σημαίνει σύγκριση, αντίθεση μεταξύ πραγμάτων. Για τον Βιτρούβιο η λε΄ξη αυτή έχει ένα στοιχειο της σύγκρισης αλλά και τη σύνθεση ιεραρχίας.

Ιεραρχία. Κατασκευή όπου εμπεριέχει το στοιχείο της σύγκρισης. Ιεράρχιση των μεγεθών στο σύνολο.

Η λέξη αυτή ακολουθείται και περιγράφεται με άλλες διάφορες. Είναι έυκολο να υποθέσουμε πως η τάξης βρίσκεται στην υπηρεσία της γεωμετρίας, όχι με τον κυριολετκικό όρο, δηλαδή του ευκλείδη. Ο λέφας φαίνεται να διαφωνεί σε α΄τυο. Υποστηρίζει πως η παρουσία της λέξης γεωμτρία στον ορισμό του Βιτρούβιου για την τάξη έχει την κυριολεκτική σημασία της έννοιας. Η γεωμετρία κατά τον Ευκλείδη λέει πως δύο πράγματα είναι συμετρικά εάν έχουν κοινό μέτρο από ένα σημείο.*** Ο λέφας έπειτα δίνει σαν παράδειγμα την ιεραρχεία της αρχαίς Ρώμης. Με την έννοια οτι ο Καίσαρας συγκριτικά με έναν κατότερο είναι μη συμετρικός,  δηλαδή δεν υπάρχει μία υπερύψοση του πρώτου σε σχέση με το δεύτερο. Ίσως αυτό προσπαθεί να πει ο Βιτρούβιος. Σε ένα κτήριο για πραδάδειγμα, όπως συνεχίζει ο Λέφας, ένα ένα τμήμα του υπερισχύει σε σχέση με ένα άλλο τότε το καθιστά μη απαραίτητο, αδιάφορο. Οπότε η τάξη δεν είναι αντικείμετνο της γεωμερείας αλλά μέσω της τάξης διατηρέιται η γεωμτρεία*** 

Ρητορική. Ο λέφας αναφέρει πως Η τάξη στη ρητορική έχει να κάνει με τη σωστή τοποθέτηση αλλά κια το σωστό μέγεθος των μερών του λόγου. Υπάρχουν όπως φαίνεται πολοί που έχουν συγκρίνει τη ρητορική με την αρχιτεκτονική. Οπότε θα μπορούσε να είχαν σχηματιστεί δύο διαφορετικές έννια και για αυτή. Η τάξη και η διάθεση. Η πρώτη έχει να κάνει με την την ποσοτητα και η διάθεση με την χωρική πτυχή.

Η ποσότητα έχει να κάνει με τον ορισμό ένος κοινού μέτρου και αυτό το κοινό μέτρο έχει να κάνει με τη σειρά του με τα μέλη του συνόλου. Ένα ορθό κατασκευασνένο σύνολο.

Η αρμονία έχει να κάνει με την επιλογή ενός μέτρου από τα επιμέρους σκέλη και μέρη του συνόλου έχοντας ως βάση την ίδια τη σχέση τον μεγεθών, φτιάχνοντας έτσι ένα ορθό, συμπαγές σύνολο. Η συμετρία έχει να κάνει με την συμαβτότητα των μέρων που φτιάχνουν ένα σύνολο βασισμένη σε ένα κοινό μέτρο. Διαφορά μεταξύ ποσότητας και συμετρίας. και τα δύ καταλήγουν στην αρμονία μέσω διαφορετικών δρόμων.

Η τάξη σχηματίζεται από την ποσότητα. όπου επιλέγονται τα μέρη του συνόλου καταλλ΄γοντας έτσι σε ένα αρμονικό σύνολο***
  
\subsection{Διάθεσης (dispositio)}
  
  Ο Βιτρούβιος χρησημοποιεί ωε συνώνυμα τις λέξεις conlocatio και effectus. 
  Τόσο η λέξη dispositio και η conlocatio αλλά και ολα τα ουσιαστικά με 
  κατάληξη -tio, -tionis, έχουν ως κύρια ερμηνεία την έννοια της εκτέλεσης μίας 
  διαδικασίας και σαν δεύτερη ερμηνεία την έννοια του αποτελέσματος μίας 
  διαδικασίας. Η έννοια της διάθεσης είναι "τοποθέτηση μαζί" και το 
  "πραγματοποιείται" ή το "λειτουργεί", όχι όμως το "αποτέλεσμα", "η 
  προκύπρουσα σύνθεση μορφή". (Vitruvious arts of architecture p.3)
  
  Τόσο ο Granger όσο και ο Morgan μεταφράζουν τη λέξη dispositio ως Arragement, δηλαδή διευθέτηση. Στην ουσία η έννοια της λέξης δεν είναι η τάξη, σειρά δηλαδή η ποιότητα ενός έργου αλλά η διαδικασία της ορθής τοποθέτησης των πργμάτων. Ο υπολογισμός των μέτρων του ίδιου του κρίσματος αλλά και των επιμέων στοιχείων του.
  
\subsection{Ιχνογραφέια, Ορθογραφεία, και τοπογραφέια}
  
  Ο Βιτρούβιος κάνει χρηση των λέξεων αυτών. Ελληνικές λέξεις για τις οποίες ο ίδιος δε χρησιμοποιεί κάποια λατινική. Οι λέξεις αυτές ερμηνεύονται ώς κάτοψη, όψη και ισως προπτική αντίστοιχα. Ο βιτρούβιος βέβαια τις χρησημοποιεί με τον δικό του τρόπο. Η ιχνογραφεία και η τοπογραφεία ορίζονται σαν διαδικασίες, Κάνουν χρήση πιξήδας, διαβήτη και χάρακα. Η ορθογραφεία από την άλλη χαρακτηρίζεται. Λειτουργούν ως οι πρώτες πράξεις που κάνει ένας αρχιτέκτονας. (Vitruvious arts of architecture p.3)
  
  Ο βιτρούβιος έχει στο νου του τη διαδικασία του σχεδιαμού και σύνθεσης, όπως ξεχωρίζει από το γενικό του σύνολο (ολοκληρωμένο έργο). Είναι η διαδικασία του σχηματισμού της κάτοψης και των σχεδίων, όχι τα ίδια τα σχέδια.
  
\subsection{Οικονομία (Distributio)}

  Η λέξη αυτή όπως χρησιμοποιείται από τον Βιτρούβιο είναι λέξη κατάληξης -tio -tionis, δηλαδή δείχνουν μία έννοια εκτέλεσης μίας διαδικασίας.
  
  Ο όρος έχει να κάνει με την διαδικασία του αρχιτέκτονα στο να σχεδιάσει, κατασκευάσει έναν λογικό προϊπολογισμό κόστους, έχοντας στο νου του την οικονομική άνεση και το κύρος του πελάτη. Ίσως να υπάρχει μία ανακρίβεια στη σκέψη του καθώς αυτός ο ορισμός τίνει να κατεγθείναιται προς την κοσμιότητα. Λογικά ο Βιτρούβιος δείχνει περισσότερη προσοχή στον πελάτή καθώς περιγράφει ορισμένες περιπτώσεις παραδείγματα και ταυτόχρονα τονίζει το οικονομικό του επίπεδο. Στην κοσμιότητας δείνει έμφαση στη μορφή του κτίσματος. (Vitruvious, arts of architecture p.4 497)
  
\subsection{Συμετρία (Symetry)}

  Η συμετρία για τον Βιτρούβιοι είναι η θεμελιώδης αρχή στην ολόκληρη έννοια του σχεδιαμόυ του. Για τον ίδιο η λέξη συμετρία δεν έχει τον ορισμό κατά τον Ευκλείδη. [Παραπομή] Αντί αυτού, στο πρωτο σκέλος της ερμινείας του, ο βιτρούβιος παραπέμπει τον ορισμός της συμμετρίας στην πλατωνική αντίληψη της, η οποία έχει ως βάση το "μέτρο". Στο δεύτερο σκέλος της παραπέμπει στην αντίληψη των Πυθαγορείων για τον κόμσο αλλά και τον Πλάτονα, ορίζοντας τη συμετρία ως αποτέλεσμα αναλογικών και αρισθητικών σχέσεων. ΄Όπως ακριβώς και στις έννοιες της τάξης και ευρυθμίας. \Cite[σ.~51,96]{vitruvius-lefas}

  \begin{description}
    \item[Απόσπασμα] "Η συμετρία είναι η συμφωνία που προκύπτει από την εναρμόνιση του κάθε μέλου του έργου με τα άλλα, Βάση ένος επιλεγμένου μεταξύ τους μέτρο."
  \end{description}

Η συμετρία η συμφωνία που προκύπτει από τη σχέση των μελώ του έργου. Έχει τις βάσεις της στις αριθμιτικές αναλογίες του ανθρώπινου σώματος, του οποίου Τα μέλη έχουν εκ φύσεως συγκεκριμένες αναλογίες μεταξύ τους. Ο πήχης, το πόδι, η παλάμη, το δάκτυλο και άλλα μέρη του σώματος καθιστούν εύρυθμο το ανθρώπινο σώμα. \cite[σ.~51]{vitruvius-lefas}.

Ακριβώς το ίδιο συμβαίνει και στα κτήρια, ή μάλλον, πρέπει να συμαβάινει κατα την διαδικασία του σχεδιαμσού και του χτισίματος. Αμέσως μετά ο Βιτρούβιος στον ορισμό της συμετρίας παρουσιάζει ορισμένα παραδείγματα για να εξηγείσει την έννοια της συμετρίας. Οι ναοί κατέχουν τη συμετρία που περιγράφει, καθώς τα μέρη τους οπως είναι ο κίονας και η τρίγλυφος, είναι εναρμονισμένα μεταξύ τους cite[σ.~51]{vitruvius-lefas}. Στη συνέχει του συγκράματος του αναφέρει και ορισμένα άλλα παρδείγματα όπως για στο βιβλιο IV, 6,1 περί της συμετρίας της Δωρικής θύρας. Η συμετρία έχει να κάνει με τις αναλογίες και τη σχέση των μεγεθών ενός κτηρίου ακριβώς όπως παρατηρείται και στις αναλογίες του σώματος του ανθρώπου. Έτσι επιτυγχάνεται στην ουσία η συμετρία, δηλαδή έτσι μπορεόυμε να επιτύχουμε την συμφωνία από την εναρμόνιση των μελών ενός συνόλου \cite[σ.~187]{lefas-fundamental}.

Όλες οι τεχνικές κατασκυές έχουν ένα τέτοιο σύστημα συμετρίας. Όπως είπαμε η συμμετρία έχει να κάνει με μία κοινή κλίμακα μέτρου, δηλαδή μία κοινή μονάδα μέτρησης. Η οποία με τη συνέχει προέρχεται από τη σχέση τους μεταξύ τους δηλαδή το σύνολο τους, το κτίσμα ολόκληρο.

Η συμμετρία έχει κάπως έναν πιο τεχνητό χαρακτήρα. Συνδέεται άμεσα με την τάξη και την Ερυθμία σε επίπεδο όπυ είναι σχεδόν απαράιτητη η πρϋπόθεση της αλλά ταυτόχρονα παραμένει ξεχωριστή. ΌΠως είπαμε η τάξη μεταφέρει μία ιεραρχική έννοια. ***

Proportio. Στα Ελληνικά αναλογία. Κατά τον Βιτρούβιο η αναλογία παραπέμπει στην έννοια της γεωμετρίας κατά τον Ευκλείδη. Η αναλογία είναι οι αριθμιτικές σχέσεις μεγεθών σε ένα έργο. Η αναλογία είναι στην ουσία η ισομετρία, ενώ η γεωμετία εμπεριέχει μία ποιτική πτυχή.

Με το να ορίζει τις αναλογίες των μελών ενός συνόλου, στην ουσία ο Βιτρούβιος ορίζει τη σχέση των μεγεθών, άρα και την ιεραρχία τους.
  
  \subsection{Ευρυθμία (Eurytmia)}
  
  Μάλλον αυτό που έχει σοτυ νου του είναι η ποίτητα του ρυθμού πιο συγκερκιμένα. Ο δυναμικός παράγωντας στη παρουσίαση του έγου, ήδη κινήσεων ενσωματομένα στις γραμμές. Η αρχιτεκτονική πρέπει να έχει καλό ρυσμό. Ευχάριστο στις αισθήσεις. Παράλληλα έινια συνδεδεμένο με την  συμμετρία.
  
  \subsection{Κοσμιότητα (Decor)}
  
  Μάλον έχει να κάνει με τη  αισθητική. Δεν φαίνεται να έχει ιδιαίτερη σχέση με τις δικές μας αισθητικές αρετές.
  Δίνει παραδείγματα τα οποία φρορούν τους γνωστούς ρυθμούς (Δορικός, Ιωνικός, Κορινθιακός) όπουυ ο κάθε ένας προορίζεται και ταιριάζει καλύτερα με συγκεκτιμένες θεότητες.
  ¨οπως φαίνεται Ο Βιτρούβιος προσπαθεί να προσδώσει μια έννοια "καταλληλότητας" στα κτήρια. Φακξιοναλισμός, μορφή ακολουθεί τη χρήση.
  
  Το ρπόβλημα που αντιμετωπίζουμε είναι σημασιολογικό. Αυτό λόγω της γλώσσας του Βιτορύβιου και των μετα φρστών.
