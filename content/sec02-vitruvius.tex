% !TEX root = ../main.tex
% !TEX encoding = UTF-8 Unicode

\section{Ο Βιτρούβιος}

Δύσκολα μπορεί κανείς να περιγράψει αναλυτικά τον Βιτρούβιο\sidenote{%
    \href{https://en.wikipedia.org/wiki/Vitruvius}{Marcus Vitruvius Pollio}}
σαν προσωπικότητα, χωρίς κίνδυνο σφαλμάτων ή ελλειπούς περιγραφής, καθώς, παρά τις επίμονες προσπάθειες ερευνητών και μελετητών, λίγα είναι γνωστά για τον ίδιο. Ότι γνωρίζουμε για την προσωπικότητα, η ζωή και την καριέρα του, προέρχεται από ελάχιστες, διάσπαρτες πηγές και αποτελούν ακόμα αντικείμενο μελέτης. Πιο συγκεκριμένα τα περισσότερα στοιχεία για το Βιτρούβιο προέρχονται από το έργο του, \emph{"Περί Αρχιτεκτονικής"} \cite{vitruvius-lefas,baldwin-1990,masterson_status_2004}.

Αυτά που γνωρίζουμε ως προς τη κοινωνική του κατάσταση είναι πως ήταν στρατιωτικός μηχανικός και αρχιτέκτονας, έζησε κατά τη Ρωμαϊκή περίοδο του πρώτου αιώνα π.Χ. Υπηρέτησε στο ρωμαϊκό  στρατό, αρχικά επί \emph{Ιούλιου} \sidenote%
    {\href{https://en.wikipedia.org/wiki/Julius_Caesar}{Gaius
    Julius Caesar} (12 Ιούλ. 100 π.Χ. – 15 Μάρτ 44 π.Χ.) ρωμαίος
    αυτοκράτορας που έπαιξε σημαντικό ρόλο στη άνοδο της ρωμαϊκής
    αυτοκρατορίας}
και στη συνέχεια επί \emph{Αύγουστου} Καίσαρα \sidenote%
    {\href{https://en.wikipedia.org/wiki/Augustus}{Augustus Caesar}
    (23 Σεπτ. 63 π.Χ. – 19 Αύγου. μ.Χ. 14)} 
με την ιδιότητα του τεχνικού των στρατιωτικών μηχανών \cite{vitruvius-lefas,enwiki:vitruvius}. Την ύστερη περίοδο της ρωμαϊκής αυτοκρατορίες στην οποία έζησε ο Βιτρούβιος ήταν σύνηθες οι αρχιτέκτονες να υπηρετούν το \emph{Λαϊκό} κόμμα του αυτοκράτορα, το οποίο ήταν υπέθυνο για τις οικοδομικές κατασκευές της αυτοκρατορίας \cite[σ. 13]{vitruvius-lefas}.

Η εποχή που έζησε ο Βιτρούβιος χαρακτηρίζεται από μία έντονη τάση για αλλαγή στις τυποποιημένες μεθόδους απόκτησης γνώσεων και εφαρμογής στις τέχνες. Ενώ μέχρι τότε, το ποιός θα γινόταν τι είδους τεχνίτης και ποιές γνώσεις έπρεπε να κατέχει, είτε μεταβιβαζόταν από πατέρα σε γιο είτε ελεγχόταν παραδοσιακά από κάστες επαγγελματιών \sidenote{%
    Νομομαθείς, γεωγράφους, δασκάλους της γλώσσας και αρχιτέκτονες}, η νέα τάση απαιτούσε την ύπαρξη εξειδικευμένων τεχνιτών από κάθε γνωσιακό κλάδο. Την τάση αυτή εκμεταλεύθηκε ο Αύγουστος Καίσαρας με το να προσλλάβει νέα άτομα, ώστε να δυναμώσει την πολιτική και κοινωνική του εξουσία. \cite{masterson_status_2004,brown-vitruvius}. Ο Βιτρούβιος ήταν ένας από αυτούς του εξειδικευμένουν τεχνικούς που ακολούθησαν τον Αύγουστο στις διάφορες μετακινήσεις του στρατού του εντός της αυτοκρατορίας, από τη Γαλλία μέχρι και την ανατολή και την Αφρική. Αυτό φαίνεται να έχει επιρεάσει το \emph{Περί Αρχιτεκτονικής} στις αναφορές που κάνει σε μνημία και περοχές που επισκέφτηκε \cite[σ. 13]{vitruvius-lefas}. Στην κατεύθυνση αυτή της αλλαγής από την κλασική ρωμαϊκή κουλτούρα σε μία νέα, βοήθησε η σύνθεση ελληνιστικών, πολιτιστικών σοιχείων στη ρωμαϊκή κουλτούρα. Όπως θα δούμε και στη συνέχεια η σύνθεση αυτή βρήκε τη καλύτερη μορφή της μέσα από την \emph{εγκύκλιος παιδεία}.


%τις ελευθέριες τέχνες της ελληνιστικής περιόδου. Άτομα τα οποία το είπαν αυτή ήταν ο Κικέρον, ο Βάρρος και ο Βιτρούβιος. Αυτό που είχαν στο νου τους ήταν μία γενική απόκτηση γνώσης για έναν επιστήμονα δεν είχε να κάνει (απαραίτητα) με την επαγγελματική του κατάρτιση πάνω στη λογική, φυσική και ηθική, δηλαδή να γίνει φιλόσοφος. Αντί αυτού στο νου τους είχαν μια γενική απόκτηση γνώσεων και πειθαρχίας. (Vitruvius and the liberal arts of arcitecture)





% Οι πληροφορίες που υπάρχουν πηγάζουν από το ίδιο του το έργο \cite[σ.~390]{masterson_status_2004}.

% Ο Βιτρούβιος είναι γνωστός για το έργο του "Περί αρχιτεκτονικής" (De architectura), ένα σύνολο συγγραμμάτων, βιβλίων τα οποία περιγράφουν τη δομή της αρχιτεκτονικής.**Τα επαγγέλματα των μηχανικών κανονικά μεταφέρονταν από γονιό σε παιδί.  *** 

% Ο Αύγουστος χρησιμοποίησε σαν ευκαιρία την παρουσία των νέων ατόμων, ειδικών  για να αυξήσει την ήδη υψηλή πολιτική και στρατιωτική του δύναμη. Ο Βιτρούβιος ήταν ένα από αυτά τα άτομα. (Status, Pay, Pleasure p.1) *** και στη περίπτωση του Βιτρούβιου οι γονείς φρόντισαν για την εκπαίδευση του σαν  αρχιτέκτονας. Ο Βιτρούβιος μάλιστ0α κάνει αναφορά στο σύγγραμμά του στους δασκάλους του που τον εκπαίδευσαν. (Λέφας σελ 13).

% Στη περίοδο που έζησε, της ύστερης Ρωμαϊκής δημοκρατίας, οι αρχιτέκτονες όφειλαν να υποστηρίζουν το λαϊκό κόμμα του Καίσαρα και του κλώδιου (Λέφας σελ 13). Συμπεραίνεται πως ο Βιτρούβιος άσκησε το επάγγελμα του αρχιτέκτονα στο στρατό του αυτοκράτορα Καίσαρα με ιδιότητα ως τεχνικός στρατιωτικών μηχανών όπου συμπεραίνουμε ότι ο ίδιος είχε ακολουθήσει τον στρατό του Καίσαρα σε διάφορες μετακινήσεις από τμήματα της Γαλλίας μέχρι την Αφρική. Οι αναμνήσεις και η εμπειρίας από τα ταξίδια του μεταφέρονται και στο έργο του. φαίνεται πως ο ίδος ήαν ένας "aparitor", δηλαδή ένας δημόσιος υπάλληλος του οποίου ο μισθός προερχόταν από το δημόσιο ταμέιο.

% Η εποχή που έζησε ο Βιτρούβιος είχε να κάνει με την αλλαγή. Μία εποχή όπου οι τέχνες και τα επαγγέλματα είχαν πια αναβαθμιστεί, κερδίζοντας νέα σημασία σε σχέση με την παράδοση. Η κλασική ανατροφή, όπου είχε θρέψει την κλασική παλιά ρωμαϊκή κουλτούρα δεν ήταν πια σχετική. Η επαγγελματική εκπαίδευση είχε πια ξεπεράσει την κλασική. Βρήκαν πάτημα στην ελληνιστική παράδωση. έτσι και η αρχιτεκτονική. (Vitruvius and the liberal arts of architecture)