% !TEX root = ../main.tex
% !TEX encoding = UTF-8 Unicode

\section{Ο Βιτρούβιος}

Δύσκολα μπορεί κανείς να περιγράψει αναλυτικά τον Βιτρούβιο\sidenote{%
    \href{https://en.wikipedia.org/wiki/Vitruvius}{Marcus Vitruvius Pollio}}
σαν προσωπικότητα, χωρίς κίνδυνο σφαλμάτων ή ελλιπούς περιγραφής, καθώς, παρά 
τις επίμονες προσπάθειες ερευνητών και μελετητών, λίγα είναι γνωστά για τον 
ίδιο. Ότι γνωρίζουμε για την προσωπικότητα, η ζωή και την καριέρα του, 
προέρχεται από ελάχιστες, διάσπαρτες πηγές και αποτελούν ακόμα αντικείμενο 
μελέτης. Πιο συγκεκριμένα τα περισσότερα στοιχεία για το Βιτρούβιο προέρχονται 
από το έργο του, \emph{"Περί Αρχιτεκτονικής"} 
\cite{vitruvius-lefas,baldwin-1990,masterson_status_2004}.

Αυτά που γνωρίζουμε ως προς τη κοινωνική του κατάσταση είναι πως ήταν στρατιωτικός μηχανικός και αρχιτέκτονας, έζησε κατά τη Ρωμαϊκή περίοδο του πρώτου αιώνα π.Χ. Υπηρέτησε στο ρωμαϊκό  στρατό, αρχικά επί \emph{Ιούλιου} \sidenote%
    {\href{https://en.wikipedia.org/wiki/Julius_Caesar}{Gaius
    Julius Caesar} (12 Ιούλ. 100 π.Χ. – 15 Μάρτ 44 π.Χ.) Ρωμαίος
    αυτοκράτορας που έπαιξε σημαντικό ρόλο στη άνοδο της ρωμαϊκής
    αυτοκρατορίας}
και στη συνέχεια επί \emph{Αύγουστου} Καίσαρα \sidenote%
    {\href{https://en.wikipedia.org/wiki/Augustus}{Augustus Caesar}
    (23 Σεπτ. 63 π.Χ. – 19 Αύγου. μ.Χ. 14)} 
με την ιδιότητα του τεχνικού των στρατιωτικών μηχανών 
\cite{vitruvius-lefas,enwiki:vitruvius}. Την ύστερη περίοδο της ρωμαϊκής 
αυτοκρατορίες στην οποία έζησε ο Βιτρούβιος ήταν σύνηθες οι αρχιτέκτονες να 
υπηρετούν το \emph{Λαϊκό} κόμμα του αυτοκράτορα, το οποίο ήταν υπεύθυνο για τις 
οικοδομικές κατασκευές της αυτοκρατορίας \cite[σ. 13]{vitruvius-lefas}.

Η εποχή που έζησε ο Βιτρούβιος χαρακτηρίζεται από μία έντονη τάση για αλλαγή 
στις τυποποιημένες μεθόδους απόκτησης γνώσεων και εφαρμογής στις τέχνες. Ενώ 
μέχρι τότε, το ποιος θα γινόταν τι είδους τεχνίτης και ποιες γνώσεις έπρεπε να 
κατέχει, είτε μεταβιβαζόταν από πατέρα σε γιο είτε ελεγχόταν παραδοσιακά από 
κάστες επαγγελματιών \sidenote{%
    Νομομαθείς, γεωγράφους, δασκάλους της γλώσσας και αρχιτέκτονες}, η νέα τάση 
    απαιτούσε την ύπαρξη εξειδικευμένων τεχνιτών από κάθε γνωσιακό κλάδο. Την 
    τάση αυτή εκμεταλλεύθηκε ο Αύγουστος Καίσαρας με το να προσλάβει νέα άτομα, 
    ώστε να δυναμώσει την πολιτική και κοινωνική του εξουσία. 
    \cite{masterson_status_2004,brown-vitruvius}. Ο Βιτρούβιος ήταν ένας από 
    αυτούς του εξειδικευμένου τεχνικούς που ακολούθησαν τον Αύγουστο στις 
    διάφορες μετακινήσεις του στρατού του εντός της αυτοκρατορίας, από τη 
    Γαλλία μέχρι και την ανατολή και την Αφρική. Αυτό φαίνεται να έχει 
    επηρεάσει το \emph{Περί Αρχιτεκτονικής} στις αναφορές που κάνει σε μνημεία 
    και περιοχές που επισκέφτηκε \cite[σ. 13]{vitruvius-lefas}. Στην κατεύθυνση 
    αυτή της αλλαγής από την κλασική ρωμαϊκή κουλτούρα σε μία νέα, βοήθησε η 
    σύνθεση ελληνιστικών, πολιτιστικών στοιχείων στη ρωμαϊκή κουλτούρα. Όπως θα 
    δούμε και στη συνέχεια η σύνθεση αυτή βρήκε τη καλύτερη μορφή της μέσα από 
    την \emph{εγκύκλιος παιδεία}.