% !TEX root = ../main.tex
% !TEX encoding = UTF-8 Unicode

\section{Vitruvius}

\cite{elwiki:vitruvius,enwiki:vitruvius,vitruvius-lefas,masterson_status_2004,baldwin-1990}

Είναι δύσκλολο να περιγράψει κανείς την προωπικότητα του Βιτορύβιου καθώς ο ίδιος αποτελέι μία αινιγματική φυσιογνομία στην ιστορία των αρχαίων πολιτισμών. Παρά τις επίμονες προσπάθειες ερευνητών και μελετητών, λίγα είναι γνωστά για τον ίδιο, ενώ η προσωπικότητά του, η ζωή του και η καριέρα του, αποτελούν ακόμα αντικείμενο μελέτης για τους ειδικούς. Το έργο του, \emph{De Architectura} ή αλλίως \emph{Περί Αρχιτεκτονικής}, αποτελεί τη κύρια πηγή πληροφοριών για αυτό το σημαντικό πρόσωπο της ιστορίας. \cite{elwiki:vitruvius,enwiki:vitruvius,vitruvius-lefas,baldwin-1990,masterson_status_2004}

Γνωρίζουμε με σιγουριά πως ο Βιτρούβιος ήταν στρατιωτικός μηχανικός και αρχιτέκτονας ο οποίος έζησε κατά τη Ρωμαϊκή περίοδο του πρώτου αιώνα π.Χ. (ημερομηνιες) και υπηρέτησε στον στρατό του Αύγουστου Καίσαρα και πιθανολογέιται, ύστερα από αναφορές, πως ήταν ακόλουθος και του αυτόκράτορα Ιούλιου Καίσαρα (Θετός πατέρας του Αύγουστου). \cite{enwiki:vitruvius} Την ύστερη περίοδο της ρωμαϊκής αυτοκρατορίες στην οποία έζησε ο Βιτορύβιος, οι υπήκοοι του αυτοκράτορα προσέφεραν τις υπηρεσίες στους στο Λαϊκό κόμμα του αυτοκράτορα και του Κλώδιου, το οποίο ήταν υπέθυνο για τις οικοδομικές κατασκευές της αυτοκρατορίας. \cite[σ. 13]{vitruvius-lefas}

Η εποχή αυτή χαρακτηρίζεται από την έντονη ανάγκη του πολιτισμού και των ανθρώπων για την αλαγή. Μία εποχή όπου οι τέχνες και τα επαγκέλματα είχαν πια αναβαθμιστεί, αποκτόντας έτσι νέα σημασία για του ανθρώπους και για την ίδια την παράδοση, καθώς η παλία Ρωμαϊκή κουλτούρα δεν ήταν πια σχετική.(Vitruvius and the liberal arts of architecture) Παράλληλα κατά την Ρωμαϊκη περίοδο γενικά, ήταν σύνηθες φαινόμενο η μεταβίβαση του επαγκέλματος του πατέρα προς τον γιο. Ο αυτοκράτορας Αύγουστος, αναμφισβήτητα ένας από τους σπουδαιότερους ηγέτες στην ιστορία, εκμεταλλεύτηκε την παρούσα κατάσταση της αυτοκρατορίας για την ανάγκη για αλλαγή με το να προσλλάβει νέα άτομα, ώστε να δυναμώσει την πολιτική και κοινωνική του εξουσία. \cite[σ. 387]{asterson_status_2004} Ο Βιτρούβιος αποτέλεσε και αύτος ένα από τα άτομα, νεοσύλεκτους όπου ακολούθησαν τον Αύγουστο στις διάφορες μετακινήσεις του στρατού του εντός της αυτοκρατορίας, από τη Γαλλία μέχρι και την ανατολή και την Αφρική. Πράγμα που φαίνεται να έχει επιρροή στη συγγραφή και στις αναφορές που κάνει ο ίδιος στο \emph{Περί Αρχιτεκτονικής} πάνω σε μνημία και περοχές από διάφορες περοχές που επισκέφτηκε. \cite[σ. 13]{vitruvius-lefas}

Με σκοπό να βρεθεί μία λύση στην αναγκαία αυτή αλλαγή από την κλασική ρωμαϊκή κουλτούρα, σε μία νέα, σκεπτικιστές υποστήριξαν πως μία σύνθεση μεταξύ ελληνιστικού και ρωμαϊκού ήταν αυτό που χρειάζονταν. Προέβλεψαν πως μέσω μιας σωστής εκπαίδευσης θα επιτυγχανόταν αυτό. Και βρήκαν αυτό που έψαχναν στη εγκύκλιος παιδεία, τις ελευθέριες τέχνες της ελληνιστικής περιόδου. Άτομα τα οποία το είπαν αυτή ήταν ο Κικέρον, ο Βάρρος και ο Βιτρούβιος. Αυτό που είχαν στο νου τους ήταν μία γενική απόκτηση γνώσης για έναν επιστήμονα δεν είχε να κάνει (απαραίτητα) με την επαγγελματική του κατάρτιση πάνω στη λογική, φυσική και ηθική, δηλαδή να γίνει φιλόσοφος. Αντί αυτού στο νου τους είχαν μια γενική απόκτηση γνώσεων και πειθαρχίας. (Vitruvius and the liberal arts of arcitecture)





% Οι πληροφορίες που υπάρχουν πηγάζουν από το ίδιο του το έργο \cite[σ.~390]{masterson_status_2004}.

% Ο Βιτρούβιος είναι γνωστός για το έργο του "Περί αρχιτεκτονικής" (De architectura), ένα σύνολο συγγραμμάτων, βιβλίων τα οποία περιγράφουν τη δομή της αρχιτεκτονικής.**Τα επαγγέλματα των μηχανικών κανονικά μεταφέρονταν από γονιό σε παιδί.  *** 

% Ο Αύγουστος χρησιμοποίησε σαν ευκαιρία την παρουσία των νέων ατόμων, ειδικών  για να αυξήσει την ήδη υψηλή πολιτική και στρατιωτική του δύναμη. Ο Βιτρούβιος ήταν ένα από αυτά τα άτομα. (Status, Pay, Pleasure p.1) *** και στη περίπτωση του Βιτρούβιου οι γονείς φρόντισαν για την εκπαίδευση του σαν  αρχιτέκτονας. Ο Βιτρούβιος μάλιστα κάνει αναφορά στο σύγγραμμά του στους δασκάλους του που τον εκπαίδευσαν. (Λέφας σελ 13).

% Στη περίοδο που έζησε, της ύστερης Ρωμαϊκής δημοκρατίας, οι αρχιτέκτονες όφειλαν να υποστηρίζουν το λαϊκό κόμμα του Καίσαρα και του κλώδιου (Λέφας σελ 13). Συμπεραίνεται πως ο Βιτρούβιος άσκησε το επάγγελμα του αρχιτέκτονα στο στρατό του αυτοκράτορα Καίσαρα με ιδιότητα ως τεχνικός στρατιωτικών μηχανών όπου συμπεραίνουμε ότι ο ίδιος είχε ακολουθήσει τον στρατό του Καίσαρα σε διάφορες μετακινήσεις από τμήματα της Γαλλίας μέχρι την Αφρική. Οι αναμνήσεις και η εμπειρίας από τα ταξίδια του μεταφέρονται και στο έργο του. φαίνεται πως ο ίδος ήαν ένας "aparitor", δηλαδή ένας δημόσιος υπάλληλος του οποίου ο μισθός προερχόταν από το δημόσιο ταμέιο.

% Η εποχή που έζησε ο Βιτρούβιος είχε να κάνει με την αλλαγή. Μία εποχή όπου οι τέχνες και τα επαγγέλματα είχαν πια αναβαθμιστεί, κερδίζοντας νέα σημασία σε σχέση με την παράδοση. Η κλασική ανατροφή, όπου είχε θρέψει την κλασική παλιά ρωμαϊκή κουλτούρα δεν ήταν πια σχετική. Η επαγγελματική εκπαίδευση είχε πια ξεπεράσει την κλασική. Βρήκαν πάτημα στην ελληνιστική παράδωση. έτσι και η αρχιτεκτονική. (Vitruvius and the liberal arts of architecture)



\subsection{Το έργο του} 

Ο Βιτρούβιος είναι γνωστός για την συγγραφή του "Περί αρχιτεκτονικής" (De 
architectura), ένα σύνολο από δέκα συγγράμματα όπου αναλύουν και περιγράφουν 
τις δομές της αρχιτεκτονικής και την αξία του αρχιτέκτονα.
Το σύγγραμμα ολοκληρώθηκε και δημοσιεύτηκε κατά την τελευταία περίοδο ζωής του 
αρχιτέκτονα και πιθανότατα κατά τη περίοδο όπου αυτοκράτορας ήταν πια ο 
Αύγουστος, υιοθετημένος ιός του Καίσαρα. ***
Τα βιβλία αποτελούνται από τεκμηριωμένες και ολοκληρωμένες σκέψεις και 
παρατηρήσεις του συγγραφέα τα οποία είναι αφιερωμένα στον ίδιο τον Αυτοκράτορα. 
Οι λόγοι που τον οδήγησαν να συγγράψει το περί αρχιτεκτονικής είναι μάλλον προς 
παροχή υποστήριξης στο οικοδομικό πρόγραμμα του Αύγουστου και στην ενημέρωση 
πάνω στην αρχιτεκτονική. (Λέφας σελ 14-15)***

Το περιεχόμενο αφορά τους τομείς του αρχιτέκτονα (1.2-3),την εκπαίδευση και τη 
συμπεριφορά του (1.6), και τα ορθά χαρακτηριστικά των δημοσίων και ιδιωτικών 
κτισμάτων (βοοκ 5-6), Υλικά (Book 2), ορθές τοποθεσίες κτηρίων (1.4-7). 
(Status, Pay, Pleasure p.6)

Στον πρόλογο του πρώτου βιβλίου αποκαλύπτει το κύριο στόχο του, είδη 
συλλογισμών πάνω στην αρχιτεκτονική. Στην αρχή του πρώτου βιβλίου συζητάει την 
γνώση του αρχιτέκτονα. έπειτα στο δεύτερο κεφάλαιο του πρώτου βιβλίου αναλύει 
την θεωρητική δομή της αρχιτεκτονικής. Από το τρίτο κεφάλαιο και μετά αφιερώνει 
το υπόλοιπο έργο του στο πρακτικό κομμάτι της αρχιτεκτονικής. (vitruvious arts 
of architectue p.2)

Το πιο σημαντικό ίσως τμήμα του συγγράμματος είναι η απαιτούμενη γνώση που 
οφείλει να έχει ένας αρχιτέκτονας.