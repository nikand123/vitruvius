\section{Το έργο του} 

Το φημισμένο έργο του \emph{"Περί Αρχιτεκτονικής"} ("De Architectura"), αποτελεί ένα από τα πλέον μελετημένα και μεταφρασμένα έργα της αρχαιότητας. Αν και το πρωτότυπο έργο δεν έχει διασωθεί η συνέχειά του στην αρχική του μορφή (λατινικά) οφείλεται σε μία αρχική αντιγραφή του Poggio Bracciolini το 1414. Έκτοτε το συγκεκριμένο έργο έχει μεταφραστεί σε δεκάδες γλώσσες από ένα μεγάλο πλήθος μελετητών και μεταφραστών.

Σαν έργο θεωρείται ίσως το σημαντικότερο γραπτό κείμενο στην ιστορία της αρχιτεκτονικής. Σε αυτό ο Βιτρούβιος περιγράφει όχι μόνο τις αρχές της αρχιτεκτονικής αλλά και τα ποιοτικά χαρακτηριστικά που πρέπει να έχει ο αρχιτέκτονας. Συγχρόνως έχει τη μορφή εγχειριδίου που κάθε αρχιτέκτονας θα πρέπει να διαθέτει και δεν θα ήταν υπερβολή να πούμε ότι η αξία του έχει μεταφερθεί μέχρι και τις μέρες μας, σχεδόν αναλλίωτη \cite[σ. 16-18]{vitruvius-lefas}.

Απαρτίζεται από 10 βιβλία που έγραψε ο Βιτρούβιος στη περίοδο της ζωής του για τον αυτοκράτορα Αύγουστο. Φαίνεται πως η χρονική σειρά με την οποία έχουν γραφτεί ξεκινάει από τα βιβλία 8, 9, 10 κατά το πρώτο στάδιο της σταδιοδρομίας του και στη συνέχει τα υπόλοιπα, με κανονική σειρά, 1, 2, 3 κλπ. τα οποία γράφτηκαν κατά το τελευταίο στάδιο της ζωής του και δημοσιεύτηκαν. Τα βιβίλα Περιέχουν ολοκληρωμένες και τεκμηριωμένες σκέψεις του συγγραφέα, ενώ αποτελούνται από ένα κύριο μέρος (εγχειρίδια, οικοδομικές περιγραφές και πληροφορίες πάνω σε κατασκευές) αλλά και από τους προλόγους οι οποίοι απευθύνονται στον Αύγουστο. Οι λόγοι που τον οδήγησαν να συντάξει το έργο του έγιναν με σκοπό να προσφέρει θεωρητικές αρχές πάνω στην αριτεκτονική, να παρέχει ενημέρωση σε αρχιτέκτονες και ιδιώτες χρηματοδότες αλλά και προσωπικής προβολής \cite{vitruvius-lefas}.

\subsection{Γλώσσα}

Ως προς τη γλώσσα και το ύφος των βιβλίων είναι κοινά αποδεκτό ότι είναι δυσνόητα ή στη καλύτερη περίπτωση δεν ακολουθούν το ύφος και το στυλ των πεζογράφων και ποιητών του πρώτου π.Χ. αιώνα (π.χ. Κικέρων). Ειδικότερα η χρησιμοποιούμενη λατινική γλώσσα αποτέλεσε εμπόδιο σε πολλές προσπάθειες μετάφρασης και ερμηνείας σε σημείο που ορισμένοι μελετητές όπως ο Frank Granger να προτέινουν "διορθώσεις" στα όχι τόσο κατανοητά μέρη  \cite[σ. 19]{vitruvius-lefas}. Σίγουρα όμως τα βιβλία είναι ενδεικτικά της τεχνικής παιδείας και γνώσεων του συγγραφέα. Χρησιμοποιεί ευρήτατα τεχνικούς όρους και εκφράσεις όπου μάλιστα πολλοί από αυτούς είναι δανισμένοι από τους αντίστοιχους αρχαιοελληνικούς.

\subsection{Τα βιβλία}

Το πρώτο βιβλίο, το οποίο αποτελέι και το κύριο μέρος μελέτης της παρούσας εργασίας, μιλάει για την αρχιτεκτονική και τη σχέση της με τις άλλες επιστήμες, τις βασικές έννοιές της, τις αρχές της αλλά και για την πολεοδομία. Τα υπόλοιπα βιβλία αφορούν περισσότερο τεχνικές αναφορές και αναλύσεις του Βιτρούβιου και δεν θα αναλυθούν στην εργασία.


%Ο Βιτρούβιος είναι γνωστός για την συγγραφή του "Περί αρχιτεκτονικής" (De architectura), ένα σύνολο από δέκα συγγράμματα όπου αναλύουν και περιγράφουν τις δομές της αρχιτεκτονικής και την αξία του αρχιτέκτονα.

%Το σύγγραμμα ολοκληρώθηκε και δημοσιεύτηκε κατά την τελευταία περίοδο ζωής του αρχιτέκτονα και πιθανότατα κατά τη περίοδο όπου αυτοκράτορας ήταν πια ο Αύγουστος, υιοθετημένος ιός του Καίσαρα. ***
%Τα βιβλία αποτελούνται από τεκμηριωμένες και ολοκληρωμένες σκέψεις και παρατηρήσεις του συγγραφέα τα οποία είναι αφιερωμένα στον ίδιο τον Αυτοκράτορα. 
%Οι λόγοι που τον οδήγησαν να συγγράψει το περί αρχιτεκτονικής είναι μάλλον προς παροχή υποστήριξης στο οικοδομικό πρόγραμμα του Αύγουστου και στην ενημέρωση πάνω στην αρχιτεκτονική. (Λέφας σελ 14-15)***

%Το περιεχόμενο αφορά τους τομείς του αρχιτέκτονα (1.2-3),την εκπαίδευση και τη συμπεριφορά του (1.6), και τα ορθά χαρακτηριστικά των δημοσίων και ιδιωτικών κτισμάτων (βοοκ 5-6), Υλικά (Book 2), ορθές τοποθεσίες κτηρίων (1.4-7). (Status, Pay, Pleasure p.6)

%Στον πρόλογο του πρώτου βιβλίου αποκαλύπτει το κύριο στόχο του, είδη συλλογισμών πάνω στην αρχιτεκτονική. Στην αρχή του πρώτου βιβλίου συζητάει την γνώση του αρχιτέκτονα. έπειτα στο δεύτερο κεφάλαιο του πρώτου βιβλίου αναλύει την θεωρητική δομή της αρχιτεκτονικής. Από το τρίτο κεφάλαιο και μετά αφιερώνει το υπόλοιπο έργο του στο πρακτικό κομμάτι της αρχιτεκτονικής. (vitruvious arts of architectue p.2)

%Το πιο σημαντικό ίσως τμήμα του συγγράμματος είναι η απαιτούμενη γνώση που οφείλει να έχει ένας αρχιτέκτονας.