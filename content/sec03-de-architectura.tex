\section{Το έργο του} 

Το πλέον φημισμένο έργο του Βιτρούβιου και αυτό που τον έκανε γνωστό διαχρονικά 
είναι το \emph{"Περί Αρχιτεκτονικής"} ("De Architectura" -- DA για συντομία). 
Αποτελεί ένα από τα πλέον μελετημένα και μεταφρασμένα έργα της αρχαιότητας, 
ανεξαρτήτως θέματος. Αν και το πρωτότυπο έργο δεν έχει διασωθεί η συνέχειά του 
στην αρχική του μορφή (λατινικά) οφείλεται σε μία αρχική αντιγραφή του Poggio 
Bracciolini το 1414. Έκτοτε το συγκεκριμένο έργο έχει μεταφραστεί σε δεκάδες 
γλώσσες από ένα μεγάλο πλήθος μελετητών και μεταφραστών.

Σαν έργο θεωρείται ίσως το σημαντικότερο γραπτό κείμενο στην ιστορία της αρχιτεκτονικής. Σε αυτό ο Βιτρούβιος περιγράφει όχι μόνο τις αρχές της αρχιτεκτονικής αλλά και τα ποιοτικά χαρακτηριστικά που πρέπει να έχει ο αρχιτέκτονας. Συγχρόνως έχει τη μορφή εγχειριδίου που κάθε αρχιτέκτονας θα πρέπει να διαθέτει και δεν θα ήταν υπερβολή να πούμε ότι η αξία του έχει μεταφερθεί μέχρι και τις μέρες μας, σχεδόν αναλλοίωτη \cite[σ. 16-18]{vitruvius-lefas}.

\subsection{Τα Δέκα Βιβλία της Αρχιτεκτονικής}

Το DA απαρτίζεται από 10 βιβλία με αντικείμενα σχετικά με την αρχιτεκτονική τα οποία ο Βιτρούβιος έγραψε για τον αυτοκράτορα Αύγουστο. Φαίνεται πως η χρονική σειρά με την οποία έχουν γραφτεί ξεκινάει από τα βιβλία 8, 9, 10 κατά το πρώτο στάδιο της σταδιοδρομίας του και στη συνέχει τα υπόλοιπα, με κανονική σειρά, 1, 2, 3 κλπ. τα οποία γράφτηκαν κατά το τελευταίο στάδιο της ζωής του και δημοσιεύτηκαν. Τα βιβλία Περιέχουν ολοκληρωμένες και τεκμηριωμένες σκέψεις του συγγραφέα, ενώ αποτελούνται από ένα κύριο μέρος (εγχειρίδια, οικοδομικές περιγραφές και πληροφορίες πάνω σε κατασκευές) αλλά και από τους προλόγους οι οποίοι απευθύνονται στον Αύγουστο. Οι λόγοι που τον οδήγησαν να συντάξει το έργο του έγιναν με σκοπό να προσφέρει θεωρητικές αρχές πάνω στην αρχιτεκτονική, να παρέχει ενημέρωση σε αρχιτέκτονες και ιδιώτες χρηματοδότες αλλά και προσωπικής προβολής \cite{vitruvius-lefas}.

Στην εργασία ασχολούμαστε αποκλειστικά με την περιγραφή και μερική ανάλυση του Bιβλίου I. Περιληπτικά, τα υπόλοιπα βιβλία πραγματεύονται τα εξής \cite{erismis_critical_2013}:

\begin{itemize}[noitemsep]
\item Βιβλίο II υλικά οικοδόμησης και μεθόδους,
\item Βιβλία III και IV θρησκευτική αρχιτεκτονική,
\item Βιβλίο V άλλες μορφές δημόσιας αρχιτεκτονικής με έμφαση στα θέατρα,
\item Βιβλίο VI και VII οικιακή αρχιτεκτονική και θέματα όπως είδη δαπέδων, ζωγραφική, χρώματα κ.λπ.,
\item Βιβλίο VIII κατασκευή υδραγωγείων και αγωγών για μεταφορά νερού
\item Βιβλίο IX μετά από μία εκτεταμένη εισαγωγή στην Αστρονομία, περιγράφει διάφορες μορφές ρολογιών και κλίσεων
\item Βιβλίο X μηχανική με ιδιαίτερη αναφορά σε συστήματα κίνησης νερού, μέτρησης αποστάσεων οδών αλλά και στρατιωτικά μηχανήματα.
\end{itemize}

\subsection{Η Γλώσσα του Βιτρούβιου}

Ως προς τη γλώσσα και το ύφος των βιβλίων είναι κοινά αποδεκτό ότι είναι 
δυσνόητα ή στη καλύτερη περίπτωση δεν ακολουθούν το ύφος και το στυλ των 
πεζογράφων και ποιητών του πρώτου π.Χ. αιώνα (π.χ. Κικέρων). Ειδικότερα η 
χρησιμοποιούμενη λατινική γλώσσα αποτέλεσε εμπόδιο σε πολλές προσπάθειες 
μετάφρασης και ερμηνείας σε σημείο που ορισμένοι μελετητές όπως ο Frank Granger 
να προτείνουν "διορθώσεις" στα όχι τόσο κατανοητά μέρη  \cite[σ. 
19]{vitruvius-lefas}. Σίγουρα όμως τα βιβλία είναι ενδεικτικά της τεχνικής 
παιδείας και γνώσεων του συγγραφέα. Χρησιμοποιεί ευρύτατα τεχνικούς όρους και 
εκφράσεις όπου μάλιστα πολλοί από αυτούς είναι δανεισμένοι από τους 
αντίστοιχους αρχαιοελληνικούς.