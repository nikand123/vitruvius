\section{Τα Ποιοτικά Χαρακτηριστικά του Αρχιτέκτονα}

Από την εισαγωγή ήδη του βιβλίου, ο Βιτρούβιος αφήνει να φανεί η σημασία που αποδίδει στο θέμα της γνώσης καθώς και στη μορφή της (θεωρητική, πρακτική). Ξεκινώντας ήδη ο Βιτρούβιος κάνει σαφές ότι δεν μπορούμε να μιλάμε για αρχιτέκτονα και αρχιτεκτονικό έργο άξιο να μελετηθεί και να ασχοληθεί κάποιος με αυτό, εφόσον δεν στηρίζεται σε κάποιο επαρκές γνωσιακό υπόβαθρο. Κάνει μία εκτενή αναφορά στα διαφορετικά είδη γνώσης στα οποία θα πρέπει να επενδύσει ο αρχιτέκτονας καθώς και στην πρακτική εξάσκησή τους.

% Στο έργο του περιγράφει αναλυτικά το τεχνικό κομμάτι της αρχιτεκτονικής. Κατασκευή και περιγραφή για το πως πρέπει να κατασκευάζονται οι ναοί, δημόσιος χώρος, υλικά κλπ. Πέρα όμως από το τεχνικό χαρακτήρα και τη λειτουργία του ως εγχειρίδιο για τους αρχιτέκτονες και άλλους ενδιαφερόμενους, το \emph{Περί Αρχιτεκτονικής} κατέχει και έναν πιο θεωριτικό ρόλο στο κλάδο της επιστήμης της αρχιτεκτονικής. Ίσως, το πιο σημαντικό κομμάτι του έργου του Βιτορύβιου είναι οι θεωρίες του και απόψεις του πάνω στην αρχιτεκτονική καθ' αυτού. 

\subsection{Η γνώση του αρχιτέκτονα}

Για τον Βιτρούβιο, ένας αρχιτέκτονας οφείλει να κατέχει ένα ευρύ πεδίο γνώσεων ώστε το αποτέλεσμα της εργασίας του να έιναι αξιομνημόνευτο. Μάλιστα αφιερώνει σχεδόν ολόκληρη την ενότητα πάνω στο συγκεκριμένο θέμα, πράγμα που φανερώνει πως είναι ιδιαίτερα σημαντικό και άξιο προς συζήτησης για τον ίδιο.

Ο αρχιτέκτονας οφείλει να είναι εφοδιασμένος με μία ποικιλία γνώσεω, που πρέπει να εκτείνονται από τη γεωμετρία (με την σημερινή ερμηνεία της λέξης) μέχρι και ιστορία, φιλοσοφία, μουσική, αλλά και ιατρική, νομική και αστρονομία \cite[σ. 392]{masterson_status_2004}. Αξιολογώντας τη διαχρονική αξία αυτής της θέσης\sidenote%
    {Σήμερα πιθανώς να φαίνεται παράδοξο, ίσως και ασυμβίβαστο η   κατοχή γνώσεων μουσικής, ιατρικής κ.λπ. παράλληλα με τις καθαρά τεχνικές γνώσεις τη αρχιτεκτονικής, καθώς μπορεί κάποιος να αναρωτηθεί για παράδειγμα, σε τι μπορεί να συμβάλλει η εξοικίωση με
    την επιστήμη της αστρονομίας πάνω στη σχεδίαση μίας
    τυπικής κατοικίας.},
επικαλούμαστε ξανά την επιχειρηματολογία του Βιτρούβιου: με την ατρονομία, ο αρχιτέκτονας μαθαίνει να προσανατολίζεται, γνωρίζει που βρίσκεται ο Βοράς, ο Νότος, η Δύση και η Ανατολή αλλά και αντιλαμβάνεται τις ιδιότητες των σημείων αυτών του ορίζοντα και την επίδραση που έχει η σωστή τοποθέτηση και φορά ενός κτηρίου σε σχέση με αυτά \cite[σ. 43]{vitruvius-lefas}. Γενικότερα υποστηρίζει πως όλες οι επιστήμες και οι κλάδοι τους έχουν μία κοινή σχέση μεταξύ τους την οποία αποδίδει με τον αρχαιοελληνικό όρο της \emph{εγκυκλίου παιδείας} (encyclios disciplina). Ο όρος έχει να κάνει ακριβώς με αυτή την απαραίτητη πολυμαθεία του αρχιτέκτονα. Όπως λεεί, μόνο η επιστήμη της αρχιτεκτονικής περιβάλλεται από τόσες άλλες επιστήμες, καθιστώντας έτσι τον αρχιτέκτονα ικανό να φτάσει στο ανώτερο επίπεδο της τέχνης του που είναι το \emph{κατασκευάζειν} \cite[σ. 45]{vitruvius-lefas}.

Φυσικά ο Βιτορύβιος δεν αναμένει από κανέναν να αριστέυει σε όλους τους τομείς των επιστημών, καθώς πιθανώς είναι πρακτικά αδύνατο. Δεν είναι δυνατόν ούτε μπορεί ένας αρχιτέκτονας, όπως αναφέρει, να είναι φιλόσοφος όπως ο Αρίσταρχος, μουσικός όπως ο Αριστόξενος ή ιατρός όπως ο Ιπποκράτης, καθώς κανένας άνθρωπος δεν μπορεί να είναι αυθεντία σε ολες τις τέχνες και επιστήμες, να κατέχει δηλαδή σε βάθος τη θεωρία τους. Ο Βιτορύβιος αντίθετα, ζητάει από τον αρχιτέκτονα μία τουλάχιστον εξοικείωση σε αυτούς τους τομείς της τέχνης και των επιστημών \cite[σ.~45]{vitruvius-lefas}. 

\subsection{Θεωρία και πράξη}

Το χαρακτηριστικό που επιτρέπει στον  ιδανικό αρχιτέκτονα να ξεχωρίσει από έναν κοινό τεχνίτη είναι η υπεροχή του στα γράμματα \cite{masterson_status_2004}. Φυσικά το επάγγελμα του τεχνίτη είναι εξίσου σημαντικό αλλά για τον Βιτρούβιο, η διαφορά είναι πως ο αρχιτέκτονας δεν βασίζεται αποκλειστικά στο τεχνικό, πρακτικό κομμάτι της κατασκευής αλλά στηρίζεται και πάνω σε ένα θεωρητικό υπόβαθρο. Είναι χαρακτηριστική η αναφορά του Βιτρούβιου:

\begin{quote}
"Οι αρχιτέκτονες, λοιπόν, που χωρίς γράμματα αρκέστηκαν στην πρακτική εξάσκηση,  δεν κατόρθωσαν να δημιουργήσουν [έργα] με κύρος ανάλογο του μόχθου που κατέβαλαν. Αυτοί που εμπιστέυθηκαν αποκλειστικά την θεωρία και τα γράμματα κυνήγησαν, όπως φαίνεται, όχι τα ίδια τα πράγματα, αλλά τη σκυά τους. Όσοι όμως κατείχαν και τα δύο επέτυχαν..." [μετάφραση Λέφας σ. 37, παρ 2]
\end{quote}

Χρησιμοποιεί μάλιστα ειδικούς όρους για να περιγράψει τις δύο αυτές διαφορετικές μορφές γνώσης: \emph{Ratiocinatio} (Θεωρία) και \emph{Fabrica} (Πράξη), τις οποίες θα αναλύσουμε στη συνέχεια. Θα πρέπει όμως πρώτα να διευκρινήσουμε τα εξής:

\begin{itemize}
\item Οι συγκεκριμένοι όροι, όπως και πολλοί άλλοι που χρησιμοποιεί, δεν έχουν απαραίτητα την ίδια έννοια που έχουν στη σύγχρονη γλώσσα. Σαφώς, έχοντας υπόψη μας την περίοδο που έζησε ο Βιτρούβιος αλλά και τις επιρροές του από τον αρχαιοελληνικό πολιτισμό, οι μελετητές μπορούν εώς ένα σημείο μόνο να μεταφράσουν και να αναλύσουν άμεσα την ορολογία του Βιτρούβιου, ενώ σε πολλές περιπτώσεις \sidenote%
    {Όπως ήδη αναφέρθηκε, η γλώσσα του Βιτρούβιου είναι σχετικά περίπλοκη ίσως και δυσνόητη, με αποτέλεσμα οι μεταφράσεις να έχουν δημιουργήσει μία σχετική σύγχυση για την πραγματική ουσία του κειμένου του συγγραφέα. Οι όροι και οι έννοιες που χρησιμοποιεί και η τοποθέτησή τους στο κείμενο, είναι αντικείμενο έντονης συζήτησης μεταξύ των μελετητών ως προς την εννοιολογική τους ερμηνεία.}
χρειάζεται να βασίσουν την ερμηνεία και τις αναλύσεις τους, ανατρέχοντας στις γλωσσικές ρίζες των όρων που χρησιμοποιεί. Με αυτό τον τρόπο επιδιώκεται η αποκρυπτογράφηση του βαθύτερου νοήματος που προσπαθέι να μεταφέρει ο Βιτρούβιος \cite{graham-education}.
\item Η άποψη του Βιτρούβιου περί θεωρίας και πράξης και η σύνδεσή τους με την ιδιότητα του αρχιτέκτονα, φαίνεται να είναι πολύ κοντά στη σχετική Πλατωνική ερμηνία. Για τους αρχαίους Έλληνες δεν υπήρχε ιδιαίτερος ορισμός για την ιδιότητα του αρχιτέκτονα, που να τον ξεχωρίζει από αυτή του τεχνίτη. Για τον Πλάτωνα λοιπόν, ο αρχιτέκτονας δεν είναι ένας τυπικός τεχνίτης αλλά ένα ορθά μελετημένο άτομο το οποίο κατέχει το πρακτικό κομμάτι του χτισίματος αλλά και γνώσεις πάνω στις επιστήμες, δηλαδή το θεωρητικό υπόβαθρο. Η \emph{πρακτική} και \emph{η γνωστική} του Πλάτωνα συνθέτουν την ίδια την επιστήμη. \cite{graham-education}
\end{itemize}

\begin{description}[style=nextline]
\item[Fabrica]

Η λέξη Fabrica ερμηνεύεται γενικά ως \emph{τέχνη} αλλά με τη γενική έννοια της \emph{Πράξης} \cite{vitruvius-lefas,graham-education}. Η έννοια δηλαδή της επεναλαμβανόμενης άσκησης του χεριού ή η ενασχόληση με τα πράγματα. Στην ουσία είναι η αρχιτεκτονική ενατένιση, το πρακτικό κομμάτι του "ξέρω πως", δηλαδή έχει να κάνει με την ανάλυση του χώρου, τη χωροταξία, κλπ. Με άλλα λόγια περιλαμβάνει την επαγγελματική γνώση του αρχιτέκτονα \cite{graham-education}. Ο Βιτρούβιος όμως, περιγράφωντας τη έννοια της fabrica, της δίνει και μία ιδιαίτερη σημασία, αυτό της πνευματικής ενασχόλησης. Όντως την συνδέει με τη λέξη \emph{meditatio}, η οποία ερμηνεύεται ως μία διαδικασία μελέτης, περισυλλογής ή σκέψης. Όπότε μέσω της διαδικασίας του meditatio αποκτάται η απαιτούμενη επαγγελματική γνώση και εμπειρία του αρχιτέκτονα. Αυτή η προσέγγιση είναι πολύ κοντά στη σχετική άποψη του Πλάτωνα\sidenote%
{Τόσο ο Πλάτωνας όσο και ο Βιτρούβιος, ερμηνεύουν τη πνευματική συνιστώσα της πρακτικής γνώσης μέσα από το παράδειγμα της μουσικής. Η μουσική σαν πράξη αποτελείται τόσο από το πρακτικό κομμάτι δηλαδή τη διαδικασία παραγωγής του ήχου μέσα από την η ενασχόληση με το μουσικό οργανο, όσο και την θεωρητική σύλληψη της μουσικής σύνθεσης, που είναι ένα καθαρά θεωρητικό κομμάτι σχετιζόμενο με τα μαθηματικά, την αρμονία και τη μεταφυσική.}
, ο οποίος αν και περιγράφει την πράξη σαν ένα είδος πρακτικής γνώσης η οποία είναι περισσότερο επιτακτική ή επαγγελματική παρά αποκλειστικά προϊόν πνευματικής νόησης (φιλοσοφική, επιστημονική ή μαθηματική), εντούτοις η \emph{πρακτική} και \emph{γνωστική} του αρχιτέκτονα είναι και οι δύο είδη πνευματικής γνώσης \cite{graham-education}.  


\item[Ratiocinatio]

Η λέξη ratiocinatio ερμηνεύεται ως \emph{Θεωρία}, με την έννοια του υπολογισμού, του σχεδιασμού.  Όπως και η fabrica έτσι και το ratiοcinatio αποτελεί μία πνευματική διαδικασία με διαφορά όμως ότι πραγματοποιείται καθαρά σε πνευματικό επίπεδο, ενώ η Fabrica έχει να κάνει περισσότερο με τη χειρονακτική ενασχόληση με τα πράγματα (meditaio) \cite{vitruvius-lefas}.

Το ratiocinatio έχει ως βάση τη λέξη \emph{ratio} που μεταφράζεται ως \emph{λόγος}. Στην αρχαιοελληνική γραμαμτική η συγκεκριμένη λέξη έχει περισσότερες από μία έννοιες, όπως θεωρία, περιγραφή ή λόγος με την μαθηματική έννοια δηλαδή η σχέση μεταξύ δύο αριθμών. Ειδιαίτερα μάλιστα η τελευταία αυτή ερμηνεία του λόγου είναι κοινή με το λατινικό \emph{ratio} και το Σωκρατικό παράγωγό τους, της \emph{λογικής} (rational). Βέβαια η σύνδεση αυτή της θεωρίας με τη λογική έχει προκαλέσει πολλά προβλήματα ερμηνείας καθώς πολλοί μεταφραστές αδυνατούν να καταλήξουν σε ένα κοινό συμπέρασμα για την έννοια της λέξης ratiocinatio, αποδίδοντας διαφορετικές ερμηνίες. Είναι χαρακτηριστικό ότι η φράση: 
\begin{quote}
"Ratiocinatio autem est quae resfabricates sollertiae acrationis pro portione (proportione) demonstrare atque explicare potest"
\end{quote}
δυσκολεύει ακόμα τους μελετητές \cite{graham-education}.
\end{description}

%Οι περισσότεροι μεταφραστές αδυνατούν να καταλήξουν σε ένα κοινό συμπέρασμα για την έννοια της λέξης ratiocinatio. Η πρόταση:  \textit{"Ratiocinatio autem est quae resfabricates sollertiae acrationis pro portione (proportione) demonstrare atque explicare potest"}, δυσκολεύευι ακόμα τουςμελετητές. Η σύγχυση έχει να κάνει με τη σύνταξη της πρότασης. Κυρίως με τη λέξη \emph{pro portione (proportione)} για το εάν χρησιμοποιείται με τεχνική έννοια που αφορά τις ειδικές γνώσεις και δεξιότητες του αρχιτέκτονα στη χρήση μαθηματικών αναλογιών ή εάν 

%Φαίνεται πως η πρόθεση του Βιτρούβιου είναι να σχηματίσει μία θεωρία που αφορά την αναλογία. Η αναλογία αυτή μπορεί να αφορά είτε τη σχέση μεταξύ των αντικειμένων είτε την αριθμητική αναλογία. \cite[σ. 86-87]{vitruvius-lefas}







% Όπως προαναφέρθηκε η γλώσσα του Βιτρούβιου είναι αρκετά περίπλοκή και αποτελείκομμάτι σύγχησης μεταξύ μεταφραστών. Η πρόταση:  Ratiocinatio autem est quae resfabricates sollertiae acrationis pro portione demonstrare atque explicare potest, δυσκολεύευι ακόμα τουςμελετητές. (sollertioa p.4)

% Πρόθεση του Βιτρούβιου είναι να σχηματίσει μία έννοια, θεωρεία που αοφρά τηναναλογία. Η λε΄ξη κατα αυτή δεν παρατειρέιται στο κείμενο του. 
% Η λέξηratiocinatio έχει ως βάση τη λέξη ratio δηλαδή λόγος. Ο λόγος στα ελληνικά έχειμια πληθόρα εννοιών.
% Η μαθηματική του έννοια είναι η σχέση μεταξύ δυο αριθμών. Ηλέξ λόγος είχε να κάνει με την τη σχέση μεταξύ δύο αριθμών (couple of twonumbers) πριν πάρει την σημερινή έννοια με βάση τον Ευκλείδη, δηλαδή το πιλίκοδύο αριθμών. (sollertiae p.5)

% Η λέξη ratio πηγάζει από την ελληνική λέξη λόγος. Με την ένοια λόγος μπορούμενα αναφερθούμε στη θεωρεία, επεξήγηση, περιγραφή, λόγος με τη μαθηματικήέννοια.. κλπ. 
% Ο Σοκράτης αναφερόταν στη λέξη λόγος ως τον λογικό ορισμόεννοιών, ακριβώς όπως και οι διάδοχοι χρησιμοποιούν τον όρο. Ομολιως και οιλέξεις ratio, rationale και ratiocinatio. (παρ. 14)

% Η θεωρεία δηλαδή έχει να κάνει με τη θεωρεία της αναλογίας (ratio). 

% Ο βιτρούβιος χρησημοποιεί σαν παράδειγμα τη σχέση ενός μουσικού μέ έναν ιατρό.Σημειώνει ως κοινό στοιχείο μεταξύ των δύο αυτών επαγγελμα΄τον τον ρυμό. Από τημία πλευρά την κίνηση του ποδιού ενός μουσικού ωστε να μετράει σωστά τις νότεςκαι από την άλλη την παλμό του ασθενή που μετράει ένας ιατρός. Ο Βιτρούβιοςκάνει αναφορά στον Αριστόξενος ο Ταραντίνος Ο οποίος έδωσε ιδηέτερη σημασία στηθεωρία της μουσικής αναλογίας, κάτι το ποίο δεν ήταν εγκεκριμένο από τηναρχαιότητα ** Σχέση μεταξύ της ακουστικής αρμονίας και της χωρικής αναλογίας. 

% Η διαφορά του ύψους των ήχων βρσίκεται στην αναλογία των νούμεεων και τονταχητήτων. Κατά τον Αριστόξενο η πρώταση αυτή είναι λάθος. Υποστηρίζει πως οιμουσικοί ήχοι έχουν να κάνου αποκλειστικά με την αίσθηση της ακοής και όχι μετην αναλογία. Ο ίδιος δεν ήταν μαθηματικός, έδινε σημασία όμως στη αίθσηση. 

% Όπως φαίνεται για τη λέξη ratiocinatio υπάρχει ακόμα σύγχηση. Το τμήμα τουκειμένου του Βιτρούβιου που περιγράφει τον ορισμό δεν είναι εύκολο ναμεταφρστεί. Όλοι οι μεταφραστές παρουσιάζουν διαφορετικές εκδοχές. Σκατά

% Ratiocinatio autem est, quae res fabricatas sollertiae ac rationis proportionedemonstrare atque explicare potest

% Φαέινεται να υπάρχει σύγχηση με την λέξη proportione ή pro portione. Ο λέφαςτη λέει μία λέξη. Σύμφωνα με την ικανότητα και τη συλλογιστική του αρχιτέκτονα.








% Από τους περισώτερους μεταφραστές υποστηρίζεται πως η λέξη fabrica μεταφράζεταιως πρακτική, πράκτις.
% Με την ένοια δηλαδή της επαναλαμβανόμενης άσκησης τουχεριού. Η λέξη έτσι όπως χρησημοποιείται από τον Βιτρούβιο δεν σημαίνει πρακτικόκτίσμα ή την τέχνη της κατασκευής. (παρ.7)

% Κατά τον μεταφραστή Joseph Gwilt η λέξη fabrica σημαίνει συχνή και συνεχήπερισυλλογή (meditatio) των τεχνών σχετικά με τη διαδικασία του χτισήματος. 
%Ηδιαδικασία αυτή είανι πνευματική και λετσι η fabrica μέσω της διαδικασίαςmeditatio αποδίδει την απιτούμενη επαγγελματική γνώση και εμπειρία. 
%Στην ουσία ηfabrica δεν αφορά αποκλειστικά χειρονακτική γνώση, αν και επιτρέπεται, καιαποκτάται με την άμεση ενασχόληση στο κτίσιμο και ανάλογες τέχνες. (παρ 10)

% Η τέχνη κατά τον Βιτρούβιο είναι η εξάσκηση της αρχιτεκτονικής ενατένισης. Τορπατκό κομμάτο του "ξέρω πως". ¨εχει να κάνει με την ανάλυση του χώρου,χωροταξία κλπ. Περιλαμβάνει την όλη επαγκελματική γνώση του αρχιτέκτονα. (παρ11)



% Ο Πλάτωνας και ο Βιτρούβιος έχουν κοινές απόψεις πάνω στο πρακτικό και τοθεωρητικό.

% Κατά τη εποχή του Πλάτωνα δεν υπήρχε διαφορά μεταξύ του αρχιτέκτονα και του τεχνίτη. Ο αρχιτέκτονας είναι ο πρώτος τεχνίτης αλλά δεν ήταν εξακριβωμένο τι το ξεχώριζε. Εξάλλου δεν θεωρούταν από τις καλές τέχνες. (παρ.3-4)

% Ο Πλάτωνας υποστηρίζει πως η "πρακτική" (praktike) και η "γνωστική" (gnostike) αποτελούν συστατικά της ενότητας της επιστήμης στο σύνολό της. (παρ. 1)

% Παραδειγματίζει ένα διακριτικό είδος πρακτικής γνώσης το οποίο είναι επιτακτικήή εκτελεστική παρά καθαρά καίρια (φιλοσοφική, μαθηματική κλπ.). Έχει να κάνει μετην εντολή παρά με επιστημονικά γεγονότα ή υπολογισμούς. (παρ. 2)

% Τόσο και ο Πλάτωνας όσο και ο Βιτρούβιος χαρακτηρίζουν τον αρχιτέκτονα όχι ως έναν απλό τεχνίτη αλλά ως ένα ορθά μελετημένο ο οποίος κατέχει το πρακτικό κομμάτι του χτισίματος και ταυτόχρονα την γνώση πάνω σε επιστήμες. (παρ. 8)

% Για τον Πλάτωνα η πρακτική και η γνωστική είναι πνευματικές γνώσεις που δεν απαιτούν χειρονακτική ενασχόληση. Ακριβώς και ο Βιτρούβιος περιγράφει σαφέστατα πως η λέξη fabrica είναι επίσης συνδεδεμένη με τη πνευματική διαδικασία. Ονομάζει δε τη συγκεκριμένη διαδικασία meditatio. (παρ. 9)

% Η τέχνη της αρχιτεκτονικής απαιτεί γνώσεις που αφορούν το "ξέρω πως" και το "ξέρω αυτό". (παρ 12)

% Κι οι δύο συγκρίνουν την αρχιτεκτονική με τη μουσική. Από τη μία τοτελείως πρακτικό κομμάτι της μουσικής, δηλαδή η ενασχόληση με τα όργανα και η παραγωγή της μουσικής μέσω αυτών και από την άλλη το τελείως θεωρητικό κομμάτι, που αφορά τα μαθηματικά και τη μεταφυσική. (παρ. 12)

% Ο Βιτρούβιος απαιτεί από τον αρχιτέκτονα να κατέχει μία πληθώρα γνώσεων, από γεωμετρία και φιλοσοφία μέχρι και ιατρική. Τονίζει βέβαια ότι δεν απαιτείται η υπεροχή στους τομείς αυτούς αλλά μία τουλάχιστον ενασχόληση και εξοικείωση. 
% Υποστηρίζει πως όλοι οι κλάδοι της γνώσης συνδέονται και έχουν κάτι κοινό μεταξύ τους. (Λέφας σελ45 11-12)Φυσικά μία τέτοια απαίτηση σε γνώσεις απαιτεί παράλληλα μελέτη εκτενούς διάρκειας. (Status, Pay, Pleasure p.7)

% Αυτό που ξεχωρίζει τον αρχιτέκτονα από ένα τυπικό τεχνίτη είναι η υπεροχή στα γράμματα. Φυσικά η ιδιότητα του τεχνίτη είναι εξίσου σημαντική. ένας αρχιτέκτονας πρέπει να κατέχει και τις δύο κατηγορίες. Συνθετική υπεροχή (Status, Pay, Pleasure p. 9)

% Όλες οι τέχνες και επιστήμες έχουν μία κοινή σχέση μεταξύ τους, δηλαδή συνδέονται. Ο Βιτρούβιος χρησιμοποιεί αυτή την πρόταση σαν επιχείρημα για το λόγω που ο αρχιτέκτονας πρέπει να κατέχει μία πληθώρα γνώσεων. Γνώσεις που στις μέρες μας θεωρούνται μη απαραίτητες ή σχεδόν μη χρήσιμες για έναν αρχιτέκτονα. Ένα χαρακτηριστικό παράδειγμα είναι η αστρονομία. (sollertiae p.5)

% Ο Βιτρούβιος τοποθετεί την αρχιτεκτονική στην εγκύκλιο παιδεία. (Status, Pay, Pleasure) (Λέφας σελ45 11-12). Η αρχιτεκτονική καταλαμβάνει την υψηλότερη θέση στον τομέα της κατασκευής, πράγμα που επιτυγχάνεται από τη νεαρή ενασχόληση και μελέτη πάνω στις τέχνες και τα γράμματα.

% Στο Βιβλίο 1. 1 παράγραφος 2, ο Βιτρούβιος αναφέρει πως άτομα τα οποία βασίστηκαν αποκλειστικά πάνω στη πρακτική εξάσκηση ή  στα γράμματα και τη θεωρία δεν κατάφεραν να επιτύχουν και να προσφέρουν ένα σωστό κτίσμα. Αντιθέτως, άτομα τα οποία ήταν οχυρωμένα και με πρακτική και θεωρητική θεωρία κατάφεραν να επιτύχουν. *** (Λέγφας σελ37 1.1 παρ 2)

% Αυτό που ξεχωρίζει τον αρχιτέκτονα από ένα τυπικό τεχνίτη είναι η υπεροχή στα γράμματα. Φυσικά η ιδιότητα του τεχνίτη είναι εξίσου σημαντική. ένας αρχιτέκτονας πρέπει να κατέχει και τις δύο κατηγορίες. Συνθετική υπεροχή (Status, Pay, Pleasure p. 9)

% Ο Βιτρούβιος απαιτεί από τον αρχιτέκτονα να κατέχει μία πληθώρα γνώσεων, από γεωμετρία και φιλοσοφία μέχρι και ιατρική. Τονίζει βέβαια ότι δεν απαιτείται η υπεροχή στους τομείς αυτούς αλλά μία τουλάχιστον ενασχόληση και εξοικείωση. Υποστηρίζει πως όλοι οι κλάδοι της γνώσης συνδέονται και έχουν κάτι κοινό μεταξύ τους. (Λέφας σελ45 11-12)Φυσικά μία τέτοια απαίτηση σε γνώσεις απαιτεί παράλληλα μελέτη εκτενούς διάρκειας. (Status, Pay, Pleasure p.7)

% Όλες οι τέχνες και επιστήμες έχουν μία κοινή σχέση μεταξύ τους, δηλαδή συνδέονται. Ο Βιτρούβιος χρησιμοποιεί αυτή την πρόταση σαν επιχείρημα για το λόγω που ο αρχιτέκτονας πρέπει να κατέχει μία πληθώρα γνώσεων. Γνώσεις που στις μέρες μας θεωρούνται μη απαραίτητες ή σχεδόν μη χρήσιμες για έναν αρχιτέκτονα. Ένα χαρακτηριστικό παράδειγμα είναι η αστρονομία. (sollertiae p.5)

% Ο Βιτρούβιος τοποθετεί την αρχιτεκτονική στην εγκύκλιο παιδεία. (Status, Pay, Pleasure) (Λέφας σελ45 11-12). Η αρχιτεκτονική καταλαμβάνει την υψηλότερη θέση στον τομέα της κατασκευής, πράγμα που επιτυγχάνεται από τη νεαρή ενασχόληση και μελέτη πάνω στις τέχνες και τα γράμματα.





% \subsection{Βιτρούβιος}

% Στην αρχή του έργου του, περιγράφει τη διαφορά μεταξύ της πρακτικής μεριάς τηςαρχιτεκτονικής και την θεωρητική (fabtica και ratiocinatio αντίστοιχα).

% Η γνώση του αρχιτέκτονα πηγάζει τόσο από την μελέτη και θεωρία (ratiocinatio)αλλά και από την προσωπική ενασχόληση με τα πράγματα δηλαδή την πράξη, τέχνη(fabrica). Δηλαδή ο αρχιτέκτονας δεν μελετάει απλά αλλά συγχρόνος συνθέτει καικατασκευάζει. (sollertiae p.4)

% Δυστοιχώς η γλώσσα του Βιτρούβιου είναι αρκετά περίπλοκη. Μελετητές καιμεταφραστές αδυνατούν να καταλήξουν σε ένα κοινό συμπέρασμα. *** (παρ. 6)

% Η γλώσσα του Βιτρούβιου αφείνει ορισμένα κενά δυσκολεύοντας έτσι την ερμεινίατων προτάσεων του. Διάφοροι μεταφραστές έχουν αποδώσει ορισμένα τμήματα τουέργου με διαφορετικό τρόπο. Έχουν δώσει διαφορετικές ερμεινίες. (sollertioa p.1)

% \subsection{Fabrica}

% Από τους περισώτερους μεταφραστές υποστηρίζεται πως η λέξη fabrica μεταφράζεταιως πρακτική, πράκτις. Με την ένοια δηλαδή της επαναλαμβανόμενης άσκησης τουχεριού. Η λέξη έτσι όπως χρησημοποιείται από τον Βιτρούβιο δεν σημαίνει πρακτικόκτίσμα ή την τέχνη της κατασκευής. (παρ.7)

% Κατά τον μεταφραστή Joseph Gwilt η λέξη fabrica σημαίνει συχνή και συνεχήπερισυλλογή (meditatio) των τεχνών σχετικά με τη διαδικασία του χτισήματος. Ηδιαδικασία αυτή είανι πνευματική και λετσι η fabrica μέσω της διαδικασίαςmeditatio αποδίδει την απιτούμενη επαγγελματική γνώση και εμπειρία. Στην ουσία ηfabrica δεν αφορά αποκλειστικά χειρονακτική γνώση, αν και επιτρέπεται, καιαποκτάται με την άμεση ενασχόληση στο κτίσιμο και ανάλογες τέχνες. (παρ 10)

% Η τέχνη κατά τον Βιτρούβιο είναι η εξάσκηση της αρχιτεκτονικής ενατένισης. Τορπατκό κομμάτο του "ξέρω πως". ¨εχει να κάνει με την ανάλυση του χώρου,χωροταξία κλπ. Περιλαμβάνει την όλη επαγκελματική γνώση του αρχιτέκτονα. (παρ11)











  
  
  
  
%  Ο {\color{red}\textbf{βιτρούβιος}} στο \emph{κείμενό} του \textit{αναφέρει} τις 6 αρχές της αρχιτεκτονικής. Τάξη, Διάθεση, ευρυθμία, συμετρία, κοσμιότητα και οικονομια. \cite{scranton_vitruvius_1974, vitruvius-lefas}
  
%\begin{enumerate}[noitemsep] %αλλίως itemize
  %\item ένα
  %\item Δύο
  
 % \begin{itemize}
   % \item ένα.1
    %\item ένα.2
  %\end{itemize}
  
  %\item Τρία
  
  %    \begin{description}
     %   \item[όρος 1] μπλαμπλαμπλά ηςθεφη.
    %    \item[όρος 2] ηςθεφη.
      %\end{description}
  
    %\item Τέσσερα
  %\end{enumerate}