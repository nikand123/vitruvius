\section{Τα Ποιοτικά Χαρακτηριστικά του Αρχιτέκτονα}

Από την εισαγωγή ήδη του βιβλίου, ο Βιτρούβιος αφήνει να φανεί η σημασία που αποδίδει στο θέμα της γνώσης καθώς και στη μορφή της (θεωρητική, πρακτική). Ξεκινώντας ήδη ο Βιτρούβιος κάνει σαφές ότι δεν μπορούμε να μιλάμε για αρχιτέκτονα και αρχιτεκτονικό έργο άξιο να μελετηθεί και να ασχοληθεί κάποιος με αυτό, εφόσον δεν στηρίζεται σε κάποιο επαρκές γνωσιακό υπόβαθρο. Κάνει μία εκτενή αναφορά στα διαφορετικά είδη γνώσης στα οποία θα πρέπει να επενδύσει ο αρχιτέκτονας καθώς και στην πρακτική εξάσκησή τους.

% Στο έργο του περιγράφει αναλυτικά το τεχνικό κομμάτι της αρχιτεκτονικής. Κατασκευή και περιγραφή για το πως πρέπει να κατασκευάζονται οι ναοί, δημόσιος χώρος, υλικά κλπ. Πέρα όμως από το τεχνικό χαρακτήρα και τη λειτουργία του ως εγχειρίδιο για τους αρχιτέκτονες και άλλους ενδιαφερόμενους, το \emph{Περί Αρχιτεκτονικής} κατέχει και έναν πιο θεωριτικό ρόλο στο κλάδο της επιστήμης της αρχιτεκτονικής. Ίσως, το πιο σημαντικό κομμάτι του έργου του Βιτορύβιου είναι οι θεωρίες του και απόψεις του πάνω στην αρχιτεκτονική καθ' αυτού. 

\subsection{Η γνώση του αρχιτέκτονα}

Για τον Βιτρούβιο, ένας αρχιτέκτονας οφείλει να κατέχει ένα ευρύ πεδίο γνώσεων 
ώστε το αποτέλεσμα της εργασίας του να είναι αξιομνημόνευτο. Μάλιστα αφιερώνει 
σχεδόν ολόκληρη την ενότητα πάνω στο συγκεκριμένο θέμα, πράγμα που φανερώνει 
πως είναι ιδιαίτερα σημαντικό και άξιο προς συζήτησης για τον ίδιο.

Ο αρχιτέκτονας οφείλει να είναι εφοδιασμένος με μία ποικιλία γνώσεων, που 
πρέπει να εκτείνονται από τη γεωμετρία (με την σημερινή ερμηνεία της λέξης) 
μέχρι και ιστορία, φιλοσοφία, μουσική, αλλά και ιατρική, νομική και αστρονομία 
\cite[σ. 392]{masterson_status_2004}. Αξιολογώντας τη διαχρονική αξία αυτής της 
θέσης\sidenote%
    {Σήμερα πιθανώς να φαίνεται παράδοξο, ίσως και ασυμβίβαστο η   κατοχή 
    γνώσεων μουσικής, ιατρικής κ.λπ. παράλληλα με τις καθαρά τεχνικές γνώσεις 
    τη αρχιτεκτονικής, καθώς μπορεί κάποιος να αναρωτηθεί για παράδειγμα, σε τι 
    μπορεί να συμβάλλει η εξοικείωση με
    την επιστήμη της αστρονομίας πάνω στη σχεδίαση μίας
    τυπικής κατοικίας.},
επικαλούμαστε ξανά την επιχειρηματολογία του Βιτρούβιου: με την αστρονομία, ο 
αρχιτέκτονας μαθαίνει να προσανατολίζεται, γνωρίζει που βρίσκεται ο Βοράς, ο 
Νότος, η Δύση και η Ανατολή αλλά και αντιλαμβάνεται τις ιδιότητες των σημείων 
αυτών του ορίζοντα και την επίδραση που έχει η σωστή τοποθέτηση και φορά ενός 
κτηρίου σε σχέση με αυτά \cite[σ. 43]{vitruvius-lefas}. Γενικότερα υποστηρίζει 
πως όλες οι επιστήμες και οι κλάδοι τους έχουν μία κοινή σχέση μεταξύ τους την 
οποία αποδίδει με τον αρχαιοελληνικό όρο της \emph{εγκυκλίου παιδείας} 
(encyclios disciplina). Ο όρος έχει να κάνει ακριβώς με αυτή την απαραίτητη 
πολυμαθεία του αρχιτέκτονα. Όπως λέει, μόνο η επιστήμη της αρχιτεκτονικής 
περιβάλλεται από τόσες άλλες επιστήμες, καθιστώντας έτσι τον αρχιτέκτονα ικανό 
να φτάσει στο ανώτερο επίπεδο της τέχνης του που είναι το \emph{κατασκευάζειν} 
\cite[σ. 45]{vitruvius-lefas}.

Φυσικά ο Βιτρούβιος δεν αναμένει από κανέναν να αριστεύει σε όλους τους τομείς 
των επιστημών, καθώς πιθανώς είναι πρακτικά αδύνατο. Δεν είναι δυνατόν ούτε 
μπορεί ένας αρχιτέκτονας, όπως αναφέρει, να είναι φιλόσοφος όπως ο Αρίσταρχος, 
μουσικός όπως ο Αριστόξενος ή ιατρός όπως ο Ιπποκράτης, καθώς κανένας άνθρωπος 
δεν μπορεί να είναι αυθεντία σε όλες τις τέχνες και επιστήμες, να κατέχει 
δηλαδή σε βάθος τη θεωρία τους. Ο Βιτρούβιος αντίθετα, ζητάει από τον 
αρχιτέκτονα μία τουλάχιστον εξοικείωση σε αυτούς τους τομείς της τέχνης και των 
επιστημών \cite[σ.~45]{vitruvius-lefas}. 

\subsection{Θεωρία και πράξη}

Το χαρακτηριστικό που επιτρέπει στον  ιδανικό αρχιτέκτονα να ξεχωρίσει από έναν κοινό τεχνίτη είναι η υπεροχή του στα γράμματα \cite{masterson_status_2004}. Φυσικά το επάγγελμα του τεχνίτη είναι εξίσου σημαντικό αλλά για τον Βιτρούβιο, η διαφορά είναι πως ο αρχιτέκτονας δεν βασίζεται αποκλειστικά στο τεχνικό, πρακτικό κομμάτι της κατασκευής αλλά στηρίζεται και πάνω σε ένα θεωρητικό υπόβαθρο. Είναι χαρακτηριστική η αναφορά του Βιτρούβιου:

\begin{quote}
{\itshape "Οι αρχιτέκτονες, λοιπόν, που χωρίς γράμματα αρκέστηκαν στην πρακτική 
εξάσκηση,  δεν κατόρθωσαν να δημιουργήσουν [έργα] με κύρος ανάλογο του μόχθου 
που κατέβαλαν. Αυτοί που εμπιστεύθηκαν αποκλειστικά την θεωρία και τα γράμματα 
κυνήγησαν, όπως φαίνεται, όχι τα ίδια τα πράγματα, αλλά τη σκιά τους. Όσοι όμως 
κατείχαν και τα δύο επέτυχαν..."}
\begin{flushright}
\footnotesize{--- Παύλος Λέφας, Vitruvii De architectura (σ. 37)}
\end{flushright}
\end{quote}

Χρησιμοποιεί μάλιστα ειδικούς όρους για να περιγράψει τις δύο αυτές 
διαφορετικές μορφές γνώσης: \emph{Ratiocinatio} (Θεωρία) και \emph{Fabrica} 
(Πράξη), τις οποίες θα αναλύσουμε στη συνέχεια. Θα πρέπει όμως πρώτα να 
διευκρινίσουμε τα εξής:

\begin{itemize}
\item Οι συγκεκριμένοι όροι, όπως και πολλοί άλλοι που χρησιμοποιεί, δεν έχουν 
απαραίτητα την ίδια έννοια που έχουν στη σύγχρονη γλώσσα. Σαφώς, έχοντας υπόψη 
μας την περίοδο που έζησε ο Βιτρούβιος αλλά και τις επιρροές του από τον 
αρχαιοελληνικό πολιτισμό, οι μελετητές μπορούν εώς ένα σημείο μόνο να 
μεταφράσουν και να αναλύσουν άμεσα την ορολογία του Βιτρούβιου, ενώ σε πολλές 
περιπτώσεις \sidenote%
    {Όπως ήδη αναφέρθηκε, η γλώσσα του Βιτρούβιου είναι σχετικά περίπλοκη ίσως και δυσνόητη, με αποτέλεσμα οι μεταφράσεις να έχουν δημιουργήσει μία σχετική σύγχυση για την πραγματική ουσία του κειμένου του συγγραφέα. Οι όροι και οι έννοιες που χρησιμοποιεί και η τοποθέτησή τους στο κείμενο, είναι αντικείμενο έντονης συζήτησης μεταξύ των μελετητών ως προς την εννοιολογική τους ερμηνεία.}
χρειάζεται να βασίσουν την ερμηνεία και τις αναλύσεις τους, ανατρέχοντας στις 
γλωσσικές ρίζες των όρων που χρησιμοποιεί. Με αυτό τον τρόπο επιδιώκεται η 
αποκρυπτογράφηση του βαθύτερου νοήματος που προσπαθεί να μεταφέρει ο Βιτρούβιος 
\cite{graham-education}.
\item Η άποψη του Βιτρούβιου περί θεωρίας και πράξης και η σύνδεσή τους με την 
ιδιότητα του αρχιτέκτονα, φαίνεται να είναι πολύ κοντά στη σχετική Πλατωνική 
ερμηνεία. Για τους αρχαίους Έλληνες δεν υπήρχε ιδιαίτερος ορισμός για την 
ιδιότητα του αρχιτέκτονα, που να τον ξεχωρίζει από αυτή του τεχνίτη. Για τον 
Πλάτωνα λοιπόν, ο αρχιτέκτονας δεν είναι ένας τυπικός τεχνίτης αλλά ένα ορθά 
μελετημένο άτομο το οποίο κατέχει το πρακτικό κομμάτι του χτισίματος αλλά και 
γνώσεις πάνω στις επιστήμες, δηλαδή το θεωρητικό υπόβαθρο. Η \emph{πρακτική} 
και \emph{η γνωστική} του Πλάτωνα συνθέτουν την ίδια την επιστήμη. 
\cite{graham-education}
\end{itemize}

\begin{description}[style=nextline]
\item[Fabrica]

Η λέξη Fabrica ερμηνεύεται γενικά ως \emph{τέχνη} αλλά με τη γενική έννοια της 
\emph{Πράξης} \cite{vitruvius-lefas,graham-education}. Η έννοια δηλαδή της 
επαναλαμβανόμενης άσκησης του χεριού ή η ενασχόληση με τα πράγματα. Στην ουσία 
είναι η αρχιτεκτονική ενατένιση, το πρακτικό κομμάτι του "ξέρω πως", δηλαδή 
έχει να κάνει με την ανάλυση του χώρου, τη χωροταξία, κλπ. Με άλλα λόγια 
περιλαμβάνει την επαγγελματική γνώση του αρχιτέκτονα \cite{graham-education}. Ο 
Βιτρούβιος όμως, περιγράφοντας τη έννοια της fabrica, της δίνει και μία 
ιδιαίτερη σημασία, αυτό της πνευματικής ενασχόλησης. Όντως την συνδέει με τη 
λέξη \emph{meditatio}, η οποία ερμηνεύεται ως μία διαδικασία μελέτης, 
περισυλλογής ή σκέψης. Οπότε μέσω της διαδικασίας του meditatio αποκτάται η 
απαιτούμενη επαγγελματική γνώση και εμπειρία του αρχιτέκτονα. Αυτή η προσέγγιση 
είναι πολύ κοντά στη σχετική άποψη του Πλάτωνα\sidenote%
{Τόσο ο Πλάτωνας όσο και ο Βιτρούβιος, ερμηνεύουν τη πνευματική συνιστώσα της 
πρακτικής γνώσης μέσα από το παράδειγμα της μουσικής. Η μουσική σαν πράξη 
αποτελείται τόσο από το πρακτικό κομμάτι δηλαδή τη διαδικασία παραγωγής του 
ήχου μέσα από την η ενασχόληση με το μουσικό όργανο, όσο και την θεωρητική 
σύλληψη της μουσικής σύνθεσης, που είναι ένα καθαρά θεωρητικό κομμάτι 
σχετιζόμενο με τα μαθηματικά, την αρμονία και τη μεταφυσική.}
, ο οποίος αν και περιγράφει την πράξη σαν ένα είδος πρακτικής γνώσης η οποία είναι περισσότερο επιτακτική ή επαγγελματική παρά αποκλειστικά προϊόν πνευματικής νόησης (φιλοσοφική, επιστημονική ή μαθηματική), εντούτοις η \emph{πρακτική} και \emph{γνωστική} του αρχιτέκτονα είναι και οι δύο είδη πνευματικής γνώσης \cite{graham-education}.  


\item[Ratiocinatio]

Η λέξη ratiocinatio ερμηνεύεται ως \emph{Θεωρία}, με την έννοια του υπολογισμού, του σχεδιασμού. Σε αντίθεση με τον όρο fabrica, που αν και έχει μία πνευματική συνιστώσα (meditaio), υπονοεί κατά βάση τη χειρονακτική ενασχόληση με τα πράγματα , το ratiοcinatio αποτελεί μία καθαρά πνευματική διαδικασία \cite{vitruvius-lefas}.

Το ratiocinatio έχει ως βάση τη λέξη \emph{ratio} που μεταφράζεται ως 
\emph{λόγος}. Στην αρχαιοελληνική γραμματική η συγκεκριμένη λέξη έχει 
περισσότερες από μία έννοιες, όπως θεωρία, περιγραφή ή λόγος με την μαθηματική 
έννοια δηλαδή η σχέση μεταξύ δύο αριθμών. Ιδιαίτερα μάλιστα η τελευταία αυτή 
ερμηνεία του λόγου είναι κοινή με το λατινικό \emph{ratio} και το Σωκρατικό 
παράγωγό τους, της \emph{λογικής} (rational). Το Ratiocinatio είναι η κατανόηση 
της δημιουργίας των πραγμάτων με γνώση και δεξιότητες: πως μπορεί να 
παρουσιασθεί αλλά και να ερμηνευτεί η ορθολογική αναλογία 
\cite{patterson-1997}.  Βέβαια η σύνδεση αυτή της θεωρίας με τη λογική έχει 
προκαλέσει πολλά προβλήματα ερμηνείας καθώς πολλοί μεταφραστές αδυνατούν να 
καταλήξουν σε ένα κοινό συμπέρασμα για την έννοια της λέξης ratiocinatio, 
αποδίδοντας διαφορετικές ερμηνείες. Είναι χαρακτηριστικό ότι η φράση:

\begin{quote}
\itshape
"Ratiocinatio autem est quae resfabricates sollertiae acrationis pro portione (proportione) demonstrare atque explicare potest"
\end{quote}

δυσκολεύει ακόμα τους μελετητές \cite{graham-education}. Επικρατέστερη λοιπόν φαίνεται να είναι η ερμηνεία του όρου ως μαθηματική αναλογία κάτι που έμμεσα υποστηρίζεται και από τα έργα μεταγενέστερων καλλιτεχνών, εμπνευσμένων από το  Βιτρούβιο και ειδικά το DA. Κλασσικό παράδειγμα είναι το γνωστό έργο του \href{https://en.wikipedia.org/wiki/Leonardo_da_Vinci}{Leonardo da Vinci}, \href{https://en.wikipedia.org/wiki/Vitruvian_Man}{\emph{"Vitruvian Man"}} (Σχ. \ref{fig:vitruvian-man}) στο οποίο απεικονίζονται οι ιδανικές ανθρώπινες αναλογίες. Ο da Vinci δηλώνει ρητά ότι το συγκεκριμένο σχέδιο είναι επηρεασμένο από τη θεωρία του Βιτρούβιου για τις αναλογίες του ανθρώπινου σώματος (βιβλίο ΙΙΙ).
\end{description}


\begin{marginfigure}%
  \shadowimage[width=\linewidth]{vitruvian-man}
  \caption{\footnotesize Ο  Άνθρωπος του Βιτρούβιου (Vitruvian Man) σχεδιασμένος
  από το Λεονάρντο ντα Βίντσι. Ένα χαρακτηριστικό έργο εφαρμογής των αρχών που 
  περιέγραψε ο Vitruvius, σχεδιασμένο από ένα από τους μεγαλύτερους ζωγράφους 
  όλων των εποχών
  (πηγή: \cite{wikipedia:vitruvianman}).}
  \label{fig:vitruvian-man}
\end{marginfigure}