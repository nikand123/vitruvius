\section{Αρχές της Αρχιτεκτονικής}

Συνεχίζοντας στο πρώτο βιβλίο και συγκεκριμένα στη δεύτερη ενότητα, ο Βιτρούβιος εισάγει έναν \emph{κανόνα} για τις αρχές της αρχιτεκτονικής(ή συνιστώσες \cite{vitruvius-lefas}). Πρόκειται για ένα σύνολο από "χαρακτηριστικά" ή "ιδιότητες" που τις ονομάζει:

\begin{itemize}[noitemsep]
\item Τάξις (Ordinatio)
\item Διάθεσης (Dispositio)
\item Ευρυθμία (Eurythmy)
\item Συμμετρία (Symmetria) 
\item Κοσμιότητα (Decor) 
\item Οικονομία (Distributio)
\end{itemize}

Φαίνεται πως η σειρά με την οποία αναλύει τις έξι αυτές συνιστώσες γίνεται με 
βάση το βαθμό "τεχνικότητας, όπου οι Τάξις, Διάθεσης και Ευρυθμία, αναφέρονται 
στη θεωρητική σύλληψη/πλαίσιο του έργου, ενώ οι Συμμετρία, Κοσμιότητα και 
Οικονομία έχουν περισσότερο τεχνικό χαρακτήρα \cite{vitruvius-lefas}. 
Παρατηρούμε δηλαδή ότι ο Βιτρούβιος κάνει σαφή διαχωρισμό μεταξύ ποιοτικών και 
τεχνικών χαρακτηριστικών χωρίς να εξαιρεί κανένα και χωρίς να υποτιμά κανένα. 
Οι αρχές του αφορούν τόσο το αισθητικό κομμάτι της επιστήμης (δηλαδή το ίδιο το 
έργο τέχνης, το κτήριο), όσο και την ίδια την τέχνη του αρχιτέκτονα (την πράξη, 
τις ενέργειες που εκτελεί και την ενασχόληση με το έργο του). Πιο 
αναλυτικά\cite{vitruvius-lefas,lefas-fundamental}: 
 
\begin{description}[style=nextline]
\item[Ordinatio]
  
Ο όρος \emph{τάξις} έχει την έννοια της ιεραρχημένης σύνθεσης δηλαδή τη ορθή αλλά και με κάποιας μορφής διαβάθμισης, τοποθέτηση των πραγμάτων, βάση ενός κοινού μέτρου. Ο Βιτρούβιος συνδέει τη συγκεκριμένη λέξη και με κάποιος επιπλέον τους όρους:

\begin{itemize}[noitemsep]
\item \emph{commoditas} και \emph{modica} οι οποίες έχουν ως βάση το μέτρο 
(modus), σημαίνουν "σύμμετρο" ή "με μέτρο", υπονοώντας πως η λέξη τάξις 
συνδέεται με την συνιστώσα της συμμετρίας. Οι δύο νέοι αυτοί όροι ίσως 
χρησιμοποιούνται με την Ευκλείδεια μαθηματική έννοια, δηλαδή τη σχέση μεταξύ 
μεγεθών, καθώς σε ένα κτίσμα για παράδειγμα, τα μέλη του (ή τα μέρη των 
επιμέρους μελών) πρέπει να έχουν μία "σωστή" θέση μεταξύ τους (σχέση μεγεθών), 
βάση ενός κοινού μέτρου.

%\cite[σ.~186]{}

\item \emph{comparatio}. Η λέξη αυτή έχει διπλή σημασία. Σημαίνει κατασκευή, 
σύνθεση και παράλληλα σύγκριση. Στην ουσία σε αυτό που θέλει να καταλήξει ο 
ίδιος είναι πως τα μεγέθη πρέπει να έχουν μία ιεράρχηση, μία σχέση μεταξύ τους 
η οποία να είναι μεν φανερή (συγκρίσιμη) αλλά να μην επικαλύπτει το ένα το 
άλλο. Παρατηρούμε δηλαδή ότι καταλήγει και πάλι στην έννοια της ισορροπίας των 
μελών ενός συνόλου. Βέβαια αυτή η ιεράρχηση βασίζεται στο "μέγεθος" του κάθε 
μέλους και δεν έχει να κάνει με την αριθμητική τους σχέση. 
\end{itemize}

%\cite[σ.~93,188]{vitruvius-lefas,lefas-fundamental}

Η τάξις επιτυγχάνεται μέσω της ποσότητας, η οποία έχει να κάνει με τον ορισμό ενός κοινού μέτρου. Μέσω αυτού του κοινού μέτρου αντιλαμβανόμαστε τα μέλη του κτίσματος, έχοντας έτσι ένα αρμονικό σύνολο. 

\item[Dispositio]
  
Η διάθεσης στην ουσία σημαίνει "διάταξη" και έχει ως έννοια την διαδικασία της 
ορθής τοποθέτησης των πραγμάτων στο χώρο, της \emph{διευθέτησης} (Granger και 
Morgan στο \cite[σ.~495]{scranton_vitruvius_1974}). Συνδέει επίσης το 
συγκεκριμένο με τον όρο \emph{conlocatio}\sidenote%
    {Τόσο η λέξη dispositio και conlocatio 
    έχουν κατάληξη -tio, -tionis, που δηλώνει
    πρωτίστως την εκτέλεση μίας διαδικασίας και
    έπειτα το αποτέλεσμα μίας διαδικασίας, πράγμα
    που σημαίνει πως υπάρχει μέσα στην λέξη
    διάθεσης, η έννοια της ενέργειας, πράξης
    με καθορισμένη τάξη που αφορά σε έναν σκοπό \cite[σ.~496]{scranton_vitruvius_1974}.}
, που σημαίνει "τοποθέτηση των πραγμάτων". 

Ο Βιτρούβιος αναφέρει πως η Διάθεσης εμφανίζεται στις αρχαιοελληνικές λέξεις \emph{ιχνογραφία}, \emph{ορθογραφία} και \emph{σκηνογραφία}, που σημαίνουν αντίστοιχα κάτοψη, όψη και αναπαραστατικό σχέδιο. Τονίζει πως οι έννοιες αυτές είναι ένα μέσο προς επίτευξη της Διάθεσης, δηλαδή συνδέονται με τη διαδικασία του σχεδιασμού και δεν είναι τα ίδια τα σχέδια. 
  
\item[Eurytmia]
  
Παρατηρούμε ότι χρησιμοποιεί αυτούσιο τον αρχαιοελληνικό όρο και όχι κάποιον 
λατινικό και μάλιστα με την πρωταρχική του σημασία, \emph{ευ-ρυθμός} δηλαδή 
ποιοτικός, καλός ρυθμός στην αρχιτεκτονική, ο οποίος πρέπει αν είναι ευχάριστος 
στις αισθήσεις. Είναι η όμορφη όψη και η ισορροπημένη εμφάνιση των μελών του 
συνόλου. Αναφέρει πως επιτυγχάνεται με την ανταπόκριση των μεγεθών του έργου, 
δηλαδή τις αναλογίες του ύψους, μήκους και πλάτους, με την συμμετρία (πάλι ίσως 
με την πιο γενική της έννοια, δηλαδή ισόρροπη) 
\cite[σ.~498]{scranton_vitruvius_1974}. 

\item[Symetria]

Ίσως η θεμελιώδης αρχή της θεωρίας του πάνω στις συνιστώσες της αρχιτεκτονικής. 
Δεν έχει την κοινή και γνωστή έννοια της Ευκλείδειας συμμετρίας αλλά εκφράζει 
την συμφωνία και την εναρμόνιση των υπομέρους τμημάτων ενός συνόλου σε σχέση με 
το ίδιο. Όπως η τάξις και η ευρυθμία έτσι και η συμμετρία στηρίζεται πάνω στην 
έννοια του μέτρου (Πλατωνική αντίληψη της συμμετρίας) και συγχρόνως είναι 
αποτέλεσμα αναλογικών και αριθμητικών σχέσεων. Συγκεκριμένα, έχει τις βάσεις 
της στις αναλογίες του ανθρώπινου σώματος του οποίου τα μέλη εκ φύσεως έχουν 
μία αναλογική σχέση μεταξύ τους. Όπως αναφέρει, ο πήχης, η παλάμη, το δάκτυλο, 
ολόκληρο το χέρι, καθιστούν εύρυθμο το ανθρώπινο σώμα, έτσι και σε ένα κτίσμα 
τα μέλη και τα μεγέθη του πρέπει να έχουν μία αναλογία μεταξύ τους ώστε να 
επιτυγχάνεται η εναρμόνιση τους με το σύνολο.

Η συμμετρία έχει έναν πιο "τεχνικό" χαρακτήρα. Συνδέεται άμεσα με την τάξις και 
τη ευρυθμία καθώς αποτελεί απαραίτητη προϋπόθεση της ύπαρξης τους. Συγχρόνως 
όμως είναι ανεξάρτητη από αυτές. Όπως ακριβώς συμμετρία (δηλαδή οι αριθμητικές 
αναλογίες) παρατηρείται στο ανθρώπινο σώμα, το ίδιο θα πρέπει να συμβαίνει και 
στα κτήρια κατά την διαδικασία του σχεδιασμού και του χτισίματος. Εδώ ο 
Βιτρούβιος προκειμένου να εξηγήσει την έννοια της συμμετρίας, παρουσιάζει το 
παράδειγμα του ναού, του οποίου τα μέρη, όπως ο κίονας και η τρίγλυφος, είναι 
εναρμονισμένα μεταξύ τους. Παρόμοια παραδείγματα επικαλείται και σε άλλα σημεία 
του DA, όπως αυτό της \emph{Δωρικής θύρας} (βιβλίο VI).

\item[Decor]
  
Η κοσμιότητα έχει να κάνει με την "αισθητική" αλλά με την έννοια του 
"καθωσπρέπει" και της "αρμόζουσας" σχέσης του κτίσματος σε σχέση με 
κοινωνικούς, ιστορικούς ή φυσικούς παράγοντες. Μιλώντας για κοσμιότητα ο 
Βιτρούβιος επιχειρεί να προσδώσει έναν χαρακτηρισμό "καταλληλότητας" στο κτίσμα 
έχοντας υπόψη του τη θεματολογία του. Παρουσιάζει δε ορισμένα παραδείγματα 
εμφάνισης του στοιχείου της κοσμιότητας όπως την επιλογή του αρμόζοντα ρυθμού 
(κορινθιακού, δωρικού κ.λπ.) στους ναούς ή όταν ένα κτήριο επωφελείται άψογα 
από τον προσανατολισμό του \cite{scranton_vitruvius_1974}. 

Γιά τον Βιτρούβιο η κοσμιότητας θα πρέπει να συνδυάζει τα εξής τρία στοιχεία:

\begin{enumerate}[noitemsep]
  \item \emph{Θεματισμό}, όταν το έργο ανταποκρίνεται στο στο "θέμα" του.
  \item \emph{Έθει}, κοσμιότητα που έχει να κάνει με την τήρηση της παράδοσης.
  \item \emph{Φύσει}, έχει να κάνει με την ένταξη του έργου στη φύση.
\end{enumerate}

Αν θέλουμε να απλοποιήσουμε την συγκεκριμένη έννοια υποθέτουμε ότι ο Βιτρούβιος υπονοεί αυτό που σήμερα αντιλαμβανόμαστε σαν "εντός του θέματος".
  
\item[Distributio]

Με το συγκεκριμένο όρο\sidenote%
    {Και αυτή η λέξη έχει κατάληξη -tio, -tionis, όπως και η dispositio, δηλώνοντας και αυτή μία ενέργεια, μία διαδικασία.} 
, ο Βιτρούβιος εννοεί τη συνολική προσέγγιση του αρχιτέκτονα, στο έργο. Θα 
πρέπει να προγραμματίσει κατάλληλα τόσο τη σχεδίαση όσο και την κατασκευή του, 
τηρώντας ταυτόχρονα ένα λογικό προϋπολογισμό κόστους. Στην προσπάθεια αυτή θα 
πρέπει να έχει υπόψη του την τοποθεσία του έργου καθώς και την οικονομική άνεση 
και το κύρος πελάτη. Επίσης η διαδικασία αυτή θα πρέπει να τηρείται όταν οι 
ανάγκες του έργου βρίσκονται εντός λογικών πλαισίων. 

Η οικονομία παραπέμπει στη έννοια της κοσμιότητας καθώς και στις δύο 
περιπτώσεις ο Βιτρούβιος μιλάει περί ένταξης του έργου εντός ορισμένων 
πλαισίων. Η κοσμιότητα αφορά τη θεματολογία του έργου ενώ η οικονομία έχει να 
κάνει περισσότερο με τον ίδιο τον πελάτη. Είναι χαρακτηριστικό ότι ο Βιτρούβιος 
δίνει ειδική σημασία στις επιθυμίες του πελάτη, όπως φαίνεται και σε άλλα 
σημεία του DA \cite{scranton_vitruvius_1974}.

\end{description}