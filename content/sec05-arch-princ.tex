\section{Αρχές της Αρχιτεκτονικής}

Στη δεύτερη ενώτητα του πρώτου βιβλίου (\emph{βιβλίο 1.2}), ο Βιτρούβιος μιλάει περί των αρχών της αρχιτεκτονικής (συνιστώσες της αρχιτεκτονικής \cite{vitruvius-lefas}), ένα σύνολο από "χαρακτηριστικά", "ιδιότητες" που την συνιστούν και τις ονομάζει: \emph{Τάξις, Διάθεσης, Ευρυθμία, Συμμετρία, Κοσμιότητα και Οικονομία}. Φαίνεται πως η σειρά με την οποία αναλύει τις έξι αυτές συνιστώσες γίνεται με βάση το βαθμό "τεχικότητας", με πρώτες τις Τάξις, διάθεσης και ευρυθμία που αφορούν ένα πιο θεωριτικό πλαίσιο, και έπειτα τις υπόλοιπες που έχουν ένα πιο τεχνηκό χαρακτήρα. \cite[σ.~49,91-92]{vitruvius-lefas}
 
Οι αρχές της αρχιτεκτονικής από τη μία αφορούν το αισθητικό κομμάτι της επιστήμης, δηλαδή το ίδιο το έργο τέχνης, κτήριο, αλλά από την άλλη, με βάση τα λέγην του Βιτρούβιου, αφορούν και την ίδια την τέχνη του αρχιτέκτονα. Εν συντομία, την πράξη του, τις ενέργειες που εκτελεί, την ενασχόληση του με το έργο του. Κατά μία έννοια οι συνιστώσες μπορούν να χωριστούν με βάση: την διαδικασία του αρχιτέκτονα ως Τάξις, Διάθεσης και Οικονομία και την ποιότητα του έργου ως Ευριθμία, Συμμετρία και Κοσμιότητα.


 
\subsection{Τάξις(Ordinatio)}
  
Η λέξη τάξις έχει την έννοια της ιεραρχιμένης σύνθεσης δηλαδή τη ορθή αλλά και με κάποιας μορφής διαβάθμισης, τοποθέτηση των πραγμάτων, βάση ενός κοινού μέτρου. \cite[σ.~92]{vitruvius-lefas} Στη περιγραφή του βιβλίου ο Βιτρούβιος κάνει χρήση των λέξεων \emph{commoditas και modica} οι οποίες έχουν ως βάση το μέτρο (modus). Οι δύο αυτές λέξεις σημαίνουν "σύμμετρο" ή "με μέτρο", υπονοόντας πως η λέξη ταξις συνδέεται με την συνιστώσα της συμμετρίας. Ίωσς ο Βιτρούβιος να χρησιμοποιεί τη λέξη αυτή με την κυριολεκτική, μαθηματική της έννοια βασισμένη στον Ευκλείδη, δηλαδή τη σχέση μεταξύ μεγεθών \cite[σ.~186]{lefas-fundamental}, καθώς σε ένα κτίσμα για παράδειγμα, τα μέλη του (ή τα μέρη των επιμέρους μελών) πρέπει να έχουν μία "σωστή" θέση μεταξύ τους (σχέση μεγεθών) βάση ενός κοινού μέτρου. Μία ισορροπία δηλαδ και να είναι συγκρίσημα.\cite[σ.~92]{vitruvius-lefas}

Μία ακόμη λέξη που χρησιμοποιεί ο Βιτρούβιος είναι η \emph{"comparatio"}. Η λέξη αυτή έχει διπλή σημασία. Σημαίνει κατασκεύη, σύνθεση και παράλληλα σύγκριση. Στην ουσία σε αυτό που θέλει να καταλήξει ο ίδιος είναι πως τα μεγέθη πρέπει να έχουν μία ιεράρχιση, μία σχέση μεταξύ τους η οποία να είναι μεν φανερή (σύγκρίσημη) αλλά δε, να μην επικαλύπτει το ένα το άλλο, καταλήγοντας πάλι στην έννοια της ισορροπίας των μελών ενών συνόλου. Βέβαια αυτή η ιεράρχιση βασίζεται στο "μέγεθος" του κάθε μέλους και δεν έχει να κάνει με την αριθμητική τους σχέση. \cite[σ.~93,188]{vitruvius-lefas,lefas-fundamental}

Η τάξις επιτυγχάνεται μέσω της ποσότητας, η οποία έχει να κάνει με τον ορισμό ενός κοινού μέτρου. Μέσω αυτού του κοινού μέτρου αντιλαμβανόμαστε τα μέλη του κτίσματος, έχοντας έτσι ένα αρμονικό σύνολο. 



% Μία ελαφριά σύγχυση μεταξύ των μεταφραστών. Δεν φαίνεται να έχουν διαφορετικές απόψεις, αλλά δεν μπορούν να καταλήξουν στο συμπέρασμα εάν η τάξις και η συμετρία συνδέονται ή εάν υπάρχει λόγος όπου θεματολογία του Βιτρούβιου είναι έτσι (πρώτα η τάξης και μετά η γεωμετρία. Ενώ η γεωμετρία είνια στην ουσιία το θεμέλιο της σκέψης του).*** \cite[σ.~187]{lefas-fundamental}

% Στην περιγραφή της τάξης, ο Βιτρούβιος κάνει χρήση δυο λέξεων. Commoditas και modica. Και οι δύο λέξειείναι παράγωγες από την ελλνική μέτρο

%Ορισμός από το βιβλίο. 

% Comparatio. Η σημασία της λέξης αυτής είνια διπλή στη περιγραφή της τάξης. Αρχικά σημαίναι σύνθεση, κατασκευή, τοποθέτηση πραγμάτων μαζί. Παράλληλα σημαίνει σύγκριση, αντίθεση μεταξύ πραγμάτων. Για τον Βιτρούβιο η λε΄ξη αυτή έχει ένα στοιχειο της σύγκρισης αλλά και τη σύνθεση ιεραρχίας.

% Ιεραρχία. Κατασκευή όπου εμπεριέχει το στοιχείο της σύγκρισης. Ιεράρχιση των μεγεθών στο σύνολο.

% Η λέξη αυτή ακολουθείται και περιγράφεται με άλλες διάφορες. Είναι έυκολο να υποθέσουμε πως η τάξης βρίσκεται στην υπηρεσία της γεωμετρίας, όχι με τον κυριολετκικό όρο, δηλαδή του ευκλείδη. Ο λέφας φαίνεται να διαφωνεί σε α΄τυο. 
% Υποστηρίζει πως η παρουσία της λέξης γεωμτρία στον ορισμό του Βιτρούβιου για την τάξη έχει την κυριολεκτική σημασία της έννοιας. Η γεωμετρία κατά τον Ευκλείδη λέει πως δύο πράγματα είναι συμετρικά εάν έχουν κοινό μέτρο από ένα σημείο.*** Ο λέφας έπειτα δίνει σαν παράδειγμα την ιεραρχεία της αρχαίς Ρώμης. Με την έννοια οτι ο Καίσαρας συγκριτικά με έναν κατότερο είναι μη συμετρικός,  δηλαδή δεν υπάρχει μία υπερύψοση του πρώτου σε σχέση με το δεύτερο. Ίσως αυτό προσπαθεί να πει ο Βιτρούβιος.
% Σε ένα κτήριο για πραδάδειγμα, όπως συνεχίζει ο Λέφας, ένα ένα τμήμα του υπερισχύει σε σχέση με ένα άλλο τότε το καθιστά μη απαραίτητο, αδιάφορο.
% Οπότε η τάξη δεν είναι αντικείμετνο της γεωμερείας αλλά μέσω της τάξης διατηρέιται η γεωμτρεία*** 

% Ρητορική. Ο λέφας αναφέρει πως Η τάξη στη ρητορική έχει να κάνει με τη σωστή τοποθέτηση αλλά κια το σωστό μέγεθος των μερών του λόγου. Υπάρχουν όπως φαίνεται πολοί που έχουν συγκρίνει τη ρητορική με την αρχιτεκτονική. Οπότε θα μπορούσε να είχαν σχηματιστεί δύο διαφορετικές έννια και για αυτή. Η τάξη και η διάθεση. Η πρώτη έχει να κάνει με την την ποσοτητα και η διάθεση με την χωρική πτυχή.

% Η ποσότητα έχει να κάνει με τον ορισμό ένος κοινού μέτρου και αυτό το κοινό μέτρο έχει να κάνει με τη σειρά του με τα μέλη του συνόλου. Ένα ορθό κατασκευασνένο σύνολο.

% Η αρμονία έχει να κάνει με την επιλογή ενός μέτρου από τα επιμέρους σκέλη και μέρη του συνόλου έχοντας ως βάση την ίδια τη σχέση τον μεγεθών, φτιάχνοντας έτσι ένα ορθό, συμπαγές σύνολο. Η συμετρία έχει να κάνει με την συμαβτότητα των μέρων που φτιάχνουν ένα σύνολο βασισμένη σε ένα κοινό μέτρο. Διαφορά μεταξύ ποσότητας και συμετρίας. και τα δύ καταλήγουν στην αρμονία μέσω διαφορετικών δρόμων.

% Η τάξη σχηματίζεται από την ποσότητα. όπου επιλέγονται τα μέρη του συνόλου καταλλ΄γοντας έτσι σε ένα αρμονικό σύνολο***



\subsection{Διάθεσης (dispositio)}
  
Η διάθεσης στην ουσία σημαίνει "διάταξη" και έχει ως έννοια την διαδικασία της ορθής τοποθέτησηςτων πραγμάτων στο χώρο. \cite[σ.~49-50,93]{vitruvius-lefas} Η λέξη που την ακολουθεί είναι το "conlocatio", που σημαίνει "τοποθέτηση των πραγμάτων". Τόσο η λέξη dispositio και conlocatio έχουν κατάληξη -tio, -tionis, που δηλώνει πρωτίστως την εκτέλεση μίας διαδικασίας και έπειτα το αποτέλεσμα μίας διαδικασίας, πράγμα που σημαίνει πως υπάρχει μέσα στην λέξη διάθεσης, η έννοια της ενέργειας, πράξης με καθορισμένη τάξη που αφορά σε έναν σκοπό. \cite[σ.~496]{scranton_vitruvius_1974}

Ο Βιτρούβιος αναφέρει πως η διάθεησς εμφανίζεται στην "ιχνογραφία". στην "ορθογραφία" και στην "σκηνογραφία", Ελληνικές λέξεις που σημαίνουν αντίστοιχα κάτοψη, όψη και αναπαραστατικό σχέδιο \cite[σ.~50-51,94]{vitruvius-lefas} Τονίζει πως οι έννοιες αυτές είναι ένα μέσο προς επίτευξη της διάθεσης, δηλαδή συνδέονται (είναι) με τη διαδικασία του σχεδιασμού και δεν είναι τα ίδια τα σχέδια. 
  
  
  
% Ο Βιτρούβιος χρησημοποιεί ωε συνώνυμα τις λέξεις conlocatio και effectus. 
% Τόσο η λέξη dispositio και η conlocatio αλλά και ολα τα ουσιαστικά με   κατάληξη -tio, -tionis, έχουν ως κύρια ερμηνεία την έννοια της εκτέλεσης μίας   διαδικασίας και σαν δεύτερη ερμηνεία την έννοια του αποτελέσματος μίας   διαδικασίας. 
%  Η έννοια της διάθεσης είναι "τοποθέτηση μαζί" και το   "πραγματοποιείται" ή το "λειτουργεί", όχι όμως το "αποτέλεσμα", "η   προκύπρουσα σύνθεση μορφή". (Vitruvious arts of architecture p.3)
  
%  Τόσο ο Granger όσο και ο Morgan μεταφράζουν τη λέξη dispositio ως Arragement, δηλαδή διευθέτηση. Στην ουσία η έννοια της λέξης δεν είναι η τάξη, σειρά δηλαδή η ποιότητα ενός έργου αλλά η διαδικασία της ορθής τοποθέτησης των πργμάτων. Ο υπολογισμός των μέτρων του ίδιου του κρίσματος αλλά και των επιμέων στοιχείων του.
  
% Ο Βιτρούβιος κάνει χρηση των λέξεων αυτών. Ελληνικές λέξεις για τις οποίες ο ίδιος δε χρησιμοποιεί κάποια λατινική. Οι λέξεις αυτές ερμηνεύονται ώς κάτοψη, όψη και ισως προπτική αντίστοιχα. 
% Ο βιτρούβιος βέβαια τις χρησημοποιεί με τον δικό του τρόπο. Η ιχνογραφεία και η τοπογραφεία ορίζονται σαν διαδικασίες, Κάνουν χρήση πιξήδας, διαβήτη και χάρακα. Η ορθογραφεία από την άλλη χαρακτηρίζεται. Λειτουργούν ως οι πρώτες πράξεις που κάνει ένας αρχιτέκτονας. (Vitruvious arts of architecture p.3)

% Ο βιτρούβιος έχει στο νου του τη διαδικασία του σχεδιαμού και σύνθεσης, όπως ξεχωρίζει από το γενικό του σύνολο (ολοκληρωμένο έργο). Είναι η διαδικασία του σχηματισμού της κάτοψης και των σχεδίων, όχι τα ίδια τα σχέδια.



\subsection{Ευρυθμία (Eurytmia)}
  
Ευρυθμία για τον Βιτρούβιο σημαίνει ποιοτικός, καλός ρυθμός στην αρχιτεκτονική, ο οποιος πρέπει αν είναι ευχάριστος στις αισθήσεις. Είναι η όμορφη όψη και η ισορροπημένη εμφάνση των μελών του συνόλου. \cite[σ.~51,95]{vitruvius-lefas} Αναφέρει πως επιτιγχάνεται με την ανταπόκριση των μεγεθών του έργου, δηλαδή τις αναλογίες του ύψους, μήκους και πλάτους, με την συμμετρία (πάλι ίσως με την πιο γενική της έννοια, δηλαδή ισόρροπη) \cite[σ.~95,498]{vitruvius-lefas,scranton_vitruvius_1974}. 

  
  
% Μάλλον αυτό που έχει σοτυ νου του είναι η ποίτητα του ρυθμού πιο συγκερκιμένα. Ο δυναμικός παράγωντας στη παρουσίαση του έγου, ήδη κινήσεων ενσωματομένα στις γραμμές. Η αρχιτεκτονική πρέπει να έχει καλό ρυσμό. Ευχάριστο στις αισθήσεις. Παράλληλα έινια συνδεδεμένο με την  συμμετρία.



\subsection{Συμετρία (Symetry)}

Η συμμετρία είναι για τον Βιτρούβιο η θεμελιώδης αρχή της θεωρίας του πάνω στις συνιστώσες της αρχιτεκτονικής. Δεν έχει την κοινή και γνωστή έννοια της συμμετρίας κατά τον Ευκλείδη, αλλά εκφράζει την συμφωνία και την εναρμόνιση των υπομέρους τμημάτων ενός συνόλου σε σχέση με το ίδιο.

Η συμμετρία όπως και η τάξις και η ευρυθμία, πατάει πάνω στην έννοια του μέτρου (Πλατωνική αντίληψη της συμμετρίας) και συχρόνως είναι αποτέλεσμα αναλογικών και αριθμητικών σχέσεων. Συγκεκριμένα, έχει τις βάσεις της στις αναλογίες του ανθρώπινου σώματος του οποίου τα μέλη εκ φύσεως έχουν μία αναλογική σχέση μεταξύ τους. Όπως αναφέρει, ο πήχης, η παλάμη, το δάκτυλο, ολόκληρο το χέρι, καθιστούν εύρυθμο το ανθρώπινο σώμα έτσι και σε ένα κτίσμα, τα μέλη και τα μεγέθη του πρέπει να έχουν μία αναλογία μεταξύ τους ώστε να επιτιγχάνεται η εναρμόνιση τους με το σύνολο. \cite[σ.~51,187]{vitruvius-lefas,lefas-fundamental}

Η συμμετρία έχει έναν πιο "τεχνικό" χαρακτήρα. Συνδέεται άμεσα με την τάξις και τη ευρυθμία καθώς αποτελέι απαραίτητη προϊπόθεση της ύπαρξης τους. Συχρόνως όμως είναι ανεξάρτητη από αυτές. Η συμετρία έχει να κάνει με τη σχέση των μελλών του έργου και τις αριθμιτικές αναλογίες που παρατηρούνται στη φύση (ανθρώπινο σώμα). \cite[σ.~96]{vitruvius-lefas}



% Η συμετρία για τον Βιτρούβιοι είναι η θεμελιώδης αρχή στην ολόκληρη έννοια του σχεδιαμόυ του. 
% Για τον ίδιο η λέξη συμετρία δεν έχει τον ορισμό κατά τον Ευκλείδη. [Παραπομή] 
% Αντί αυτού, στο πρωτο σκέλος της ερμινείας του, ο βιτρούβιος παραπέμπει τον ορισμός της συμμετρίας στην πλατωνική αντίληψη της, η οποία έχει ως βάση το "μέτρο". 
% Στο δεύτερο σκέλος της παραπέμπει στην αντίληψη των Πυθαγορείων για τον κόμσο αλλά και τον Πλάτονα, ορίζοντας τη συμετρία ως αποτέλεσμα αναλογικών και αρισθητικών σχέσεων. ΄Όπως ακριβώς και στις έννοιες της τάξης και ευρυθμίας. \Cite[σ.~51,96]{vitruvius-lefas}

%  \begin{description}
%    \item[Απόσπασμα] "Η συμετρία είναι η συμφωνία που προκύπτει από την εναρμόνιση του κάθε μέλου του έργου με τα άλλα, Βάση ένος επιλεγμένου μεταξύ τους μέτρο."
%  \end{description}

% Η συμετρία η συμφωνία που προκύπτει από τη σχέση των μελώ του έργου. Έχει τις βάσεις της στις αριθμιτικές αναλογίες του ανθρώπινου σώματος, του οποίου Τα μέλη έχουν εκ φύσεως συγκεκριμένες αναλογίες μεταξύ τους. Ο πήχης, το πόδι, η παλάμη, το δάκτυλο και άλλα μέρη του σώματος καθιστούν εύρυθμο το ανθρώπινο σώμα. \cite[σ.~51]{vitruvius-lefas}.

% Ακριβώς το ίδιο συμβαίνει και στα κτήρια, ή μάλλον, πρέπει να συμαβάινει κατα την διαδικασία του σχεδιαμσού και του χτισίματος. Αμέσως μετά ο Βιτρούβιος στον ορισμό της συμετρίας παρουσιάζει ορισμένα παραδείγματα για να εξηγείσει την έννοια της συμετρίας. Οι ναοί κατέχουν τη συμετρία που περιγράφει, καθώς τα μέρη τους οπως είναι ο κίονας και η τρίγλυφος, είναι εναρμονισμένα μεταξύ τους cite[σ.~51]{vitruvius-lefas}. Στη συνέχει του συγκράματος του αναφέρει και ορισμένα άλλα παρδείγματα όπως για στο βιβλιο IV, 6,1 περί της συμετρίας της Δωρικής θύρας. Η συμετρία έχει να κάνει με τις αναλογίες και τη σχέση των μεγεθών ενός κτηρίου ακριβώς όπως παρατηρείται και στις αναλογίες του σώματος του ανθρώπου. Έτσι επιτυγχάνεται στην ουσία η συμετρία, δηλαδή έτσι μπορεόυμε να επιτύχουμε την συμφωνία από την εναρμόνιση των μελών ενός συνόλου \cite[σ.~187]{lefas-fundamental}.

% Όλες οι τεχνικές κατασκυές έχουν ένα τέτοιο σύστημα συμετρίας. Όπως είπαμε η συμμετρία έχει να κάνει με μία κοινή κλίμακα μέτρου, δηλαδή μία κοινή μονάδα μέτρησης. Η οποία με τη συνέχει προέρχεται από τη σχέση τους μεταξύ τους δηλαδή το σύνολο τους, το κτίσμα ολόκληρο.

% Η συμμετρία έχει κάπως έναν πιο τεχνητό χαρακτήρα. Συνδέεται άμεσα με την τάξη και την Ερυθμία σε επίπεδο όπυ είναι σχεδόν απαράιτητη η πρϋπόθεση της αλλά ταυτόχρονα παραμένει ξεχωριστή. ΌΠως είπαμε η τάξη μεταφέρει μία ιεραρχική έννοια. ***

% Proportio. Στα Ελληνικά αναλογία. Κατά τον Βιτρούβιο η αναλογία παραπέμπει στην έννοια της γεωμετρίας κατά τον Ευκλείδη. Η αναλογία είναι οι αριθμιτικές σχέσεις μεγεθών σε ένα έργο. Η αναλογία είναι στην ουσία η ισομετρία, ενώ η γεωμετία εμπεριέχει μία ποιτική πτυχή.

% Με το να ορίζει τις αναλογίες των μελών ενός συνόλου, στην ουσία ο Βιτρούβιος ορίζει τη σχέση των μεγεθών, άρα και την ιεραρχία τους.



\subsection{Κοσμιότητα (Decor)}
  
Η κοσμιότητα έχει να κάνει με την "αισθητική" αλλά με την έννοια του "καθωσπρεπει" και της "αρμόζουσας" σχέσης του κτίσματος σε σχέση με κοινωνικούς, ιστορικούς ή φυσικούς παράγωντες. Μιλόντας για κοσμιότητα ο Βιτρούβιος επιχειρέι να προσδώσει έναν χαρακτηρισμό "καταλληλότητας" στο κτίσμα έχοντας υπόψη του τη θεματολογία του. Στην συνέχεια του συλλογισμού του παρουσιάζει ορισμένα παραδέιγματα παρουσίας της κοσμιότητας όπως την επιλογή του αρμόζων ρυθμού (κορινθιακός, δωρικός, κλπ.) για τους ναούς κάθε θεότητας ή όταν ένα κτήριο εποφελείται άψογα από τον προσανατολισμό του. \cite[σ.~498-499]{scranton_vitruvius_1974}Καταλλήγοντας έτσι στο συμπέρασμα πως υπάρχουν τρεις περιπτώσεις κοσμιότητας, οι οποίες είναι:

\begin{itemize}
  \item Θεματισμό: Όταν το έργο ανταποκρίνεται στο στο "θέμα" του.
  \item Έθει: Κοσμιότητα που έχει να κάνει με την τήρηση της παράδωσης.
  \item Φύσει: Έχει να κάνει με την ένταση του έργου στη φύση.
\end{itemize}
Ένα έργο κατέχει την κοσμιότητα όταν εφαρμόζει τις τρείς αυτές περιπτώσεις. \cite[σ.~51-53,97]{vitruvius-lefas}

Ο Βιτρούβιος επιχειρεί να προσδώσει μία έννοια "καταλληλότητας" του κτηρίου όταν αναφέρεται σε αυτή τη λέξη δηλαδή να είναι εντός του θέματος
  
  
  
  
% Μάλον έχει να κάνει με τη  αισθητική. Δεν φαίνεται να έχει ιδιαίτερη σχέση με τις δικές μας αισθητικές αρετές.

% Δίνει παραδείγματα τα οποία φρορούν τους γνωστούς ρυθμούς (Δορικός, Ιωνικός, Κορινθιακός) όπουυ ο κάθε ένας προορίζεται και ταιριάζει καλύτερα με συγκεκτιμένες θεότητες.

% ¨οπως φαίνεται Ο Βιτρούβιος προσπαθεί να προσδώσει μια έννοια "καταλληλότητας" στα κτήρια. Φακξιοναλισμός, μορφή ακολουθεί τη χρήση.
  
% Το ρπόβλημα που αντιμετωπίζουμε είναι σημασιολογικό. Αυτό λόγω της γλώσσας του Βιτορύβιου και των μετα φρστών.
  


%Διαβάζοντας το εκίεμνο του παρατηστουμε οτι αυτές οι αρχές τις αρχιτεκτονικής αναφέρονται και αποτελούν το αισθητικό κομάτι ενός κτηρίου, τις ποιότητες δηάδή ενός κτηρίου. Το προιόν του αρχιτέκτονα. 

% Βιτρούβιο όμως έχει κάτι άλλο στο νου του. ναι μεν λέει πως οι αρχές αφορούν το αισθητικό κομμάτι, δηλαδή το έργο τέχνης καθέ αυτού, αλλά αναφέρει πως αφορούν και τηη ίδια την τέχνη του αρχιτέκτονα, δηλα΄δή τι κάνιε ο ίδιος. Η πρακτική του ενέργεια. 
%Κατά μία έννοια τις χωρίζει σε τάξη, διάθεση, και οικονομία και η δεύτερη κατηγορία είναι η ευριθμία, η συμετρία και η κοσμιότητα. Η πρώτη έχει να κάνει με την διαδικάσίατου αρχιτέκτονα. Η δεύτερη έχει να κάνει με τις ποιότητες. (Vitruvious arts of architecture p.3 και 5)
  
%  Όπως έχει αναφερθεί η γλώσσα του Βιτρούβιου είναι κάπως περίπλοκή. ο ίδιος   αναφέρει στο τέτλος του πρώτου κεφαλάιου του πρώτου βιβλίου, οτι δεν είναι   φιλόσοφος ούτε είναι ειδικός στη γραμματιιή αλλά ε΄νας αρχιτέκτονας, ειδικός   στον τομέα του. Πάνω κάτω υπονοεί πως τα δικά του λε΄γην έχουν μία δοση   προτοτυπίας. Όπως είπαμε η γλώσσα που χρησιμοποιεί δεν επιτρέπει την σίγουρη   και ορθή μετάφραση. και οι διαφορετικές μεταφράσεις δεν βοηθάνε επίσης.   Αφείνεται έτσι χώρος για περετέρο ερμεινεία των λε΄γην του. (Vitruvious arts   of architecture p.3)
  
%  Για παράδειγμα η λέξη διάθεσης (arragement), μία απότ ις αρχές της αρχιτεκτονικής που αναφέρει ο ΒΙτρούβιος, έχει διαφορετικές σημασίες στα αγγλικά. Η πράξη του να τοποθετούμε πράγματα στο χώρο ή η τοποθέτηση των πραγμάτων μεταξύ τουυς σε ένα σύνολο. (Vitruvious arts of architecture p.3)  
  
  

\subsection{Οικονομία (Distributio)}

ο όρος έχει να κάνει με τη διαδικασία του αρχιτέκτονα στο να σχεδιάσει και να κατασκευάσει έναν λογικό προϊπολογισμό κόστους, έχοντας στο νού του την τοποθεσία του έργου, την οικονομική άνεση και το κύρος πελάτη. Τηρείται, όπως λεεί, όταν οι ανάγκες του έργου βρίσκονται εντώς λογίκών πλαισίων. \cite[σ.~55,97]{vitruvius-lefas} Η λατινική λέξη είναι και εκείνη της οικογένειας με κατάληξη σε -tio, -tionis, όπως και η dispositio, δηλώνοντας και αυτή μία ενέργεια, μία διαδικασία. 

Η οικονομία παραπέμπει στη έννοια της κοσμιότητας καθώς και στις δύο περιπτώσεις ο Βιτρούβιος μιλάει περί ένταξης του έργου εντός ορισμένων πλαισίων. Η κοσμιότητα αφορά τη θεματολογία του έργου ενώ η οικονομία έχει να κάνει περισσότερο με τον ίδιο τον πελάτη.



% Η λέξη αυτή όπως χρησιμοποιείται από τον Βιτρούβιο είναι λέξη κατάληξης -tio -tionis, δηλαδή δείχνουν μία έννοια εκτέλεσης μίας διαδικασίας.
  
%Ο όρος έχει να κάνει με την διαδικασία του αρχιτέκτονα στο να σχεδιάσει, κατασκευάσει έναν λογικό προϊπολογισμό κόστους, έχοντας στο νου του την οικονομική άνεση και το κύρος του πελάτη. 
% Ίσως να υπάρχει μία ανακρίβεια στη σκέψη του καθώς αυτός ο ορισμός τίνει να κατεγθείναιται προς την κοσμιότητα. Λογικά ο Βιτρούβιος δείχνει περισσότερη προσοχή στον πελάτή καθώς περιγράφει ορισμένες περιπτώσεις παραδείγματα και ταυτόχρονα τονίζει το οικονομικό του επίπεδο. Στην κοσμιότητας δείνει έμφαση στη μορφή του κτίσματος. (Vitruvious, arts of architecture p.4 497)