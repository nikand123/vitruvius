\section{Συμπεράσματα}

Μέσα από αυτή τη μικρή αναδρομή/ανασκόπηση του φημισμένου έργου του Βιτρούβιου και παρά τις όποιες διαφορετικές προσεγγίσεις ως προς τις επιμέρους ερμηνείες, διαπιστώσαμε ότι το έργο παραμένει επίκαιρο και σύγχρονο. Ίσως η συνεισφορά του στην εξέλιξη αυτής της ίδιας της επιστήμης της αρχιτεκτονικής, να μην είναι σημαντική αλλά η αξία του σαν ένας οδηγός των ποιοτικών χαρακτηριστικών που πρέπει να διακρίνουν, τόσο το ίδιο το αρχιτεκτονικό έργο όσο και τον αρχιτέκτονα, είναι παραδόξως πολύ μεγάλη ακόμη και σήμερα.

Είναι χαρακτηριστικό ότι όλες οι κανονιστικού τύπου αρχές της αρχιτεκτονικής 
που περιγράφει ο Βιτρούβιος, ισχύουν σχεδόν αναλλοίωτες και σήμερα. Επιπλέον, 
ακόμα και τα σχετικώς υποκειμενικά χαρακτηριστικά που πρέπει να διακρίνουν έναν 
αρχιτέκτονα, σήμερα φαίνονται ίσως πιο επίκαιρα από ποτέ. Είναι κοινά παραδεκτό 
ότι σήμερα διανύουμε τη περίοδο της τέταρτης βιομηχανικής επανάστασης, ένα από 
τα βασικά χαρακτηριστικά της οποίας είναι οι απαίτηση για ένα ευρύ πεδίο 
διεπιστημονικών γνώσεων, κάτι που ο Βιτρούβιος περιέγραψε για τον αρχιτέκτονα 
ήδη από το 100 π.Χ.

Η αξία του έργου του Βιτρούβιου εστιάζεται όμως και στο γεγονός του εύρους των 
διαφορετικών επιστημονικών τομέων που συνδυάζει με αυτόν της αρχιτεκτονικής. 
Παρά το γεγονός ότι στη παρούσα εργασία περιοριστήκαμε στο πρώτο μόνο βιβλίο 
του έργου του, οι αναφορές και τα παραδείγματα από τα υπόλοιπα μέρη, οδηγούν 
στο συμπέρασμα ότι ελάχιστα παρόμοια έργα ακόμη και σήμερα προσεγγίζουν την 
αρχιτεκτονική με τόσο ολιστικό τρόπο. 