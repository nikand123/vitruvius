% !TEX program = xelatex
% !TEX encoding = UTF-8 Unicode

\PassOptionsToPackage{dvipsnames}{xcolors}
\documentclass[%
              nobib,
              sfsidenotes,
              notoc,
              justified,
              a4paper,
              twoside,
              nohyper,
              ]%
              {tufte-handout}




%%%%%%%%%%%%%%%%%%%%%%%%%%%%%%%%%%%%%%%%%%%%%%%%%%%%%%%%%%%%%%%%%%%%%%%%%%%%%%%
% Το πακέτο tufte 'εχει πρόβλημα όταν χρησιμοποιείται με xelatex. Συγκεκριμένα
% χαλάει η σχεδίαση για την απεικόνιση κεφαλαίων χαρακτήρων και ειδικότερα
% η εντολή /MakeTextUppercase (ορίζεται στην κλάση tufte στο tufte-common.def).
% Ο παρακάτω κώδικας λύνει το πρόβλημα (το βρήκα εδώ:
% https://tex.stackexchange.com/questions/202142/problems-compiling-tufte-title-page-in-xelatex)
\usepackage{ifxetex}
\ifxetex
  \newcommand{\textls}[2][5]{%
    \begingroup\addfontfeatures{LetterSpace=#1}#2\endgroup
  }
  \renewcommand{\allcapsspacing}[1]{\textls[15]{#1}}
  \renewcommand{\smallcapsspacing}[1]{\textls[10]{#1}}
  \renewcommand{\allcaps}[1]{\textls[15]{\MakeTextUppercase{#1}}}
  \renewcommand{\smallcaps}[1]{\smallcapsspacing{\scshape\MakeTextLowercase{#1}}}
  \renewcommand{\textsc}[1]{\smallcapsspacing{\textsmallcaps{#1}}}
  \usepackage{fontspec}
\fi
%%%%%%%%%%%%%%%%%%%%%%%%%%%%%%%%%%%%%%%%%%%%%%%%%%%%%%%%%%%%%%%%%%%%%%%%%%%%%%%

\usepackage{todonotes}

\usepackage{enumitem}

    \usepackage{amsmath}
     
    % Set up the images/graphics package
    \usepackage{graphicx}
       \setkeys{Gin}{width=\linewidth,totalheight=\textheight,keepaspectratio}
       \graphicspath{{img/}}
                 
    
                 
    % The following package makes prettier tables.  We're all about the bling!
    \usepackage{booktabs}
                 
    % The fancyvrb package lets us customize the formatting of verbatim
    % environments.  We use a slightly smaller font.
    \usepackage{fancyvrb}
       \fvset{fontsize=\normalsize}
                 
    % Small sections of multiple columns
    \usepackage{multicol}
      
    % Ρυθμίσεις γλώσσας (ελληνικά)
    \usepackage{polyglossia}        % Μηχανή στοιχειοθεσίας πολλών γλωσσών
        \defaultfontfeatures{Mapping=tex-text}
    
    \usepackage{fontspec}     % Επιλογή και ρύθμιση γραμματοσειρών
        \setdefaultlanguage[variant=modern]{greek}  % Επιλογή βασικής γλώσσας κειμένου
        \setotherlanguages{english}                % Επιλογή δευτερεύουσας γλώσσας κειμένου
  %\newfontfamily\greekfont{Georgia Pro}        % Ορισμός γραμματοσειράς που θα
  % χρησιμοποιήσω στη συνέχεια. Έχουν smallcaps: GFS Artemisia, GFS Elpis,
  % GFS Didot (πολύ καλή), GFS Bodoni (πολύ καλή), GFS Neohellenic (καλή)
  %\newfontfamily\greekfontsc{GFS Bodoni}
 % \newfontfamily\greekfontsf{Arial} % Sans Serif γραμματοσειρά
        \setmainfont{Arial}[Script=Greek]   % Η βασική γραμματοσειρά του κειμένου
        \setsansfont{Arial}[Script=Greek] % Η Sans γραμματοσειρά
        \setmonofont{Arial}[Script=Greek]

%%%% Πακέτο tikz για τη διαχείριση γραφικών
\usepackage{tikz}
\usepackage{environ} % χρησιμοποιείται στη συνέχεια για τη δημιουργία νέου 
                     % περιβάλλοντος tikz
  \usetikzlibrary{arrows.meta, chains, positioning}
  \usetikzlibrary{calc,matrix,shadows}


%%%%%%%%%%%%%%%%%%%%%%%%%%%%%%%%%%%%%%%%%%%%%%%%%%%%%%%%%%%%%%%%%%%%%%%%%%%%%%%
%%%% DROP SHADOW
% Με τον παρακάτω κώδικα τελικά δημιουργούμε μία εντολή, την \shadowimage, η 
% οποία μπορεί να αντικαταστήσει την \includegraphics. Η ιδιαιτερότητά της είναι
% ότι εισάγει τις εικόνες με μία ψευδαίσθηση 3D. Το βρήκα εδώ:
% https://tex.stackexchange.com/questions/81842/creating-a-drop-shadow-with-guassian-blur 
% code adapted from https://tex.stackexchange.com/a/11483/3954

% some parameters for customization
\def\shadowshift{3pt,-3pt}
\def\shadowradius{6pt}

\colorlet{innercolor}{black!60}
\colorlet{outercolor}{gray!05}

% this draws a shadow under a rectangle node
\newcommand\drawshadow[1]{
    \begin{pgfonlayer}{shadow}
        \shade[outercolor,inner color=innercolor,outer color=outercolor] ($(#1.south west)+(\shadowshift)+(\shadowradius/2,\shadowradius/2)$) circle (\shadowradius);
        \shade[outercolor,inner color=innercolor,outer color=outercolor] ($(#1.north west)+(\shadowshift)+(\shadowradius/2,-\shadowradius/2)$) circle (\shadowradius);
        \shade[outercolor,inner color=innercolor,outer color=outercolor] ($(#1.south east)+(\shadowshift)+(-\shadowradius/2,\shadowradius/2)$) circle (\shadowradius);
        \shade[outercolor,inner color=innercolor,outer color=outercolor] ($(#1.north east)+(\shadowshift)+(-\shadowradius/2,-\shadowradius/2)$) circle (\shadowradius);
        \shade[top color=innercolor,bottom color=outercolor] ($(#1.south west)+(\shadowshift)+(\shadowradius/2,-\shadowradius/2)$) rectangle ($(#1.south east)+(\shadowshift)+(-\shadowradius/2,\shadowradius/2)$);
        \shade[left color=innercolor,right color=outercolor] ($(#1.south east)+(\shadowshift)+(-\shadowradius/2,\shadowradius/2)$) rectangle ($(#1.north east)+(\shadowshift)+(\shadowradius/2,-\shadowradius/2)$);
        \shade[bottom color=innercolor,top color=outercolor] ($(#1.north west)+(\shadowshift)+(\shadowradius/2,-\shadowradius/2)$) rectangle ($(#1.north east)+(\shadowshift)+(-\shadowradius/2,\shadowradius/2)$);
        \shade[outercolor,right color=innercolor,left color=outercolor] ($(#1.south west)+(\shadowshift)+(-\shadowradius/2,\shadowradius/2)$) rectangle ($(#1.north west)+(\shadowshift)+(\shadowradius/2,-\shadowradius/2)$);
        \filldraw ($(#1.south west)+(\shadowshift)+(\shadowradius/2,\shadowradius/2)$) rectangle ($(#1.north east)+(\shadowshift)-(\shadowradius/2,\shadowradius/2)$);
    \end{pgfonlayer}
}

% create a shadow layer, so that we don't need to worry about overdrawing other things
\pgfdeclarelayer{shadow} 
\pgfsetlayers{shadow,main}

\newsavebox\mybox
\newlength\mylen


% Η εντολή \shadowimage!!!

\newcommand\shadowimage[2][]{%
\setbox0=\hbox{\includegraphics[#1]{#2}}
\setlength\mylen{\wd0}
\ifnum\mylen<\ht0
\setlength\mylen{\ht0}
\fi
\divide \mylen by 120
\def\shadowshift{\mylen,-\mylen}
\def\shadowradius{\the\dimexpr\mylen+\mylen+\mylen\relax}
\begin{tikzpicture}
\node[anchor=south west,inner sep=0] (image) at (0,0) {\includegraphics[#1]{#2}};
\drawshadow{image}
\end{tikzpicture}}
%%%%%%%%%%%%%%%%%%%%%%%%%%%%%%%%%%%%%%%%%%%%%%%%%%%%%%%%%%%%%%%%%%%%%%%%%%%%%%
%%%%%%%%%%%%%%%%%%%%%%%%%%%%%%%%%%%%%%%%%%%%%%%%%%%%%%%%%%%%%%%%%%%%%%%%%%%%%%%
% ΧΡΩΜΑΤΑ
% Χρειάζεται το πακέτο xcolor που φορτώνεται αυτόματα από την κλάση (δες στην
% αρχή). Μπορείς να χρησιμοποιήσεις όποια από τα παρακάτω χρώματα θέλεις ή
% να δημιουργήσεις δικά σου.
\definecolor{shadecolor} {gray}{0.94}%{gray}{0.95,0.95,0.95}
\definecolor{light-gray} {rgb}{0.97,0.97,0.97}
\definecolor{codegreen}  {rgb}{0,0.6,0}
\definecolor{codegray}   {rgb}{0.5,0.5,0.5}
\definecolor{codepurple} {rgb}{0.58,0,0.82}
\definecolor{backcolour} {rgb}{0.95,0.95,0.92}

\definecolor{intcolor}{HTML}{CA0020}
\definecolor{extcolor}{HTML}{0571B0}

\definecolor{toc} {rgb}{102, 6, 0}

\definecolor{background}{HTML}{F8F9F9}
\definecolor{border}{HTML}{A93226}

\definecolor{thered}    {rgb} {0.65,0.04,0.07}
\definecolor{thegreen}  {rgb} {0.06,0.44,0.08}
\definecolor{theblue}   {rgb} {0.02,0.04,0.48}
\definecolor{sectioning}{gray}{0.44}
\definecolor{thegrey}   {gray}{0.5}
\definecolor{theframe}  {gray}{0.75}
\definecolor{theshade}  {gray}{0.94}

\definecolor{webgreen}{rgb}{0,0.75,0}
\definecolor{webred}{rgb}{0.75,0,0}
%%%%%%%%%%%%%%%%%%%%%%%%%%%%%%%%%%%%%%%%%%%%%%%%%%%%%%%%%%%%%%%%%%%%%%%%%%%%%%%

%%%%%%%%%%%%%%%%%%%%%%%%%%%%%%%%%%%%%%%%%%%%%%%%%%%%%%%%%%%%%%%%%%%%%%%%%%%%%%%
%ΥΠΟΣΤΗΡΙΞΗ ΒΙΒΛΙΟΓΡΑΦΙΚΩΝ ΑΝΑΦΟΡΩΝ
% ΠΡΟΣΟΧΗ! η biblatex να φορτώνεται μετά το polyglossia (δες preamble-lang.tex)

\usepackage{hyphenat}
\usepackage[
  style     = ieee, %style=verbose,
  citestyle = numeric,
  bibstyle  = numeric,
  %autocite  = footnote,
  backend   = biber,
  sorting   = none,
  backref   = true,
]{biblatex}

\DefineBibliographyStrings{greek}{%
  backrefpage = {σελίδα},% originally "cited on page"
  backrefpages = {σελίδες},% originally "cited on pages"
}

% Μακροεντολή για την εκτύπωση τοπικής βιβλιογραφίας ανά κεφάλαιο. Χρησιμοποίησέ
% το όνομα της μακροεντολής (subbibforchapter) σαν επιλογή στην τοπική εκτύπωση
% βιβλιογραφίας π.χ. στο κεφάλαιο που θέλεις να εκτυπωθεί η τοπική βιβλιογραφία
% γράψε \printbibliography[heading = subbibforchapter, ...]. Η τοπική βιβλιογραφία
% θα εκτυπωθεί σαν ακόμη μία ενότητα του κεφαλαίου
% Το βρήκα εδώ https://tex.stackexchange.com/questions/49941/multiple-bibliographies-and-one-global-bibliography-all-with-global-labels
% [\refname\ \thechapter~Κεφαλαίου]

\defbibheading{subbibforchapter}[\refname]{\section{#1}}
% αν θέλεις μη αριθμημένη ενότητα χρησιμοποίησε ...{\section*{#1}}

\addbibresource{./content/mainbiblio.bib}
\addbibresource{./content/websources.bib}

\renewcommand*{\bibfont}{\small}
%%%%%%%%%%%%%%%%%%%%%%%%%%%%%%%%%%%%%%%%%%%%%%%%%%%%%%%%%%%%%%%%%%%%%%%%%%%%%%%
%%%%%%%%%%%%%%%%%%%%%%%%%%%%%%%%%%%%%%%%%%%%%%%%%%%%%%%%%%%%%%%%%%%%%%%%%%%%%%%
%% Ρυθμίσεις πακέτου hyperref

\usepackage{hyperref}
\hypersetup{
  colorlinks  = true,
  linkcolor   = Red!60!black, %intcolor, %BrickRed,
  anchorcolor = black,
  citecolor   = Green,
  filecolor   = cyan,
  menucolor   = Red!60!black,
  runcolor    = cyan,
  urlcolor    = extcolor, %NavyBlue,
%  bookmarks   = true,          % Το pdf αρχείο να έχει bookmarks
  pdfpagemode = UseOutlines,   % Όταν ανοίγει το pdf να φαίνεται η δομή του
                               % εγγράφου (κεφάλαια, ενότητες)
  pdftitle={},                 % title
  pdfauthor={},                % author
  pdfsubject={},               % subject of the document
  pdfcreator={},
}

\urlstyle{same}


\title{Βιτρούβιος μία εμβληματική μορφή της Αρχιτεκτονικής}
\author[Νικόλαος Ανδρόνικος Μαυρόπουλος]{Νικόλαος Ανδρόνικος Μαυρόπουλος}
%\date{24 January 2009}  % if the \date{} command is left out, the current date will be used

\begin{document}
              
  \maketitle% this prints the handout title, author, and date
  
  
  \bigskip            
  \begin{abstract}
  \noindent This document describes the Tufte handout \LaTeX\ document style.
  It also provides examples and comments on the style's use.  Only a brief
  overview is presented here; for a complete reference, see the sample book.
  \end{abstract}
  \bigskip
              
  
              
  

\section{Εισαγωγή}\label{sec:intro}

The Tufte-\LaTeX\ document classes define a style similar to the
style Edward Tufte uses in his books and handouts.  Tufte's style is known
for its extensive use of sidenotes, tight integration of graphics with
text, and well-set typography.  This document aims to be at once a
demonstration of the features of the Tufte-\LaTeX\ document classes
and a style guide to their use. \cite{erismis_critical_2013}

\begin{marginfigure}%
  \shadowimage[width=\linewidth]{vitruvian-man}
  \caption{\footnotesize Ο  Άνθρωπος του Βιτρούβιου (Vitruvian Man) σχεδιασμένος
  από το Λεονάρτο ντα Βίντσι
  (\href{https://en.wikipedia.org/wiki/Leonardo_da_Vinci}{Leonardo da Vinci}).
  Ένα χαρακτηριστικό έργο εφαρμογής των αρχών που περιέγραψε ο Vitruvius,
  σχεδιασμένο από ένα από τους μεγαλύτερους ζωγράφους όλων των εποχών
  (πηγή: \cite{wikipedia:vitruvianman}).}
  \label{fig:vitruvian-man}
\end{marginfigure}
  % !TEX root = ../main.tex

\section{Σημειώσεις Ανδρόνικου}

\subsection{Σταδιοδρομία}

Ο Μάρκος πόλλιο Βιτρούβιος ήταν στρατηωτικός μηχανικός, αρχιτέκτονας και συγκραφέας ο οποίος έζησε κατά τη Ρωμαική περίοδο του πρώτου αιώνα π.Χ. Ο Βιτρούβιος είναι γνωστός για το έργο του "Περί αρχιτεκτονικής" (De architectura), ένα σύνολο συγραμμάτων, βιβλίων τα οποία περιγράφουν τη δομή της αρχιτεκτονικής.** Λίγα είναι γνωστά για τον ίδιο. Οι πληροφορίες που υπάρχουν πηγάζουν από το ιίδιο του το έργο \cite[σ.~390]{masterson_status_2004}.

Τα επαγγέλματα των μηαχανικών κανονικά μεταφέρονταν από γονιό σε παιδί.  *** 
Ο Αύγουστος χρησιμοποίησε σαν ευκαιρία την παρουσία των νέων ατόμων, ειδεικών για να αυξήση την ήδη υψηλή πολιτική και στρατηωτική του δήναμη. Ο Βιτρούβιος ήταν ένα από αυτά τα άτομα. (Status, Pay, Pleasure p.1) ***

και στη περίπτωση του Βιτρούβιου οι γονείς φρόντισαν για την εκπαίδευση του σαν αρχιτέκτονας. Ο Βιτρούβιος μάλιστα κάνει αναφορά στο σύγκραμμα του στους δασκάλους του που τον εκπαίδευσαν. (Λέφας σελ 13).

Στη περίοδο που έζησε, της ύστερης Ρωμαικής δημοκταρίας, οι αρχιτέκτονες ώφειλαν να υποστηρίζουν το λαικό κόμμα του Καίσαρα και του κλώδιου (Λέφας σελ 13). Συμπερένεται πως ο Βιτρούβιος άσκησε το επάγκελμα του αρχιτέκτονα στο στρατό του αυτοκράτορα Καίσαρα με ιδιότητα ως τεχνικός στρατηωτικών μηχανών όπου συμπερένουμε οτι ο ίδιος είχε ακολουθήσει τον στρατό του Καίσαρα σε διάφορες μετακινήσεις από τμήματα της Γαλλίας μέχρι την Αφρική. Οι αναμνήσεις και η εμπειρίας από τα ταξίδια του μεταφέρονται και στο έργο του. φαίνεται πως ο ίδος ήαν ένας "aparitor", δηλαδή ένας δημόσιος υπάλληλος του οποίου ο μισθός προερχόταν από το δημόσιο ταμέιο.

Η εποχή που έζησε ο Βιτρούβιος είχε να κάνει με την αλλαγή. Μία εποχή όπου οι τέχνες και τα επαγκέλματα είχαν πια αναβαθμιστεί, κερδίζοντας νέα σημασία σε σχέση με την παράδωση. Η κλασική ανατροφή, όπου είχε θρέψει την κλασική παλιά ρωμαική κουλτούρα δεν ήταν πια σχετική. Η επαγκελαμτική εκπαιδευση είχε πια ξεπεράσει την κλασική. Βρήκαν πάτημα στην ελληνιστική παράδωση. ¨ετσι και η αρχιτεκτονική. (Vitruvius and the liberal arts of arcitecture)

Με σκοπό να βρεθεί μία λύση στην αναγκαία αυτή αλλάγή από την κλασική ρωμαική κουλτούρα, σε μία νέα, σκεπτικιστές υποστήριξαν πως μία συνθεση μεταξύ ελληνιστικου και ρωμαικού ήταν αυτό που χρειάζοντν. Προέβλεψαν πως μέσω μιας σωστής εκπαίδευσης θα επιτυγχανόταν αυτό. Και βρήκαν αυτό που έψαχναν στη εγκύκλειος παιδέια, τις ελευθέριες τέχνες της ελληνιστικής περιόδου. Άτομα τα οποία το είπαν αυτή ήταν ο Κικέρον, ο Βάρρος και ο Βιτρούβιος. Αυτό που είχνα στο νού τους ήταν μία γενική απόκτηση γνώσης για έναν επιστήμονα δεν είχε να κάνει (απαραίτητα) με την επαγγελματική του κατάρτηση πάνω στη λογική, φυσική και ηθηκή, δηλαδή να γίνει φιλόσοφος. Αντί αυτού στο νου τους είχνα μια γενική απόκτηση γνώσεων και απόκτηση πειθαρχέιας. (Vitruvius and the liberal arts of arcitecture)

\subsection{Το έργο του} 

Ο Βιτρούβιος είναι γνωστός για την συγκραφή του "Περί αρχιτεκτονικής" (De architectura), ένα σύνολο από δέκα συγγράμματα όπου αναλύουν και περιγράφουν τις δομές της αρχιτεκτονικής και την αξία του αρχιτέκτονα.
Το σύγγραμμα ολοκληρώθηκε και δημοσιεύτηκε κατά την τελευταία πείοδο ζωής του αρχιτέκτονα και πιθανότατα κατά τη περίοδο όπου αυτοκράτορας ήταν πια ο Αύγουστος, υιοθετημένος ιός του Καίσαρα. ***
Τα βιβλία αποτελούνται από τεκμητιωμένες και ολοκληρωμένες σκέψεις και παρατηρήσεις του συγκραφέα τα οποία είναι αφειερωμένα στον ίδιο τον Αυτοκ΄ρατορα. Οι λόγοι που τον οδήγησαν να συγγράψει τομπερι αρχιτεκτονικής είναι μάλλον προς παροχή υποστήριξης στο οικοδομικό πρόγραμμα του Αύγουστου και στην ενημέρωση πάνω στην αρχιτεκτονική. (Λέφας σελ 14-15)***

Το περιεχόμενο αφορά τους τομέες του αρχιτέκτονα (1.2-3),την εκπαίδευση και τη συμπεριφορά του (1.6), και τα ορθά χαρακτηριστικά των δημοσίων και ιδιοτικών κτισμάτων (βοοκ 5-6), Υλικά (Book 2), ορθές τοποθεσίες κτηρίων (1.4-7). (Status, Pay, Pleasure p.6)

Στον πρόλογο του πρώτου βιβλίου αποκαλέιπτει το κύριο στόχο του, είδη συλογισμών πάνω στην αρχιτεκτονική. Στην αρχή του ρπωτου βιβλίου συζητάει την γνώση του αρχιτέκτονα. έπειτα στο δευτερο κεφάλεο του ρπώτου βιβλίου αναλύει την θεωρητική δομή της αρχιτεκτονικής.από τοτ ρίτο κεφάλαιο και μετά αφειερώνει το υπόλοιπο έγο του στο πρακτικό κοιμμάτι της αρχιτεκτονικής. (vitruvious arts of architectue p.2)

Το πιο σημαντικό ίσως τμήμα του συγγράμμματος είναι η απαιτούμενη γνώση που ωφείλει να έχει ένας αρχιτέκτονας.

\subsection{Η εκπαίδευση του αρχιτέκτονα}
 
\subsection{πρακτική και θεωρητηκή. Fabrica \& Ratiocinatio}
 
 Στο Βιβλίο 1. 1 παράγραφος 2, ο Βιτρούβιος αναφέρει πως άτομα τα οποία βασίστηκαν αποκλειστικά πάνω στη πρακτική εξάσκηση ή  στα γράμματα και τη θεωρεία δεν κατάφεραν να επιτύχουν και να προσφέρουν ένα σωστό κτίσμα. Ατιθέτως, άτομα τα οποία ήταν οχειρωμένα και με πρακτική και θεωρητικιή θεωρεία κατάφεραν να επιτύγχουν. *** (Λέγφας σελ37 1.1 παρ 2)
 
 Αυτό που ξεχωρίζει τον αρχιτέκτονα από ένα τυπικό τεχνήτη είναι η υπεροχή στα γράμματα. Φυσικά η ιδιότητα του τεχνήτη είανι εξήσου σημαντική. ¨ενας αρχιτέκτονας πρέπει να κατέχιε και τις δύο κατηγορίες. Συνθετική υπεροχή (Status, Pay, Pleasure p. 9)
 
Ο Βιτρούβιος απαιτεί από τον αρχιτέκτονα να κατέχει μία πληθόρα γνώσεων, από γεωμετρία και φιλοσοφία μέχρι και ιατρική. Τονίζει βέβαια οτι δεν απαιτείται η υπεροχή στους τομέες αυτούς αλλά μία τουλάχιστον ενασχόληση και εξοικείωση. Υποστηρίζει πως όλοι οι κλαδοι της γνώσης συνδέονται και έχουν κάτι κοινό μεταξύ τους. (Λέφας σελ45 11-12)
Φυσικά μία τέτοια απαίτηση σε γνώσεις απαιτέι παράλληλα μελέτη εκτετανής διάρκειας. (Status, Pay, Pleasure p.7)

Όλες οι τέχνες και επιστήμες έχουν μία κοινή σχέση μεταξύ τους, δηλαδή συνδέονται. Ο βιτρούβιος χρησημοποιεί αυτή την πρόταση σαν επιχείτημα για το λόγω που ο αρχιτέκτονας πρέιπει να κατέχει μία πληθόρα γνώσεων. Γνώσεις που στις μέρες μας θεωρούνται μη απαραίτητς ή σχεδόν μη χρήσημες για έναν αρχιτέκτονα. Ένα χαρακτηριστικό παράδειγμα είναι η αστρονομία. (sollertiae p.5)

 Ο Βιτρούβιος τοποθετέι την αρχιτεκτονική στην εγκύκλειο παιδέια. (Status, Pay, Pleasure) (Λέφας σελ45 11-12)
 Η αρχιτεκτονική καταλμαβάνει την υψηλότερη θέση στον τομέα της κατασκευής, πράγμα που επιτιγχάνεται από τη νεαρή ενασχόληση και μελέτη πάνω στις τέχνες και τα γράμματα.
 
\subsection{Πρακτική και γνωστική}
 
\subsection{Από Πλάτονα σε Βιτρούτβιο}
 
 Ο Πλάτονας και ο Βιτρούβιος έχουν κοινές απόψεις πάνω στο πρακτικό και το θεωρητικό.
 
 Κατά τη εποχή του πλάτονα δεν υπήρχε διαφορά μεταξύ του αρχιτέκτονα και του τεχνήτη. Ο αρχιτέτκονας είναι ο πρώτος τεχνήτης αλλά δεν ήταν εξακριβομένο τι το ξεχώριζε. Εξάλου δεν θεωρούταν από τις καλές τέχνες. (παρ.3-4)
 
 Ο Πλάτονας υποστιρίζει πως η "πρακτική" (praktike) και η "γνωστική" (gnostike) αποτελούν συστατικά της ενότητας της επιστήμης στο σ΄θνολο της. (παρ. 1)
 
 Παραδειγματίζει ένα διακριτικό είδος πρακτικής γνώσης το οποίο είναι επιτακτική ή εκτελεστική παρά καθαρά καίρια (φιλοσοφική, μαθηματική κλπ.). Έχει να κάνει με την εντολή παρά με επιστημονικά γεγονότα ή υπολογισμούς. (παρ. 2)
 
 Τόσο και ο Πλάτονας όσο και ο Βιτρούβιος χαρακτηρίζουν τον αρχιτέκτονα όχι ως έναν απλλό τεχνήτη αλλά ως ένα ορθά μελετημένο ο οποίος κατέχει το πρακτικό κομάτη του χτισήματος και ταυτόχρονα την γνώση πάνω σε επιστήμες. (παρ. 8)
 
 Για τον Πλάτονα η πρακτική και η γνωστική είναι πνευματικές γνώσεις που δεν απαιτούν χειρονακτική ενασχόληση. Ακριβώς και ο Βιτρούβιος περιγράφει σαφέστατα πως η λέξη fabrica είναι επίσης συνδεδεμένη με τη πνευματική διαδικασία. Ονομάζει αυτή τη διαδικασία meditatio. (παρ. 9)
 
 Η τέχνη της αρχιτετονικής απαιτέι γνώσεις που αφορούν το "ξέρω πως" και το "ξέρω αυτό". (παρ 12)
 
 Κι οι δύο συγκρίνουν την την αρχιτεκτονική με τη μουσική. Από τη μία το τελείως πρακτικό κομμάτι της μουσικής, δηλαδή η ενασχόληση με τα όργανα και η παραγγωγή της μουσικής μέσω αυτών και από την άλλη το τελείως θεωριτικό κομμάτι, που αφορά τα μαθηματικά και τη μεταφυσική. (παρ. 12)
 
\subsection{Βιτρούβιος}
 
 Στην αρχή του έργου του, περιγράφει τη διαφορά μεταξύ της πρακτικής μεριάς της αρχιτεκτονικής και την θεωρητική (fabtica και ratiocinatio αντίστοιχα).
 
 Η γνώση του αρχιτέκτονα πηγάζει τόσο από την μελέτη και θεωρία (ratiocinatio) αλλά και από την προσωπική ενασχόληση με τα πράγματα δηλαδή την πράξη, τέχνη (fabrica). Δηλαδή ο αρχιτέκτονας δεν μελετάει απλά αλλά συγχρόνος συνθέτει και κατασκευάζει. (sollertiae p.4)
 
 Δυστοιχώς η γλώσσα του Βιτρούβιου είναι αρκετά περίπλοκη. Μελετητές και μεταφραστές αδυνατούν να καταλήξουν σε ένα κοινό συμπέρασμα. *** (παρ. 6)
 
 Η γλώσσα του Βιτρούβιου αφείνει ορισμένα κενά δυσκολεύοντας έτσι την ερμεινία των προτάσεων του. Διάφοροι μεταφραστές έχουν αποδώσει ορισμένα τμήματα του έργου με διαφορετικό τρόπο. Έχουν δώσει διαφορετικές ερμεινίες. (sollertioa p.1)
 
\subsection{Fabrica}
 
 Από τους περισώτερους μεταφραστές υποστηρίζεται πως η λέξη fabrica μεταφράζεται ως πρακτική, πράκτις. Με την ένοια δηλαδή της επαναλαμβανόμενης άσκησης του χεριού. Η λέξη έτσι όπως χρησημοποιείται από τον Βιτρούβιο δεν σημαίνει πρακτικό κτίσμα ή την τέχνη της κατασκευής. (παρ.7)
 
 Κατά τον μεταφραστή Joseph Gwilt η λέξη fabrica σημαίνει συχνή και συνεχή περισυλλογή (meditatio) των τεχνών σχετικά με τη διαδικασία του χτισήματος. Η διαδικασία αυτή είανι πνευματική και λετσι η fabrica μέσω της διαδικασίας meditatio αποδίδει την απιτούμενη επαγγελματική γνώση και εμπειρία. Στην ουσία η fabrica δεν αφορά αποκλειστικά χειρονακτική γνώση, αν και επιτρέπεται, και αποκτάται με την άμεση ενασχόληση στο κτίσιμο και ανάλογες τέχνες. (παρ 10)
 
 Η τέχνη κατά τον Βιτρούβιο είναι η εξάσκηση της αρχιτεκτονικής ενατένισης. Το ρπατκό κομμάτο του "ξέρω πως". ¨εχει να κάνει με την ανάλυση του χώρου, χωροταξία κλπ. Περιλαμβάνει την όλη επαγκελματική γνώση του αρχιτέκτονα. (παρ 11)
 
\subsection{ratiocinatio}
 
 Όπως προαναφέρθηκε η γλώσσα του Βιτρούβιου είναι αρκετά περίπλοκή και αποτελεί κομμάτι σύγχησης μεταξύ μεταφραστών. Η πρόταση:  Ratiocinatio autem est quae res fabricates sollertiae ac
rationis pro portione demonstrare atque explicare potest, δυσκολεύευι ακόμα τους μελετητές. (sollertioa p.4)
 
 Πρόθεση του Βιτρούβιου είναι να σχηματίσει μία έννοια, θεωρεία που αοφρά την αναλογία. Η λε΄ξη κατα αυτή δεν παρατειρέιται στο κείμενο του. Η λέξη ratiocinatio έχει ως βάση τη λέξη ratio δηλαδή λόγος. Ο λόγος στα ελληνικά έχει μια πληθόρα εννοιών. Η μαθηματική του έννοια είναι η σχέση μεταξύ δυο αριθμών. Η λέξ λόγος είχε να κάνει με την τη σχέση μεταξύ δύο αριθμών (couple of two numbers) πριν πάρει την σημερινή έννοια με βάση τον Ευκλείδη, δηλαδή το πιλίκο δύο αριθμών. (sollertiae p.5)
 
 Η λέξη ratio πηγάζει από την ελληνική λέξη λόγος. Με την ένοια λόγος μπορούμε να αναφερθούμε στη θεωρεία, επεξήγηση, περιγραφή, λόγος με τη μαθηματική έννοια.. κλπ. Ο Αριστοτέλης αναφερόταν στη λέξη λόγος ως τον λογικό ορισμό εννοιών, ακριβώς όπως και οι διάδοχοι χρησιμοποιούν τον όρο. Ομολιως και οι λέξεις ratio, rationale και ratiocinatio. (παρ. 14)
 
 Η θεωρεία δηλαδή έχει να κάνει με τη θεωρεία της αναλογίας (ratio). 
 
 Ο βιτρούβιος χρησημοποιεί σαν παράδειγμα τη σχέση ενός μουσικού μέ έναν ιατρό. Σημειώνει ως κοινό στοιχείο μεταξύ των δύο αυτών επαγγελμα΄τον τον ρυμό. Από τη μία πλευρά την κίνηση του ποδιού ενός μουσικού ωστε να μετράει σωστά τις νότες και από την άλλη την παλμό του ασθενή που μετράει ένας ιατρός. Ο Βιτρούβιος κάνει αναφορά στον Αριστόξενος ο Ταραντίνος Ο οποίος έδωσε ιδηέτερη σημασία στη θεωρία της μουσικής αναλογίας, κάτι το ποίο δεν ήταν εγκεκριμένο από την αρχαιότητα ** Σχέση μεταξύ της ακουστικής αρμονίας και της χωρικής αναλογίας. 
 
 Η διαφορά του ύψους των ήχων βρσίκεται στην αναλογία των νούμεεων και τον ταχητήτων. Κατά τον Αριστόξενο η πρώταση αυτή είναι λάθος. Υποστηρίζει πως οι μουσικοί ήχοι έχουν να κάνου αποκλειστικά με την αίσθηση της ακοής και όχι με την αναλογία. Ο ίδιος δεν ήταν μαθηματικός, έδινε σημασία όμως στη αίθσηση. 
 
 Όπως φαίνεται για τη λέξη ratiocinatio υπάρχει ακόμα σύγχηση. Το τμήμα του κειμένου του Βιτρούβιου που περιγράφει τον ορισμό δεν είναι εύκολο να μεταφρστεί. Όλοι οι μεταφραστές παρουσιάζουν διαφορετικές εκδοχές. Σκατά
 
  Ratiocinatio autem est, quae res fabricatas sollertiae ac rationis proportione demonstrare atque explicare potest
  
  Φαέινεται να υπάρχει σύγχηση με την λέξη proportione ή pro portione. Ο λέφας τη λέει μία λέξη. 
  Σύμφωνα με την ικανότητα και τη συλλογιστική του αρχιτέκτονα.
  
\subsection{Οι αρχές της αρχιτεκτονικής}
  
  Ο {\color{red}\textbf{βιτρούβιος}} στο \emph{κείμενό} του \textit{αναφέρει} τις 6 αρχές της αρχιτεκτονικής. Τάξη, Διάθεση, ευρυθμία, συμετρία, κοσμιότητα και οικονομια. \cite{scranton_vitruvius_1974, vitruvius-lefas}
  
\begin{enumerate}[noitemsep] %αλλίως itemize
  \item ένα
  \item Δύο
  
  \begin{itemize}
    \item ένα.1
    \item ένα.2
  \end{itemize}
  
  \item Τρία
  
    \begin{description}
      \item[όρος 1] μπλαμπλαμπλά ηςθεφη.
      \item[όρος 2] ηςθεφη.
    \end{description}
  
  \item Τέσσερα
\end{enumerate}

Διαβάζοντας το εκίεμνο του παρατηστουμε οτι αυτές οι αρχές τις αρχιτεκτονικής 
αναφέρονται και αποτελούν το αισθητικό κομάτι ενός κτηρίου, τις ποιότητες δηάδή 
ενός κτηρίου. Το προιόν του αρχιτέκτονα. Ο Βιτρούβιο όμως έχει κάτι άλλο στο 
νου του. ναι μεν λέει πως οι αρχές αφορούν το αισθητικό κομμάτι, δηλαδή το έργο 
τέχνης καθέ αυτού, αλλά αναφέρει πως αφορούν και τηη ίδια την τέχνη του 
αρχιτέκτονα, δηλα΄δή τι κάνιε ο ίδιος. Η πρακτική του ενέργεια. Κατά μία έννοια 
τις χωρίζει σε τάξη, διάθεση, και οικονομία και η δεύτερη κατηγορία είναι η 
ευριθμία, η συμετρία και η κοσμιότητα. Η πρώτη έχει να κάνει με την 
διαδικάσίατου αρχιτέκτονα. Η δεύτερη έχει να κάνει με τις ποιότητες. 
(Vitruvious arts of architecture p.3 και 5)
  
  Όπως έχει αναφερθεί η γλώσσα του Βιτρούβιου είναι κάπως περίπλοκή. ο ίδιος 
  αναφέρει στο τέτλος του πρώτου κεφαλάιου του πρώτου βιβλίου, οτι δεν είναι 
  φιλόσοφος ούτε είναι ειδικός στη γραμματιιή αλλά ε΄νας αρχιτέκτονας, ειδικός 
  στον τομέα του. Πάνω κάτω υπονοεί πως τα δικά του λε΄γην έχουν μία δοση 
  προτοτυπίας. Όπως είπαμε η γλώσσα που χρησιμοποιεί δεν επιτρέπει την σίγουρη 
  και ορθή μετάφραση. και οι διαφορετικές μεταφράσεις δεν βοηθάνε επίσης. 
  Αφείνεται έτσι χώρος για περετέρο ερμεινεία των λε΄γην του. (Vitruvious arts 
  of architecture p.3)
  
  Για παράδειγμα η λέξη διάθεσης (arragement), μία απότ ις αρχές της αρχιτεκτονικής που αναφέρει ο ΒΙτρούβιος, έχει διαφορετικές σημασίες στα αγγλικά. Η πράξη του να τοποθετούμε πράγματα στο χώρο ή η τοποθέτηση των πραγμάτων μεταξύ τουυς σε ένα σύνολο. (Vitruvious arts of architecture p.3)  
  
\subsection{Διάθεσης (dispositio)}
  
  Ο Βιτρούβιος χρησημοποιεί ωε συνώνυμα τις λέξεις conlocatio και effectus. 
  Τόσο η λέξη dispositio και η conlocatio αλλά και ολα τα ουσιαστικά με 
  κατάληξη -tio, -tionis, έχουν ως κύρια ερμηνεία την έννοια της εκτέλεσης μίας 
  διαδικασίας και σαν δεύτερη ερμηνεία την έννοια του αποτελέσματος μίας 
  διαδικασίας. Η έννοια της διάθεσης είναι "τοποθέτηση μαζί" και το 
  "πραγματοποιείται" ή το "λειτουργεί", όχι όμως το "αποτέλεσμα", "η 
  προκύπρουσα σύνθεση μορφή". (Vitruvious arts of architecture p.3)
  
  Τόσο ο Granger όσο και ο Morgan μεταφράζουν τη λέξη dispositio ως Arragement, δηλαδή διευθέτηση. Στην ουσία η έννοια της λέξης δεν είναι η τάξη, σειρά δηλαδή η ποιότητα ενός έργου αλλά η διαδικασία της ορθής τοποθέτησης των πργμάτων. Ο υπολογισμός των μέτρων του ίδιου του κρίσματος αλλά και των επιμέων στοιχείων του.
  
\subsection{Ιχνογραφέια, Ορθογραφεία, και τοπογραφέια}
  
  Ο Βιτρούβιος κάνει χρηση των λέξεων αυτών. Ελληνικές λέξεις για τις οποίες ο ίδιος δε χρησιμοποιεί κάποια λατινική. Οι λέξεις αυτές ερμηνεύονται ώς κάτοψη, όψη και ισως προπτική αντίστοιχα. Ο βιτρούβιος βέβαια τις χρησημοποιεί με τον δικό του τρόπο. Η ιχνογραφεία και η τοπογραφεία ορίζονται σαν διαδικασίες, Κάνουν χρήση πιξήδας, διαβήτη και χάρακα. Η ορθογραφεία από την άλλη χαρακτηρίζεται. Λειτουργούν ως οι πρώτες πράξεις που κάνει ένας αρχιτέκτονας. (Vitruvious arts of architecture p.3)
  
  Ο βιτρούβιος έχει στο νου του τη διαδικασία του σχεδιαμού και σύνθεσης, όπως ξεχωρίζει από το γενικό του σύνολο (ολοκληρωμένο έργο). Είναι η διαδικασία του σχηματισμού της κάτοψης και των σχεδίων, όχι τα ίδια τα σχέδια.
  
\subsection{Οικονομία (Distributio)}

  Η λέξη αυτή όπως χρησιμοποιείται από τον Βιτρούβιο είναι λέξη κατάληξης -tio -tionis, δηλαδή δείχνουν μία έννοια εκτέλεσης μίας διαδικασίας.
  
  Ο όρος έχει να κάνει με την διαδικασία του αρχιτέκτονα στο να σχεδιάσει, κατασκευάσει έναν λογικό προϊπολογισμό κόστους, έχοντας στο νου του την οικονομική άνεση και το κύρος του πελάτη. Ίσως να υπάρχει μία ανακρίβεια στη σκέψη του καθώς αυτός ο ορισμός τίνει να κατεγθείναιται προς την κοσμιότητα. Λογικά ο Βιτρούβιος δείχνει περισσότερη προσοχή στον πελάτή καθώς περιγράφει ορισμένες περιπτώσεις παραδείγματα και ταυτόχρονα τονίζει το οικονομικό του επίπεδο. Στην κοσμιότητας δείνει έμφαση στη μορφή του κτίσματος. (Vitruvious, arts of architecture p.4 497)
  
\subsection{Συμετρία (Symetry)}

  Η συμετρία για τον Βιτρούβιοι είναι η θεμελιώδης αρχή στην ολόκληρη έννοια του σχεδιαμόυ του. Για τον ίδιο η λε΄ξη συμετρία δεν έχει τον ορισμό που είναι γνωστός με βάση τον Ευκλείδη, δηλαδή την ισοροποία και το ίδιο μέτρο. Ο βιτρούβιος, , χρησημοποιούν τη λέξη ως μία κοινή κλίμακα μέτρου. Έχοντας όλα τα έχοντας όλα τα μέρη ενός μεγέθους λογικά μετρήσιμα από την άποψη μιας κοινής μονάδας που προέρχεται από την ίδια την εργασία. ** 
  
  \subsection{Ευρυθμία (Eurytmia)}
  
  Μάλλον αυτό που έχει σοτυ νου του είναι η ποίτητα του ρυθμού πιο συγκερκιμένα. Ο δυναμικός παράγωντας στη παρουσίαση του έγου, ήδη κινήσεων ενσωματομένα στις γραμμές. Η αρχιτεκτονική πρέπει να έχει καλό ρυσμό. Ευχάριστο στις αισθήσεις. Παράλληλα έινια συνδεδεμένο με την  συμμετρία.
  
  \subsection{Κοσμιότητα (Decor)}
  
  Μάλον έχει να κάνει με τη  αισθητική. Δεν φαίνεται να έχει ιδιαίτερη σχέση με τις δικές μας αισθητικές αρετές.
  Δίνει παραδείγματα τα οποία φρορούν τους γνωστούς ρυθμούς (Δορικός, Ιωνικός, Κορινθιακός) όπουυ ο κάθε ένας προορίζεται και ταιριάζει καλύτερα με συγκεκτιμένες θεότητες.
  ¨οπως φαίνεται Ο Βιτρούβιος προσπαθεί να προσδώσει μια έννοια "καταλληλότητας" στα κτήρια. Φακξιοναλισμός, μορφή ακολουθεί τη χρήση.
  
  Το ρπόβλημα που αντιμετωπίζουμε είναι σημασιολογικό. Αυτό λόγω της γλώσσας του Βιτορύβιου και των μετα φρστών. 
  
  
 % \begin{fullwidth}
    %\printbibheading[heading=bibintoc, title={Αναφορές}]
    \printbibliography[title={Βιβλιογραφικές αναφορές}]
%  \end{fullwidth}
              
              
\end{document}