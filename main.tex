% !TEX program = xelatex
% !TEX encoding = UTF-8

\PassOptionsToPackage{dvipsnames}{xcolors}
\documentclass[%
              nobib,
              sfsidenotes,
              notoc,
              justified,
              a4paper,
              twoside,
              nohyper,
              ]%
              {tufte-handout}




%%%%%%%%%%%%%%%%%%%%%%%%%%%%%%%%%%%%%%%%%%%%%%%%%%%%%%%%%%%%%%%%%%%%%%%%%%%%%%%
% Το πακέτο tufte 'εχει πρόβλημα όταν χρησιμοποιείται με xelatex. Συγκεκριμένα
% χαλάει η σχεδίαση για την απεικόνιση κεφαλαίων χαρακτήρων και ειδικότερα
% η εντολή /MakeTextUppercase (ορίζεται στην κλάση tufte στο tufte-common.def).
% Ο παρακάτω κώδικας λύνει το πρόβλημα (το βρήκα εδώ:
% https://tex.stackexchange.com/questions/202142/problems-compiling-tufte-title-page-in-xelatex)
\usepackage{ifxetex}
\ifxetex
  \newcommand{\textls}[2][5]{%
    \begingroup\addfontfeatures{LetterSpace=#1}#2\endgroup
  }
  \renewcommand{\allcapsspacing}[1]{\textls[15]{#1}}
  \renewcommand{\smallcapsspacing}[1]{\textls[10]{#1}}
  \renewcommand{\allcaps}[1]{\textls[15]{\MakeTextUppercase{#1}}}
  \renewcommand{\smallcaps}[1]{\smallcapsspacing{\scshape\MakeTextLowercase{#1}}}
  \renewcommand{\textsc}[1]{\smallcapsspacing{\textsmallcaps{#1}}}
  \usepackage{fontspec}
\fi
%%%%%%%%%%%%%%%%%%%%%%%%%%%%%%%%%%%%%%%%%%%%%%%%%%%%%%%%%%%%%%%%%%%%%%%%%%%%%%%

\usepackage{todonotes}

\usepackage{enumitem}

    \usepackage{amsmath}
     
    % Set up the images/graphics package
    \usepackage{graphicx}
       \setkeys{Gin}{width=\linewidth,totalheight=\textheight,keepaspectratio}
       \graphicspath{{img/}}
                 
    
                 
    % The following package makes prettier tables.  We're all about the bling!
    \usepackage{booktabs}
                 
    % The fancyvrb package lets us customize the formatting of verbatim
    % environments.  We use a slightly smaller font.
    \usepackage{fancyvrb}
       \fvset{fontsize=\normalsize}
                 
    % Small sections of multiple columns
    \usepackage{multicol}
      
    % Ρυθμίσεις γλώσσας (ελληνικά)
    \usepackage{polyglossia}        % Μηχανή στοιχειοθεσίας πολλών γλωσσών
        \defaultfontfeatures{Mapping=tex-text}
    
    \usepackage{fontspec}     % Επιλογή και ρύθμιση γραμματοσειρών
        \setdefaultlanguage[variant=modern]{greek}  % Επιλογή βασικής γλώσσας κειμένου
        \setotherlanguages{english}                % Επιλογή δευτερεύουσας γλώσσας κειμένου
  %\newfontfamily\greekfont{Georgia Pro}        % Ορισμός γραμματοσειράς που θα
  % χρησιμοποιήσω στη συνέχεια. Έχουν smallcaps: GFS Artemisia, GFS Elpis,
  % GFS Didot (πολύ καλή), GFS Bodoni (πολύ καλή), GFS Neohellenic (καλή)
  %\newfontfamily\greekfontsc{GFS Bodoni}
 % \newfontfamily\greekfontsf{Arial} % Sans Serif γραμματοσειρά
        \setmainfont{Arial}[Script=Greek]   % Η βασική γραμματοσειρά του κειμένου
        \setsansfont{Arial}[Script=Greek] % Η Sans γραμματοσειρά
        \setmonofont{Arial}[Script=Greek]

%%%% Πακέτο tikz για τη διαχείριση γραφικών
\usepackage{tikz}
\usepackage{environ} % χρησιμοποιείται στη συνέχεια για τη δημιουργία νέου 
                     % περιβάλλοντος tikz
  \usetikzlibrary{arrows.meta, chains, positioning}
  \usetikzlibrary{calc,matrix,shadows}


%%%%%%%%%%%%%%%%%%%%%%%%%%%%%%%%%%%%%%%%%%%%%%%%%%%%%%%%%%%%%%%%%%%%%%%%%%%%%%%
%%%% DROP SHADOW
% Με τον παρακάτω κώδικα τελικά δημιουργούμε μία εντολή, την \shadowimage, η 
% οποία μπορεί να αντικαταστήσει την \includegraphics. Η ιδιαιτερότητά της είναι
% ότι εισάγει τις εικόνες με μία ψευδαίσθηση 3D. Το βρήκα εδώ:
% https://tex.stackexchange.com/questions/81842/creating-a-drop-shadow-with-guassian-blur 
% code adapted from https://tex.stackexchange.com/a/11483/3954

% some parameters for customization
\def\shadowshift{3pt,-3pt}
\def\shadowradius{6pt}

\colorlet{innercolor}{black!60}
\colorlet{outercolor}{gray!05}

% this draws a shadow under a rectangle node
\newcommand\drawshadow[1]{
    \begin{pgfonlayer}{shadow}
        \shade[outercolor,inner color=innercolor,outer color=outercolor] ($(#1.south west)+(\shadowshift)+(\shadowradius/2,\shadowradius/2)$) circle (\shadowradius);
        \shade[outercolor,inner color=innercolor,outer color=outercolor] ($(#1.north west)+(\shadowshift)+(\shadowradius/2,-\shadowradius/2)$) circle (\shadowradius);
        \shade[outercolor,inner color=innercolor,outer color=outercolor] ($(#1.south east)+(\shadowshift)+(-\shadowradius/2,\shadowradius/2)$) circle (\shadowradius);
        \shade[outercolor,inner color=innercolor,outer color=outercolor] ($(#1.north east)+(\shadowshift)+(-\shadowradius/2,-\shadowradius/2)$) circle (\shadowradius);
        \shade[top color=innercolor,bottom color=outercolor] ($(#1.south west)+(\shadowshift)+(\shadowradius/2,-\shadowradius/2)$) rectangle ($(#1.south east)+(\shadowshift)+(-\shadowradius/2,\shadowradius/2)$);
        \shade[left color=innercolor,right color=outercolor] ($(#1.south east)+(\shadowshift)+(-\shadowradius/2,\shadowradius/2)$) rectangle ($(#1.north east)+(\shadowshift)+(\shadowradius/2,-\shadowradius/2)$);
        \shade[bottom color=innercolor,top color=outercolor] ($(#1.north west)+(\shadowshift)+(\shadowradius/2,-\shadowradius/2)$) rectangle ($(#1.north east)+(\shadowshift)+(-\shadowradius/2,\shadowradius/2)$);
        \shade[outercolor,right color=innercolor,left color=outercolor] ($(#1.south west)+(\shadowshift)+(-\shadowradius/2,\shadowradius/2)$) rectangle ($(#1.north west)+(\shadowshift)+(\shadowradius/2,-\shadowradius/2)$);
        \filldraw ($(#1.south west)+(\shadowshift)+(\shadowradius/2,\shadowradius/2)$) rectangle ($(#1.north east)+(\shadowshift)-(\shadowradius/2,\shadowradius/2)$);
    \end{pgfonlayer}
}

% create a shadow layer, so that we don't need to worry about overdrawing other things
\pgfdeclarelayer{shadow} 
\pgfsetlayers{shadow,main}

\newsavebox\mybox
\newlength\mylen


% Η εντολή \shadowimage!!!

\newcommand\shadowimage[2][]{%
\setbox0=\hbox{\includegraphics[#1]{#2}}
\setlength\mylen{\wd0}
\ifnum\mylen<\ht0
\setlength\mylen{\ht0}
\fi
\divide \mylen by 120
\def\shadowshift{\mylen,-\mylen}
\def\shadowradius{\the\dimexpr\mylen+\mylen+\mylen\relax}
\begin{tikzpicture}
\node[anchor=south west,inner sep=0] (image) at (0,0) {\includegraphics[#1]{#2}};
\drawshadow{image}
\end{tikzpicture}}
%%%%%%%%%%%%%%%%%%%%%%%%%%%%%%%%%%%%%%%%%%%%%%%%%%%%%%%%%%%%%%%%%%%%%%%%%%%%%%
%%%%%%%%%%%%%%%%%%%%%%%%%%%%%%%%%%%%%%%%%%%%%%%%%%%%%%%%%%%%%%%%%%%%%%%%%%%%%%%
% ΧΡΩΜΑΤΑ
% Χρειάζεται το πακέτο xcolor που φορτώνεται αυτόματα από την κλάση (δες στην
% αρχή). Μπορείς να χρησιμοποιήσεις όποια από τα παρακάτω χρώματα θέλεις ή
% να δημιουργήσεις δικά σου.
\definecolor{shadecolor} {gray}{0.94}%{gray}{0.95,0.95,0.95}
\definecolor{light-gray} {rgb}{0.97,0.97,0.97}
\definecolor{codegreen}  {rgb}{0,0.6,0}
\definecolor{codegray}   {rgb}{0.5,0.5,0.5}
\definecolor{codepurple} {rgb}{0.58,0,0.82}
\definecolor{backcolour} {rgb}{0.95,0.95,0.92}

\definecolor{intcolor}{HTML}{CA0020}
\definecolor{extcolor}{HTML}{0571B0}

\definecolor{toc} {rgb}{102, 6, 0}

\definecolor{background}{HTML}{F8F9F9}
\definecolor{border}{HTML}{A93226}

\definecolor{thered}    {rgb} {0.65,0.04,0.07}
\definecolor{thegreen}  {rgb} {0.06,0.44,0.08}
\definecolor{theblue}   {rgb} {0.02,0.04,0.48}
\definecolor{sectioning}{gray}{0.44}
\definecolor{thegrey}   {gray}{0.5}
\definecolor{theframe}  {gray}{0.75}
\definecolor{theshade}  {gray}{0.94}

\definecolor{webgreen}{rgb}{0,0.75,0}
\definecolor{webred}{rgb}{0.75,0,0}
%%%%%%%%%%%%%%%%%%%%%%%%%%%%%%%%%%%%%%%%%%%%%%%%%%%%%%%%%%%%%%%%%%%%%%%%%%%%%%%

%%%%%%%%%%%%%%%%%%%%%%%%%%%%%%%%%%%%%%%%%%%%%%%%%%%%%%%%%%%%%%%%%%%%%%%%%%%%%%%
%ΥΠΟΣΤΗΡΙΞΗ ΒΙΒΛΙΟΓΡΑΦΙΚΩΝ ΑΝΑΦΟΡΩΝ
% ΠΡΟΣΟΧΗ! η biblatex να φορτώνεται μετά το polyglossia (δες preamble-lang.tex)

\usepackage{hyphenat}
\usepackage[
  style     = ieee, %style=verbose,
  citestyle = numeric,
  bibstyle  = numeric,
  %autocite  = footnote,
  backend   = biber,
  sorting   = none,
  backref   = true,
]{biblatex}

\DefineBibliographyStrings{greek}{%
  backrefpage = {σελίδα},% originally "cited on page"
  backrefpages = {σελίδες},% originally "cited on pages"
}

% Μακροεντολή για την εκτύπωση τοπικής βιβλιογραφίας ανά κεφάλαιο. Χρησιμοποίησέ
% το όνομα της μακροεντολής (subbibforchapter) σαν επιλογή στην τοπική εκτύπωση
% βιβλιογραφίας π.χ. στο κεφάλαιο που θέλεις να εκτυπωθεί η τοπική βιβλιογραφία
% γράψε \printbibliography[heading = subbibforchapter, ...]. Η τοπική βιβλιογραφία
% θα εκτυπωθεί σαν ακόμη μία ενότητα του κεφαλαίου
% Το βρήκα εδώ https://tex.stackexchange.com/questions/49941/multiple-bibliographies-and-one-global-bibliography-all-with-global-labels
% [\refname\ \thechapter~Κεφαλαίου]

\defbibheading{subbibforchapter}[\refname]{\section{#1}}
% αν θέλεις μη αριθμημένη ενότητα χρησιμοποίησε ...{\section*{#1}}

\addbibresource{./content/mainbiblio.bib}
\addbibresource{./content/websources.bib}

\renewcommand*{\bibfont}{\small}
%%%%%%%%%%%%%%%%%%%%%%%%%%%%%%%%%%%%%%%%%%%%%%%%%%%%%%%%%%%%%%%%%%%%%%%%%%%%%%%
%%%%%%%%%%%%%%%%%%%%%%%%%%%%%%%%%%%%%%%%%%%%%%%%%%%%%%%%%%%%%%%%%%%%%%%%%%%%%%%
%% Ρυθμίσεις πακέτου hyperref

\usepackage{hyperref}
\hypersetup{
  colorlinks  = true,
  linkcolor   = Red!60!black, %intcolor, %BrickRed,
  anchorcolor = black,
  citecolor   = Green,
  filecolor   = cyan,
  menucolor   = Red!60!black,
  runcolor    = cyan,
  urlcolor    = extcolor, %NavyBlue,
%  bookmarks   = true,          % Το pdf αρχείο να έχει bookmarks
  pdfpagemode = UseOutlines,   % Όταν ανοίγει το pdf να φαίνεται η δομή του
                               % εγγράφου (κεφάλαια, ενότητες)
  pdftitle={},                 % title
  pdfauthor={},                % author
  pdfsubject={},               % subject of the document
  pdfcreator={},
}

\urlstyle{same}


\title{Βιτρούβιος μία εμβληματική μορφή της Αρχιτεκτονικής}
\author[Νικόλαος Ανδρόνικος Μαυρόπουλος]{Νικόλαος Ανδρόνικος Μαυρόπουλος}
%\date{24 January 2009}  % if the \date{} command is left out, the current date will be used

\begin{document}
              
  \maketitle% this prints the handout title, author, and date
  
  
  \bigskip            
  \begin{abstract}
  \noindent This document describes the Tufte handout \LaTeX\ document style.
  It also provides examples and comments on the style's use.  Only a brief
  overview is presented here; for a complete reference, see the sample book.
  \end{abstract}
  \bigskip
              
  
              
  

\section{Εισαγωγή}\label{sec:intro}

The Tufte-\LaTeX\ document classes define a style similar to the
style Edward Tufte uses in his books and handouts.  Tufte's style is known
for its extensive use of sidenotes, tight integration of graphics with
text, and well-set typography.  This document aims to be at once a
demonstration of the features of the Tufte-\LaTeX\ document classes
and a style guide to their use. \cite{erismis_critical_2013}

\begin{marginfigure}%
  \shadowimage[width=\linewidth]{vitruvian-man}
  \caption{\footnotesize Ο  Άνθρωπος του Βιτρούβιου (Vitruvian Man) σχεδιασμένος
  από το Λεονάρτο ντα Βίντσι
  (\href{https://en.wikipedia.org/wiki/Leonardo_da_Vinci}{Leonardo da Vinci}).
  Ένα χαρακτηριστικό έργο εφαρμογής των αρχών που περιέγραψε ο Vitruvius,
  σχεδιασμένο από ένα από τους μεγαλύτερους ζωγράφους όλων των εποχών
  (πηγή: \cite{wikipedia:vitruvianman}).}
  \label{fig:vitruvian-man}
\end{marginfigure}
  % !TEX root = ../main.tex
% !TEX encoding = UTF-8 Unicode

\section{Vitruvius}

Είναι δύσκολο να περιγράψει κανείς την προσωπικότητα του Βιτρούβιου καθώς, παρά 
τις επίμονες προσπάθειες μελετητών και ερευνητών, ο Βιτρούβιος παραμένει 
μία αινιγματική φυσιογνωμία όσον αφορά την ακριβή γνώση της ταυτότητας, του 
χρονολόγιου της ύπαρξής του και της καριέρας του. Η εικόνα που έχουμε 
σχηματίσει μέχρι σήμερα για την προσωπικότητα και τα ενδιαφέροντά του, 
προκύπτει από το έργο του \emph{De Architectura}, αλλά αυτό είναι όλο 
\cite{baldwin-1990}.  

  \section{Το έργο του} 

Το πλέον φημισμένο έργο του Βιτρούβιου και αυτό που τον έκανε γνωστό διαχρονικά είναι το \emph{"Περί Αρχιτεκτονικής"} ("De Architectura" -- DA για συντομία). Aποτελεί ένα από τα πλέον μελετημένα και μεταφρασμένα έργα της αρχαιότητας, ανεξαρτήτως θέματος. Αν και το πρωτότυπο έργο δεν έχει διασωθεί η συνέχειά του στην αρχική του μορφή (λατινικά) οφείλεται σε μία αρχική αντιγραφή του Poggio Bracciolini το 1414. Έκτοτε το συγκεκριμένο έργο έχει μεταφραστεί σε δεκάδες γλώσσες από ένα μεγάλο πλήθος μελετητών και μεταφραστών.

Σαν έργο θεωρείται ίσως το σημαντικότερο γραπτό κείμενο στην ιστορία της αρχιτεκτονικής. Σε αυτό ο Βιτρούβιος περιγράφει όχι μόνο τις αρχές της αρχιτεκτονικής αλλά και τα ποιοτικά χαρακτηριστικά που πρέπει να έχει ο αρχιτέκτονας. Συγχρόνως έχει τη μορφή εγχειριδίου που κάθε αρχιτέκτονας θα πρέπει να διαθέτει και δεν θα ήταν υπερβολή να πούμε ότι η αξία του έχει μεταφερθεί μέχρι και τις μέρες μας, σχεδόν αναλλοίωτη \cite[σ. 16-18]{vitruvius-lefas}.

\subsection{Τα Δέκα Βιβλία της Αρχιτεκτονικής}

Το DA απαρτίζεται από 10 βιβλία με αντικείμενα σχετικά με την αρχιτεκτονική τα οποία ο Βιτρούβιος έγραψε για τον αυτοκράτορα Αύγουστο. Φαίνεται πως η χρονική σειρά με την οποία έχουν γραφτεί ξεκινάει από τα βιβλία 8, 9, 10 κατά το πρώτο στάδιο της σταδιοδρομίας του και στη συνέχει τα υπόλοιπα, με κανονική σειρά, 1, 2, 3 κλπ. τα οποία γράφτηκαν κατά το τελευταίο στάδιο της ζωής του και δημοσιεύτηκαν. Τα βιβλία Περιέχουν ολοκληρωμένες και τεκμηριωμένες σκέψεις του συγγραφέα, ενώ αποτελούνται από ένα κύριο μέρος (εγχειρίδια, οικοδομικές περιγραφές και πληροφορίες πάνω σε κατασκευές) αλλά και από τους προλόγους οι οποίοι απευθύνονται στον Αύγουστο. Οι λόγοι που τον οδήγησαν να συντάξει το έργο του έγιναν με σκοπό να προσφέρει θεωρητικές αρχές πάνω στην αρχιτεκτονική, να παρέχει ενημέρωση σε αρχιτέκτονες και ιδιώτες χρηματοδότες αλλά και προσωπικής προβολής \cite{vitruvius-lefas}.

Στην εργασία ασχολούμαστε αποκλειστικά με την περιγραφή και μερική ανάλυση του Bιβλίου I. Περιληπτικά, τα υπόλοιπα βιβλία πραγματεύονται τα εξής \cite{erismis_critical_2013}:

\begin{itemize}[noitemsep]
\item Βιβλίο II υλικά οικοδόμησης και μεθόδους,
\item Βιβλία III και IV θρησκευτική αρχιτεκτονική,
\item Βιβλίο V άλλες μορφές δημόσιας αρχιτεκτονικής με έμφαση στα θέατρα,
\item Βιβλίο VI και VII οικιακή αρχιτεκτονική και θέματα όπως είδη δαπέδων, ζωγραφική, χρώματα κ.λπ.,
\item Βιβλίο VIII κατασκευή υδραγωγείων και αγωγών για μεταφορά νερού
\item Βιβλίο IX μετά από μία εκτεταμένη εισαγωγή στην Αστρονομία, περιγράφει διάφορες μορφές ρολογιών και κλίσεων
\item Βιβλίο X μηχανική με ιδιαίτερη αναφορά σε συστήματα κίνησης νερού, μέτρησης αποστάσεων οδών αλλά και στρατιωτικά μηχανήματα.
\end{itemize}

\subsection{Η Γλώσσα του Βιτρούβιου}

Ως προς τη γλώσσα και το ύφος των βιβλίων είναι κοινά αποδεκτό ότι είναι δυσνόητα ή στη καλύτερη περίπτωση δεν ακολουθούν το ύφος και το στυλ των πεζογράφων και ποιητών του πρώτου π.Χ. αιώνα (π.χ. Κικέρων). Ειδικότερα η χρησιμοποιούμενη λατινική γλώσσα αποτέλεσε εμπόδιο σε πολλές προσπάθειες μετάφρασης και ερμηνείας σε σημείο που ορισμένοι μελετητές όπως ο Frank Granger να προτείνουν "διορθώσεις" στα όχι τόσο κατανοητά μέρη  \cite[σ. 19]{vitruvius-lefas}. Σίγουρα όμως τα βιβλία είναι ενδεικτικά της τεχνικής παιδείας και γνώσεων του συγγραφέα. Χρησιμοποιεί ευρύτατα τεχνικούς όρους και εκφράσεις όπου μάλιστα πολλοί από αυτούς είναι δανεισμένοι από τους αντίστοιχους αρχαιοελληνικούς.





%Ο Βιτρούβιος είναι γνωστός για την συγγραφή του "Περί αρχιτεκτονικής" (De architectura), ένα σύνολο από δέκα συγγράμματα όπου αναλύουν και περιγράφουν τις δομές της αρχιτεκτονικής και την αξία του αρχιτέκτονα.

%Το σύγγραμμα ολοκληρώθηκε και δημοσιεύτηκε κατά την τελευταία περίοδο ζωής του αρχιτέκτονα και πιθανότατα κατά τη περίοδο όπου αυτοκράτορας ήταν πια ο Αύγουστος, υιοθετημένος ιός του Καίσαρα. ***
%Τα βιβλία αποτελούνται από τεκμηριωμένες και ολοκληρωμένες σκέψεις και παρατηρήσεις του συγγραφέα τα οποία είναι αφιερωμένα στον ίδιο τον Αυτοκράτορα. 
%Οι λόγοι που τον οδήγησαν να συγγράψει το περί αρχιτεκτονικής είναι μάλλον προς παροχή υποστήριξης στο οικοδομικό πρόγραμμα του Αύγουστου και στην ενημέρωση πάνω στην αρχιτεκτονική. (Λέφας σελ 14-15)***

%Το περιεχόμενο αφορά τους τομείς του αρχιτέκτονα (1.2-3),την εκπαίδευση και τη συμπεριφορά του (1.6), και τα ορθά χαρακτηριστικά των δημοσίων και ιδιωτικών κτισμάτων (βοοκ 5-6), Υλικά (Book 2), ορθές τοποθεσίες κτηρίων (1.4-7). (Status, Pay, Pleasure p.6)

%Στον πρόλογο του πρώτου βιβλίου αποκαλύπτει το κύριο στόχο του, είδη συλλογισμών πάνω στην αρχιτεκτονική. Στην αρχή του πρώτου βιβλίου συζητάει την γνώση του αρχιτέκτονα. έπειτα στο δεύτερο κεφάλαιο του πρώτου βιβλίου αναλύει την θεωρητική δομή της αρχιτεκτονικής. Από το τρίτο κεφάλαιο και μετά αφιερώνει το υπόλοιπο έργο του στο πρακτικό κομμάτι της αρχιτεκτονικής. (vitruvious arts of architectue p.2)

%Το πιο σημαντικό ίσως τμήμα του συγγράμματος είναι η απαιτούμενη γνώση που οφείλει να έχει ένας αρχιτέκτονας.
  \section{Τα Ποιοτικά Χαρακτηριστικά του Αρχιτέκτονα}

Από την εισαγωγή ήδη του βιβλίου, ο Βιτρούβιος αφήνει να φανεί η σημασία που αποδίδει στο θέμα της γνώσης καθώς και στη μορφή της (θεωρητική, πρακτική). Ξεκινώντας ήδη ο Βιτρούβιος κάνει σαφές ότι δεν μπορούμε να μιλάμε για αρχιτέκτονα και αρχιτεκτονικό έργο άξιο να μελετηθεί και να ασχοληθεί κάποιος με αυτό, εφόσον δεν στηρίζεται σε κάποιο επαρκές γνωσιακό υπόβαθρο. Κάνει μία εκτενή αναφορά στα διαφορετικά είδη γνώσης στα οποία θα πρέπει να επενδύσει ο αρχιτέκτονας καθώς και στην πρακτική εξάσκησή τους.

% Στο έργο του περιγράφει αναλυτικά το τεχνικό κομμάτι της αρχιτεκτονικής. Κατασκευή και περιγραφή για το πως πρέπει να κατασκευάζονται οι ναοί, δημόσιος χώρος, υλικά κλπ. Πέρα όμως από το τεχνικό χαρακτήρα και τη λειτουργία του ως εγχειρίδιο για τους αρχιτέκτονες και άλλους ενδιαφερόμενους, το \emph{Περί Αρχιτεκτονικής} κατέχει και έναν πιο θεωριτικό ρόλο στο κλάδο της επιστήμης της αρχιτεκτονικής. Ίσως, το πιο σημαντικό κομμάτι του έργου του Βιτορύβιου είναι οι θεωρίες του και απόψεις του πάνω στην αρχιτεκτονική καθ' αυτού. 

\subsection{Η γνώση του αρχιτέκτονα}

Για τον Βιτρούβιο, ένας αρχιτέκτονας οφείλει να κατέχει ένα ευρύ πεδίο γνώσεων 
ώστε το αποτέλεσμα της εργασίας του να είναι αξιομνημόνευτο. Μάλιστα αφιερώνει 
σχεδόν ολόκληρη την ενότητα πάνω στο συγκεκριμένο θέμα, πράγμα που φανερώνει 
πως είναι ιδιαίτερα σημαντικό και άξιο προς συζήτησης για τον ίδιο.

Ο αρχιτέκτονας οφείλει να είναι εφοδιασμένος με μία ποικιλία γνώσεων, που 
πρέπει να εκτείνονται από τη γεωμετρία (με την σημερινή ερμηνεία της λέξης) 
μέχρι και ιστορία, φιλοσοφία, μουσική, αλλά και ιατρική, νομική και αστρονομία 
\cite[σ. 392]{masterson_status_2004}. Αξιολογώντας τη διαχρονική αξία αυτής της 
θέσης\sidenote%
    {Σήμερα πιθανώς να φαίνεται παράδοξο, ίσως και ασυμβίβαστο η   κατοχή 
    γνώσεων μουσικής, ιατρικής κ.λπ. παράλληλα με τις καθαρά τεχνικές γνώσεις 
    τη αρχιτεκτονικής, καθώς μπορεί κάποιος να αναρωτηθεί για παράδειγμα, σε τι 
    μπορεί να συμβάλλει η εξοικείωση με
    την επιστήμη της αστρονομίας πάνω στη σχεδίαση μίας
    τυπικής κατοικίας.},
επικαλούμαστε ξανά την επιχειρηματολογία του Βιτρούβιου: με την αστρονομία, ο 
αρχιτέκτονας μαθαίνει να προσανατολίζεται, γνωρίζει που βρίσκεται ο Βοράς, ο 
Νότος, η Δύση και η Ανατολή αλλά και αντιλαμβάνεται τις ιδιότητες των σημείων 
αυτών του ορίζοντα και την επίδραση που έχει η σωστή τοποθέτηση και φορά ενός 
κτηρίου σε σχέση με αυτά \cite[σ. 43]{vitruvius-lefas}. Γενικότερα υποστηρίζει 
πως όλες οι επιστήμες και οι κλάδοι τους έχουν μία κοινή σχέση μεταξύ τους την 
οποία αποδίδει με τον αρχαιοελληνικό όρο της \emph{εγκυκλίου παιδείας} 
(encyclios disciplina). Ο όρος έχει να κάνει ακριβώς με αυτή την απαραίτητη 
πολυμαθεία του αρχιτέκτονα. Όπως λέει, μόνο η επιστήμη της αρχιτεκτονικής 
περιβάλλεται από τόσες άλλες επιστήμες, καθιστώντας έτσι τον αρχιτέκτονα ικανό 
να φτάσει στο ανώτερο επίπεδο της τέχνης του που είναι το \emph{κατασκευάζειν} 
\cite[σ. 45]{vitruvius-lefas}.

Φυσικά ο Βιτρούβιος δεν αναμένει από κανέναν να αριστεύει σε όλους τους τομείς 
των επιστημών, καθώς πιθανώς είναι πρακτικά αδύνατο. Δεν είναι δυνατόν ούτε 
μπορεί ένας αρχιτέκτονας, όπως αναφέρει, να είναι φιλόσοφος όπως ο Αρίσταρχος, 
μουσικός όπως ο Αριστόξενος ή ιατρός όπως ο Ιπποκράτης, καθώς κανένας άνθρωπος 
δεν μπορεί να είναι αυθεντία σε όλες τις τέχνες και επιστήμες, να κατέχει 
δηλαδή σε βάθος τη θεωρία τους. Ο Βιτρούβιος αντίθετα, ζητάει από τον 
αρχιτέκτονα μία τουλάχιστον εξοικείωση σε αυτούς τους τομείς της τέχνης και των 
επιστημών \cite[σ.~45]{vitruvius-lefas}. 

\subsection{Θεωρία και πράξη}

Το χαρακτηριστικό που επιτρέπει στον  ιδανικό αρχιτέκτονα να ξεχωρίσει από έναν κοινό τεχνίτη είναι η υπεροχή του στα γράμματα \cite{masterson_status_2004}. Φυσικά το επάγγελμα του τεχνίτη είναι εξίσου σημαντικό αλλά για τον Βιτρούβιο, η διαφορά είναι πως ο αρχιτέκτονας δεν βασίζεται αποκλειστικά στο τεχνικό, πρακτικό κομμάτι της κατασκευής αλλά στηρίζεται και πάνω σε ένα θεωρητικό υπόβαθρο. Είναι χαρακτηριστική η αναφορά του Βιτρούβιου:

\begin{quote}
{\itshape "Οι αρχιτέκτονες, λοιπόν, που χωρίς γράμματα αρκέστηκαν στην πρακτική 
εξάσκηση,  δεν κατόρθωσαν να δημιουργήσουν [έργα] με κύρος ανάλογο του μόχθου 
που κατέβαλαν. Αυτοί που εμπιστεύθηκαν αποκλειστικά την θεωρία και τα γράμματα 
κυνήγησαν, όπως φαίνεται, όχι τα ίδια τα πράγματα, αλλά τη σκιά τους. Όσοι όμως 
κατείχαν και τα δύο επέτυχαν..."}
\begin{flushright}
\footnotesize{--- Παύλος Λέφας, Vitruvii De architectura (σ. 37)}
\end{flushright}
\end{quote}

Χρησιμοποιεί μάλιστα ειδικούς όρους για να περιγράψει τις δύο αυτές 
διαφορετικές μορφές γνώσης: \emph{Ratiocinatio} (Θεωρία) και \emph{Fabrica} 
(Πράξη), τις οποίες θα αναλύσουμε στη συνέχεια. Θα πρέπει όμως πρώτα να 
διευκρινίσουμε τα εξής:

\begin{itemize}
\item Οι συγκεκριμένοι όροι, όπως και πολλοί άλλοι που χρησιμοποιεί, δεν έχουν 
απαραίτητα την ίδια έννοια που έχουν στη σύγχρονη γλώσσα. Σαφώς, έχοντας υπόψη 
μας την περίοδο που έζησε ο Βιτρούβιος αλλά και τις επιρροές του από τον 
αρχαιοελληνικό πολιτισμό, οι μελετητές μπορούν εώς ένα σημείο μόνο να 
μεταφράσουν και να αναλύσουν άμεσα την ορολογία του Βιτρούβιου, ενώ σε πολλές 
περιπτώσεις \sidenote%
    {Όπως ήδη αναφέρθηκε, η γλώσσα του Βιτρούβιου είναι σχετικά περίπλοκη ίσως και δυσνόητη, με αποτέλεσμα οι μεταφράσεις να έχουν δημιουργήσει μία σχετική σύγχυση για την πραγματική ουσία του κειμένου του συγγραφέα. Οι όροι και οι έννοιες που χρησιμοποιεί και η τοποθέτησή τους στο κείμενο, είναι αντικείμενο έντονης συζήτησης μεταξύ των μελετητών ως προς την εννοιολογική τους ερμηνεία.}
χρειάζεται να βασίσουν την ερμηνεία και τις αναλύσεις τους, ανατρέχοντας στις 
γλωσσικές ρίζες των όρων που χρησιμοποιεί. Με αυτό τον τρόπο επιδιώκεται η 
αποκρυπτογράφηση του βαθύτερου νοήματος που προσπαθεί να μεταφέρει ο Βιτρούβιος 
\cite{graham-education}.
\item Η άποψη του Βιτρούβιου περί θεωρίας και πράξης και η σύνδεσή τους με την 
ιδιότητα του αρχιτέκτονα, φαίνεται να είναι πολύ κοντά στη σχετική Πλατωνική 
ερμηνεία. Για τους αρχαίους Έλληνες δεν υπήρχε ιδιαίτερος ορισμός για την 
ιδιότητα του αρχιτέκτονα, που να τον ξεχωρίζει από αυτή του τεχνίτη. Για τον 
Πλάτωνα λοιπόν, ο αρχιτέκτονας δεν είναι ένας τυπικός τεχνίτης αλλά ένα ορθά 
μελετημένο άτομο το οποίο κατέχει το πρακτικό κομμάτι του χτισίματος αλλά και 
γνώσεις πάνω στις επιστήμες, δηλαδή το θεωρητικό υπόβαθρο. Η \emph{πρακτική} 
και \emph{η γνωστική} του Πλάτωνα συνθέτουν την ίδια την επιστήμη. 
\cite{graham-education}
\end{itemize}

\begin{description}[style=nextline]
\item[Fabrica]

Η λέξη Fabrica ερμηνεύεται γενικά ως \emph{τέχνη} αλλά με τη γενική έννοια της 
\emph{Πράξης} \cite{vitruvius-lefas,graham-education}. Η έννοια δηλαδή της 
επαναλαμβανόμενης άσκησης του χεριού ή η ενασχόληση με τα πράγματα. Στην ουσία 
είναι η αρχιτεκτονική ενατένιση, το πρακτικό κομμάτι του "ξέρω πως", δηλαδή 
έχει να κάνει με την ανάλυση του χώρου, τη χωροταξία, κλπ. Με άλλα λόγια 
περιλαμβάνει την επαγγελματική γνώση του αρχιτέκτονα \cite{graham-education}. Ο 
Βιτρούβιος όμως, περιγράφοντας τη έννοια της fabrica, της δίνει και μία 
ιδιαίτερη σημασία, αυτό της πνευματικής ενασχόλησης. Όντως την συνδέει με τη 
λέξη \emph{meditatio}, η οποία ερμηνεύεται ως μία διαδικασία μελέτης, 
περισυλλογής ή σκέψης. Οπότε μέσω της διαδικασίας του meditatio αποκτάται η 
απαιτούμενη επαγγελματική γνώση και εμπειρία του αρχιτέκτονα. Αυτή η προσέγγιση 
είναι πολύ κοντά στη σχετική άποψη του Πλάτωνα\sidenote%
{Τόσο ο Πλάτωνας όσο και ο Βιτρούβιος, ερμηνεύουν τη πνευματική συνιστώσα της 
πρακτικής γνώσης μέσα από το παράδειγμα της μουσικής. Η μουσική σαν πράξη 
αποτελείται τόσο από το πρακτικό κομμάτι δηλαδή τη διαδικασία παραγωγής του 
ήχου μέσα από την η ενασχόληση με το μουσικό όργανο, όσο και την θεωρητική 
σύλληψη της μουσικής σύνθεσης, που είναι ένα καθαρά θεωρητικό κομμάτι 
σχετιζόμενο με τα μαθηματικά, την αρμονία και τη μεταφυσική.}
, ο οποίος αν και περιγράφει την πράξη σαν ένα είδος πρακτικής γνώσης η οποία είναι περισσότερο επιτακτική ή επαγγελματική παρά αποκλειστικά προϊόν πνευματικής νόησης (φιλοσοφική, επιστημονική ή μαθηματική), εντούτοις η \emph{πρακτική} και \emph{γνωστική} του αρχιτέκτονα είναι και οι δύο είδη πνευματικής γνώσης \cite{graham-education}.  


\item[Ratiocinatio]

Η λέξη ratiocinatio ερμηνεύεται ως \emph{Θεωρία}, με την έννοια του υπολογισμού, του σχεδιασμού. Σε αντίθεση με τον όρο fabrica, που αν και έχει μία πνευματική συνιστώσα (meditaio), υπονοεί κατά βάση τη χειρονακτική ενασχόληση με τα πράγματα , το ratiοcinatio αποτελεί μία καθαρά πνευματική διαδικασία \cite{vitruvius-lefas}.

Το ratiocinatio έχει ως βάση τη λέξη \emph{ratio} που μεταφράζεται ως 
\emph{λόγος}. Στην αρχαιοελληνική γραμματική η συγκεκριμένη λέξη έχει 
περισσότερες από μία έννοιες, όπως θεωρία, περιγραφή ή λόγος με την μαθηματική 
έννοια δηλαδή η σχέση μεταξύ δύο αριθμών. Ιδιαίτερα μάλιστα η τελευταία αυτή 
ερμηνεία του λόγου είναι κοινή με το λατινικό \emph{ratio} και το Σωκρατικό 
παράγωγό τους, της \emph{λογικής} (rational). Το Ratiocinatio είναι η κατανόηση 
της δημιουργίας των πραγμάτων με γνώση και δεξιότητες: πως μπορεί να 
παρουσιασθεί αλλά και να ερμηνευτεί η ορθολογική αναλογία 
\cite{patterson-1997}.  Βέβαια η σύνδεση αυτή της θεωρίας με τη λογική έχει 
προκαλέσει πολλά προβλήματα ερμηνείας καθώς πολλοί μεταφραστές αδυνατούν να 
καταλήξουν σε ένα κοινό συμπέρασμα για την έννοια της λέξης ratiocinatio, 
αποδίδοντας διαφορετικές ερμηνείες. Είναι χαρακτηριστικό ότι η φράση:

\begin{quote}
\itshape
"Ratiocinatio autem est quae resfabricates sollertiae acrationis pro portione (proportione) demonstrare atque explicare potest"
\end{quote}

δυσκολεύει ακόμα τους μελετητές \cite{graham-education}. Επικρατέστερη λοιπόν φαίνεται να είναι η ερμηνεία του όρου ως μαθηματική αναλογία κάτι που έμμεσα υποστηρίζεται και από τα έργα μεταγενέστερων καλλιτεχνών, εμπνευσμένων από το  Βιτρούβιο και ειδικά το DA. Κλασσικό παράδειγμα είναι το γνωστό έργο του \href{https://en.wikipedia.org/wiki/Leonardo_da_Vinci}{Leonardo da Vinci}, \href{https://en.wikipedia.org/wiki/Vitruvian_Man}{\emph{"Vitruvian Man"}} (Σχ. \ref{fig:vitruvian-man}) στο οποίο απεικονίζονται οι ιδανικές ανθρώπινες αναλογίες. Ο da Vinci δηλώνει ρητά ότι το συγκεκριμένο σχέδιο είναι επηρεασμένο από τη θεωρία του Βιτρούβιου για τις αναλογίες του ανθρώπινου σώματος (βιβλίο ΙΙΙ).
\end{description}


\begin{marginfigure}%
  \shadowimage[width=\linewidth]{vitruvian-man}
  \caption{\footnotesize Ο  Άνθρωπος του Βιτρούβιου (Vitruvian Man) σχεδιασμένος
  από το Λεονάρντο ντα Βίντσι. Ένα χαρακτηριστικό έργο εφαρμογής των αρχών που 
  περιέγραψε ο Vitruvius, σχεδιασμένο από ένα από τους μεγαλύτερους ζωγράφους 
  όλων των εποχών
  (πηγή: \cite{wikipedia:vitruvianman}).}
  \label{fig:vitruvian-man}
\end{marginfigure}
  \section{Αρχές της Αρχιτεκτονικής}

Συνεχίζοντας στο πρώτο βιβλίο και συγκεκριμένα στη δεύτερη ενότητα, ο Βιτρούβιος εισάγει έναν \emph{κανόνα} για τις αρχές της αρχιτεκτονικής(ή συνιστώσες \cite{vitruvius-lefas}). Πρόκειται για ένα σύνολο από "χαρακτηριστικά" ή "ιδιότητες" που τις ονομάζει:

\begin{itemize}[noitemsep]
\item Τάξις (Ordinatio)
\item Διάθεσης (Dispositio)
\item Ευρυθμία (Eurythmy)
\item Συμμετρία (Symmetria) 
\item Κοσμιότητα (Decor) 
\item Οικονομία (Distributio)
\end{itemize}

Φαίνεται πως η σειρά με την οποία αναλύει τις έξι αυτές συνιστώσες γίνεται με 
βάση το βαθμό "τεχνικότητας, όπου οι Τάξις, Διάθεσης και Ευρυθμία, αναφέρονται 
στη θεωρητική σύλληψη/πλαίσιο του έργου, ενώ οι Συμμετρία, Κοσμιότητα και 
Οικονομία έχουν περισσότερο τεχνικό χαρακτήρα \cite{vitruvius-lefas}. 
Παρατηρούμε δηλαδή ότι ο Βιτρούβιος κάνει σαφή διαχωρισμό μεταξύ ποιοτικών και 
τεχνικών χαρακτηριστικών χωρίς να εξαιρεί κανένα και χωρίς να υποτιμά κανένα. 
Οι αρχές του αφορούν τόσο το αισθητικό κομμάτι της επιστήμης (δηλαδή το ίδιο το 
έργο τέχνης, το κτήριο), όσο και την ίδια την τέχνη του αρχιτέκτονα (την πράξη, 
τις ενέργειες που εκτελεί και την ενασχόληση με το έργο του). Πιο 
αναλυτικά\cite{vitruvius-lefas,lefas-fundamental}: 
 
\begin{description}[style=nextline]
\item[Ordinatio]
  
Ο όρος \emph{τάξις} έχει την έννοια της ιεραρχημένης σύνθεσης δηλαδή τη ορθή αλλά και με κάποιας μορφής διαβάθμισης, τοποθέτηση των πραγμάτων, βάση ενός κοινού μέτρου. Ο Βιτρούβιος συνδέει τη συγκεκριμένη λέξη και με κάποιος επιπλέον τους όρους:

\begin{itemize}[noitemsep]
\item \emph{commoditas} και \emph{modica} οι οποίες έχουν ως βάση το μέτρο 
(modus), σημαίνουν "σύμμετρο" ή "με μέτρο", υπονοώντας πως η λέξη τάξις 
συνδέεται με την συνιστώσα της συμμετρίας. Οι δύο νέοι αυτοί όροι ίσως 
χρησιμοποιούνται με την Ευκλείδεια μαθηματική έννοια, δηλαδή τη σχέση μεταξύ 
μεγεθών, καθώς σε ένα κτίσμα για παράδειγμα, τα μέλη του (ή τα μέρη των 
επιμέρους μελών) πρέπει να έχουν μία "σωστή" θέση μεταξύ τους (σχέση μεγεθών), 
βάση ενός κοινού μέτρου.

%\cite[σ.~186]{}

\item \emph{comparatio}. Η λέξη αυτή έχει διπλή σημασία. Σημαίνει κατασκευή, 
σύνθεση και παράλληλα σύγκριση. Στην ουσία σε αυτό που θέλει να καταλήξει ο 
ίδιος είναι πως τα μεγέθη πρέπει να έχουν μία ιεράρχηση, μία σχέση μεταξύ τους 
η οποία να είναι μεν φανερή (συγκρίσιμη) αλλά να μην επικαλύπτει το ένα το 
άλλο. Παρατηρούμε δηλαδή ότι καταλήγει και πάλι στην έννοια της ισορροπίας των 
μελών ενός συνόλου. Βέβαια αυτή η ιεράρχηση βασίζεται στο "μέγεθος" του κάθε 
μέλους και δεν έχει να κάνει με την αριθμητική τους σχέση. 
\end{itemize}

%\cite[σ.~93,188]{vitruvius-lefas,lefas-fundamental}

Η τάξις επιτυγχάνεται μέσω της ποσότητας, η οποία έχει να κάνει με τον ορισμό ενός κοινού μέτρου. Μέσω αυτού του κοινού μέτρου αντιλαμβανόμαστε τα μέλη του κτίσματος, έχοντας έτσι ένα αρμονικό σύνολο. 

\item[Dispositio]
  
Η διάθεσης στην ουσία σημαίνει "διάταξη" και έχει ως έννοια την διαδικασία της 
ορθής τοποθέτησης των πραγμάτων στο χώρο, της \emph{διευθέτησης} (Granger και 
Morgan στο \cite[σ.~495]{scranton_vitruvius_1974}). Συνδέει επίσης το 
συγκεκριμένο με τον όρο \emph{conlocatio}\sidenote%
    {Τόσο η λέξη dispositio και conlocatio 
    έχουν κατάληξη -tio, -tionis, που δηλώνει
    πρωτίστως την εκτέλεση μίας διαδικασίας και
    έπειτα το αποτέλεσμα μίας διαδικασίας, πράγμα
    που σημαίνει πως υπάρχει μέσα στην λέξη
    διάθεσης, η έννοια της ενέργειας, πράξης
    με καθορισμένη τάξη που αφορά σε έναν σκοπό \cite[σ.~496]{scranton_vitruvius_1974}.}
, που σημαίνει "τοποθέτηση των πραγμάτων". 

Ο Βιτρούβιος αναφέρει πως η Διάθεσης εμφανίζεται στις αρχαιοελληνικές λέξεις \emph{ιχνογραφία}, \emph{ορθογραφία} και \emph{σκηνογραφία}, που σημαίνουν αντίστοιχα κάτοψη, όψη και αναπαραστατικό σχέδιο. Τονίζει πως οι έννοιες αυτές είναι ένα μέσο προς επίτευξη της Διάθεσης, δηλαδή συνδέονται με τη διαδικασία του σχεδιασμού και δεν είναι τα ίδια τα σχέδια. 
  
\item[Eurytmia]
  
Παρατηρούμε ότι χρησιμοποιεί αυτούσιο τον αρχαιοελληνικό όρο και όχι κάποιον 
λατινικό και μάλιστα με την πρωταρχική του σημασία, \emph{ευ-ρυθμός} δηλαδή 
ποιοτικός, καλός ρυθμός στην αρχιτεκτονική, ο οποίος πρέπει αν είναι ευχάριστος 
στις αισθήσεις. Είναι η όμορφη όψη και η ισορροπημένη εμφάνιση των μελών του 
συνόλου. Αναφέρει πως επιτυγχάνεται με την ανταπόκριση των μεγεθών του έργου, 
δηλαδή τις αναλογίες του ύψους, μήκους και πλάτους, με την συμμετρία (πάλι ίσως 
με την πιο γενική της έννοια, δηλαδή ισόρροπη) 
\cite[σ.~498]{scranton_vitruvius_1974}. 

\item[Symetria]

Ίσως η θεμελιώδης αρχή της θεωρίας του πάνω στις συνιστώσες της αρχιτεκτονικής. 
Δεν έχει την κοινή και γνωστή έννοια της Ευκλείδειας συμμετρίας αλλά εκφράζει 
την συμφωνία και την εναρμόνιση των υπομέρους τμημάτων ενός συνόλου σε σχέση με 
το ίδιο. Όπως η τάξις και η ευρυθμία έτσι και η συμμετρία στηρίζεται πάνω στην 
έννοια του μέτρου (Πλατωνική αντίληψη της συμμετρίας) και συγχρόνως είναι 
αποτέλεσμα αναλογικών και αριθμητικών σχέσεων. Συγκεκριμένα, έχει τις βάσεις 
της στις αναλογίες του ανθρώπινου σώματος του οποίου τα μέλη εκ φύσεως έχουν 
μία αναλογική σχέση μεταξύ τους. Όπως αναφέρει, ο πήχης, η παλάμη, το δάκτυλο, 
ολόκληρο το χέρι, καθιστούν εύρυθμο το ανθρώπινο σώμα, έτσι και σε ένα κτίσμα 
τα μέλη και τα μεγέθη του πρέπει να έχουν μία αναλογία μεταξύ τους ώστε να 
επιτυγχάνεται η εναρμόνιση τους με το σύνολο.

Η συμμετρία έχει έναν πιο "τεχνικό" χαρακτήρα. Συνδέεται άμεσα με την τάξις και 
τη ευρυθμία καθώς αποτελεί απαραίτητη προϋπόθεση της ύπαρξης τους. Συγχρόνως 
όμως είναι ανεξάρτητη από αυτές. Όπως ακριβώς συμμετρία (δηλαδή οι αριθμητικές 
αναλογίες) παρατηρείται στο ανθρώπινο σώμα, το ίδιο θα πρέπει να συμβαίνει και 
στα κτήρια κατά την διαδικασία του σχεδιασμού και του χτισίματος. Εδώ ο 
Βιτρούβιος προκειμένου να εξηγήσει την έννοια της συμμετρίας, παρουσιάζει το 
παράδειγμα του ναού, του οποίου τα μέρη, όπως ο κίονας και η τρίγλυφος, είναι 
εναρμονισμένα μεταξύ τους. Παρόμοια παραδείγματα επικαλείται και σε άλλα σημεία 
του DA, όπως αυτό της \emph{Δωρικής θύρας} (βιβλίο VI).

\item[Decor]
  
Η κοσμιότητα έχει να κάνει με την "αισθητική" αλλά με την έννοια του 
"καθωσπρέπει" και της "αρμόζουσας" σχέσης του κτίσματος σε σχέση με 
κοινωνικούς, ιστορικούς ή φυσικούς παράγοντες. Μιλώντας για κοσμιότητα ο 
Βιτρούβιος επιχειρεί να προσδώσει έναν χαρακτηρισμό "καταλληλότητας" στο κτίσμα 
έχοντας υπόψη του τη θεματολογία του. Παρουσιάζει δε ορισμένα παραδείγματα 
εμφάνισης του στοιχείου της κοσμιότητας όπως την επιλογή του αρμόζοντα ρυθμού 
(κορινθιακού, δωρικού κ.λπ.) στους ναούς ή όταν ένα κτήριο επωφελείται άψογα 
από τον προσανατολισμό του \cite{scranton_vitruvius_1974}. 

Γιά τον Βιτρούβιο η κοσμιότητας θα πρέπει να συνδυάζει τα εξής τρία στοιχεία:

\begin{enumerate}[noitemsep]
  \item \emph{Θεματισμό}, όταν το έργο ανταποκρίνεται στο στο "θέμα" του.
  \item \emph{Έθει}, κοσμιότητα που έχει να κάνει με την τήρηση της παράδοσης.
  \item \emph{Φύσει}, έχει να κάνει με την ένταξη του έργου στη φύση.
\end{enumerate}

Αν θέλουμε να απλοποιήσουμε την συγκεκριμένη έννοια υποθέτουμε ότι ο Βιτρούβιος υπονοεί αυτό που σήμερα αντιλαμβανόμαστε σαν "εντός του θέματος".
  
\item[Distributio]

Με το συγκεκριμένο όρο\sidenote%
    {Και αυτή η λέξη έχει κατάληξη -tio, -tionis, όπως και η dispositio, δηλώνοντας και αυτή μία ενέργεια, μία διαδικασία.} 
, ο Βιτρούβιος εννοεί τη συνολική προσέγγιση του αρχιτέκτονα, στο έργο. Θα 
πρέπει να προγραμματίσει κατάλληλα τόσο τη σχεδίαση όσο και την κατασκευή του, 
τηρώντας ταυτόχρονα ένα λογικό προϋπολογισμό κόστους. Στην προσπάθεια αυτή θα 
πρέπει να έχει υπόψη του την τοποθεσία του έργου καθώς και την οικονομική άνεση 
και το κύρος πελάτη. Επίσης η διαδικασία αυτή θα πρέπει να τηρείται όταν οι 
ανάγκες του έργου βρίσκονται εντός λογικών πλαισίων. 

Η οικονομία παραπέμπει στη έννοια της κοσμιότητας καθώς και στις δύο 
περιπτώσεις ο Βιτρούβιος μιλάει περί ένταξης του έργου εντός ορισμένων 
πλαισίων. Η κοσμιότητα αφορά τη θεματολογία του έργου ενώ η οικονομία έχει να 
κάνει περισσότερο με τον ίδιο τον πελάτη. Είναι χαρακτηριστικό ότι ο Βιτρούβιος 
δίνει ειδική σημασία στις επιθυμίες του πελάτη, όπως φαίνεται και σε άλλα 
σημεία του DA \cite{scranton_vitruvius_1974}.

\end{description}
  \section{Συμπεράσματα}
  
  
 % \begin{fullwidth}
    \printbibliography[title={Βιβλιογραφικές αναφορές}]
%  \end{fullwidth}
              
              
\end{document}
