%%%%%%%%%%%%%%%%%%%%%%%%%%%%%%%%%%%%%%%%%%%%%%%%%%%%%%%%%%%%%%%%%%%%%%%%%%%%%%%
%ΥΠΟΣΤΗΡΙΞΗ ΒΙΒΛΙΟΓΡΑΦΙΚΩΝ ΑΝΑΦΟΡΩΝ
% ΠΡΟΣΟΧΗ! η biblatex να φορτώνεται μετά το polyglossia (δες preamble-lang.tex)

\usepackage{hyphenat}
\usepackage[
  style     = ieee, %style=verbose,
  citestyle = numeric,
  bibstyle  = numeric,
  %autocite  = footnote,
  backend   = biber,
  sorting   = none,
  backref   = true,
]{biblatex}

\DefineBibliographyStrings{greek}{%
  backrefpage = {σελίδα},% originally "cited on page"
  backrefpages = {σελίδες},% originally "cited on pages"
}

% Μακροεντολή για την εκτύπωση τοπικής βιβλιογραφίας ανά κεφάλαιο. Χρησιμοποίησέ
% το όνομα της μακροεντολής (subbibforchapter) σαν επιλογή στην τοπική εκτύπωση
% βιβλιογραφίας π.χ. στο κεφάλαιο που θέλεις να εκτυπωθεί η τοπική βιβλιογραφία
% γράψε \printbibliography[heading = subbibforchapter, ...]. Η τοπική βιβλιογραφία
% θα εκτυπωθεί σαν ακόμη μία ενότητα του κεφαλαίου
% Το βρήκα εδώ https://tex.stackexchange.com/questions/49941/multiple-bibliographies-and-one-global-bibliography-all-with-global-labels
% [\refname\ \thechapter~Κεφαλαίου]

\defbibheading{subbibforchapter}[\refname]{\section{#1}}
% αν θέλεις μη αριθμημένη ενότητα χρησιμοποίησε ...{\section*{#1}}

\addbibresource{./content/mainbiblio.bib}
%\addbibresource{./content/websources.bib}

\renewcommand*{\bibfont}{\small}
%%%%%%%%%%%%%%%%%%%%%%%%%%%%%%%%%%%%%%%%%%%%%%%%%%%%%%%%%%%%%%%%%%%%%%%%%%%%%%%