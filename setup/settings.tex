


%%%%%%%%%%%%%%%%%%%%%%%%%%%%%%%%%%%%%%%%%%%%%%%%%%%%%%%%%%%%%%%%%%%%%%%%%%%%%%%
% Το πακέτο tufte 'εχει πρόβλημα όταν χρησιμοποιείται με xelatex. Συγκεκριμένα
% χαλάει η σχεδίαση για την απεικόνιση κεφαλαίων χαρακτήρων και ειδικότερα
% η εντολή /MakeTextUppercase (ορίζεται στην κλάση tufte στο tufte-common.def).
% Ο παρακάτω κώδικας λύνει το πρόβλημα (το βρήκα εδώ:
% https://tex.stackexchange.com/questions/202142/problems-compiling-tufte-title-page-in-xelatex)
\usepackage{ifxetex}
\ifxetex
  \newcommand{\textls}[2][5]{%
    \begingroup\addfontfeatures{LetterSpace=#1}#2\endgroup
  }
  \renewcommand{\allcapsspacing}[1]{\textls[15]{#1}}
  \renewcommand{\smallcapsspacing}[1]{\textls[10]{#1}}
  \renewcommand{\allcaps}[1]{\textls[15]{\MakeTextUppercase{#1}}}
  \renewcommand{\smallcaps}[1]{\smallcapsspacing{\scshape\MakeTextLowercase{#1}}}
  \renewcommand{\textsc}[1]{\smallcapsspacing{\textsmallcaps{#1}}}
  \usepackage{fontspec}
\fi
%%%%%%%%%%%%%%%%%%%%%%%%%%%%%%%%%%%%%%%%%%%%%%%%%%%%%%%%%%%%%%%%%%%%%%%%%%%%%%%

    \usepackage{amsmath}
     
    % Set up the images/graphics package
    \usepackage{graphicx}
       \setkeys{Gin}{width=\linewidth,totalheight=\textheight,keepaspectratio}
       \graphicspath{{img/}}
                 
    
                 
    % The following package makes prettier tables.  We're all about the bling!
    \usepackage{booktabs}
                 
    % The fancyvrb package lets us customize the formatting of verbatim
    % environments.  We use a slightly smaller font.
    \usepackage{fancyvrb}
       \fvset{fontsize=\normalsize}
                 
    % Small sections of multiple columns
    \usepackage{multicol}
      
    % Ρυθμίσεις γλώσσας (ελληνικά)
    \usepackage{polyglossia}        % Μηχανή στοιχειοθεσίας πολλών γλωσσών
        \defaultfontfeatures{Mapping=tex-text}
    
    \usepackage{fontspec}     % Επιλογή και ρύθμιση γραμματοσειρών
        \setdefaultlanguage[variant=modern]{greek}  % Επιλογή βασικής γλώσσας κειμένου
        \setotherlanguages{english}                % Επιλογή δευτερεύουσας γλώσσας κειμένου
  %\newfontfamily\greekfont{Georgia Pro}        % Ορισμός γραμματοσειράς που θα
  % χρησιμοποιήσω στη συνέχεια. Έχουν smallcaps: GFS Artemisia, GFS Elpis,
  % GFS Didot (πολύ καλή), GFS Bodoni (πολύ καλή), GFS Neohellenic (καλή)
  %\newfontfamily\greekfontsc{GFS Bodoni}
 % \newfontfamily\greekfontsf{Arial} % Sans Serif γραμματοσειρά
        \setmainfont{Arial}[Script=Greek]   % Η βασική γραμματοσειρά του κειμένου
        \setsansfont{Arial}[Script=Greek] % Η Sans γραμματοσειρά
        \setmonofont{Arial}[Script=Greek]